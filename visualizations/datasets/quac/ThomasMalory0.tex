\documentclass[11pt,a4paper, onecolumn]{article}
\usepackage{times}
\usepackage{latexsym}
\usepackage{url}
\usepackage{textcomp}
\usepackage{bbm}
\usepackage{amsmath}
\usepackage{booktabs}
\usepackage{tabularx}
\usepackage{graphicx}
\usepackage{dialogue}
\usepackage{mathtools}
\usepackage{hyperref}
%\hypersetup{draft}

\usepackage{multirow}
\usepackage{mdframed}
\usepackage{tcolorbox}

\usepackage{xcolor,pifont}
%\newcommand{\cmark}{\ding{51}}
%\newcommand{\xmark}{\ding{55}}

\setcounter{topnumber}{2}
\setcounter{bottomnumber}{2}
\setcounter{totalnumber}{4}
\renewcommand{\topfraction}{0.75}
\renewcommand{\bottomfraction}{0.75}
\renewcommand{\textfraction}{0.05}
\renewcommand{\floatpagefraction}{0.6}

\newcommand\cmark {\textcolor{green}{\ding{52}}}
\newcommand\xmark {\textcolor{red}{\ding{55}}}
\mdfdefinestyle{dialogue}{
    backgroundcolor=yellow!20,
    innermargin=5pt
}
\usepackage{amssymb}
\usepackage{soul}
\makeatletter

\begin{document}

\hspace{15pt}{\textbf{Section}:Thomas Malory0\\}
\\ Context: Most of what is known about Malory stems from the accounts describing him in the prayers found in the Winchester Manuscript. He is described as a ''knyght presoner'', distinguishing him from the other six individuals also bearing the name Thomas Malory in the 15th century when Le Morte d'Arthur was written. At the end of the ''Tale of King Arthur'' (Books I-IV in the printing by William Caxton) is written: ''For this was written by a knight prisoner Thomas Malleorre, that God send him good recovery.'' At the end of ''The Tale of Sir Gareth'' (Caxton's Book VII): ''And I pray you all that readeth this tale to pray for him that this wrote, that God send him good deliverance soon and hastily.'' At the conclusion of the ''Tale of Sir Tristram'' (Caxton's VIII-XII): ''Here endeth the second book of Sir Tristram de Lyones, which was drawn out of the French by Sir Thomas Malleorre, knight, as Jesu be his help.'' Finally, at the conclusion of the whole book: ''The Most Piteous Tale of the Morte Arthure Sanz Gwerdon par le shyvalere Sir Thomas Malleorre, knight, Jesu aide ly pur votre bon mercy.'' However, all these are replaced by Caxton with a final colophon reading: ''I pray you all gentlemen and gentlewomen that readeth this book of Arthur and his knights, from the beginning to the ending, pray for me while I am alive, that God send me good deliverance and when I am dead, I pray you all pray for my soul. For this book was ended the ninth year of the reign of King Edward the Fourth by Sir Thomas Maleore, knight, as Jesu help him for his great might, as he is the servant of Jesu both day and night.'' It should be noted that, with the exception of the first sentence of the final colophon, all the above references to Thomas Malory as a knight are, grammatically speaking, in the third person singular, which leaves open the possibility that they were added by a copyist, either in Caxton's workshop or elsewhere. However, scholarly consensus, as has been previously mentioned in this article, is that these references to knighthood refer to a real person and that that person is the author of Le Morte D'Arthur. The author was educated, as some of his material ''was drawn out of the French,'' which suggests that he might have been from a wealthy family. A claimant's age must also fit the time of writing. CANNOTANSWER

\begin{figure}[t] \small \begin{tcolorbox}[boxsep=0pt,left=5pt,right=0pt,top=2pt,colback = yellow!5] \begin{dialogue}
 \small 
 \speak{Student}{\bf what is his identity }
\speak{Teacher}\colorbox{pink!25}{$\hookrightarrow$}
{ ``'' (Most of what is known about Malory stems from the accounts describing him in the prayers found in the Winchester Manuscript. ) }
\\
\speak{Student}{\bf how is he described }
\speak{Teacher}\colorbox{pink!25}{$\hookrightarrow$}
{ ``'' (He is described as a ''knyght presoner'', distinguishing him from the other six individuals also bearing the name Thomas Malory ) }
\\
\speak{Student}{\bf anything else interesting }
\speak{Teacher}\colorbox{pink!25}{$\hookrightarrow$}
{ ``'' (''For this was written by a knight prisoner Thomas Malleorre, that God send him good recovery.'' ) }
\\
\speak{Student}{\bf did he write anything else }
\speak{Teacher}\colorbox{pink!25}{$\hookrightarrow$}
\colorbox{red!25}{Yes,}
{ ``'' (And I pray you all that readeth this tale to pray for him that this wrote, that God send him good deliverance soon and hastily. ) }
\\
\speak{Student}{\bf how were his books recieved }
\speak{Teacher}\colorbox{pink!25}{$\hookrightarrow$}
{ ``'' (CANNOTANSWER ) }
\\
\speak{Student}{\bf was he actually a knight }
\speak{Teacher}\colorbox{pink!25}{$\not\hookrightarrow$}
{ ``'' (Thomas Maleore, knight, as Jesu help him for his great might, as he is the servant of Jesu both day and night. ) }
\\
\speak{Student}{\bf who made him a knight }
\speak{Teacher}\colorbox{pink!25}{$\not\hookrightarrow$}
{ ``'' (CANNOTANSWER ) }
\\
\speak{Student}{\bf what was his most famous book }
\speak{Teacher}\colorbox{pink!25}{$\hookrightarrow$}
{ ``'' (CANNOTANSWER ) }
\\
\speak{Student}{\bf when was it published }
\speak{Teacher}\colorbox{pink!25}{$\not\hookrightarrow$}
{ ``'' (CANNOTANSWER ) }
\\
 \end{dialogue}\end{tcolorbox}\end{figure}

\end{document}

