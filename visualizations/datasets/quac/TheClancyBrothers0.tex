\documentclass[11pt,a4paper, onecolumn]{article}
\usepackage{times}
\usepackage{latexsym}
\usepackage{url}
\usepackage{textcomp}
\usepackage{bbm}
\usepackage{amsmath}
\usepackage{booktabs}
\usepackage{tabularx}
\usepackage{graphicx}
\usepackage{dialogue}
\usepackage{mathtools}
\usepackage{hyperref}
%\hypersetup{draft}

\usepackage{multirow}
\usepackage{mdframed}
\usepackage{tcolorbox}

\usepackage{xcolor,pifont}
%\newcommand{\cmark}{\ding{51}}
%\newcommand{\xmark}{\ding{55}}

\setcounter{topnumber}{2}
\setcounter{bottomnumber}{2}
\setcounter{totalnumber}{4}
\renewcommand{\topfraction}{0.75}
\renewcommand{\bottomfraction}{0.75}
\renewcommand{\textfraction}{0.05}
\renewcommand{\floatpagefraction}{0.6}

\newcommand\cmark {\textcolor{green}{\ding{52}}}
\newcommand\xmark {\textcolor{red}{\ding{55}}}
\mdfdefinestyle{dialogue}{
    backgroundcolor=yellow!20,
    innermargin=5pt
}
\usepackage{amssymb}
\usepackage{soul}
\makeatletter

\begin{document}

\hspace{15pt}{\textbf{Section}:The Clancy Brothers0\\}
\\ Context: In March 1956, Tommy Makem was unemployed. He had recently moved to Dover, New Hampshire, where many of his family members had emigrated to work in the local cotton mills. He had found a job there making printing presses but had an accident when a two-ton steel press that he was guiding with his hand broke from its chain. The falling press tore the tendons from the bone in three of the fingers of his left hand. His hand in a sling, and knowing the Clancy brothers in New York, he decided that he would like to make a record with them. He told this to Paddy Clancy, who with the sponsorship of Diane Hamilton and the assistance of his brother Liam founded a record company, Tradition Records, in 1956. Paddy agreed and together he, Tom, Liam, and Tommy Makem recorded an album of Irish rebel songs, The Rising of the Moon, one of the new label's first releases. Paddy's harmonica provided the only musical accompaniment for this debut album. Little thought was given to continuing as a singing group. They all were busy establishing theatrical careers for themselves, in addition to their work at Tradition Records. But the album was a local success and requests were often demanded for the brothers and Tommy Makem to sing some of their songs at parties and informal pub settings. Slowly, the singing gigs began to outweigh the acting gigs and by 1959, serious thought was given to a new album. Liam had developed some guitar skills, Tommy's hand had healed enough he was again able to play tin whistle and bagpipes, and the times spent singing together had improved their style. No longer were they the rough, mostly unaccompanied group of actors singing for an album to jumpstart a record label; they were becoming a professional singing group. The release of their second album, this one of Irish drinking songs called Come Fill Your Glass with Us, solidified their new careers as singers. The album was a success, and they made many appearances on the pub circuit in New York, Chicago, and Boston. It was at their first official gig after Come Fill Your Glass With Us that the group finally found a name for themselves. The nightclub owner asked for a name to put on the marquee, but they had not decided on one yet. Unable to agree on a name (which included suggestions like The Beggermen, The Tinkers and even The Chieftains) the owner decided for them, simply billing them as ''The Clancy Brothers and Tommy Makem''. The name stuck. They decided to try singing full-time for six months. If their singing was successful, they would continue with it; if not, then they would return to acting. The Clancy brothers and Tommy Makem proved successful as a singing group and in early 1961, they attracted the attention of scouts from The Ed Sullivan Show. The Clancy Brothers' mother read news of the terrible ice and snow storms in New York City and sent Aran sweaters for her sons and Tommy Makem to keep them warm. They wore the sweaters for the first time at the Blue Angel nightclub in Manhattan, simply as part of their regular winter clothes. When the group's manager Marty Erlichman, who had been searching for a special ''look'' for the group, saw the sweaters, he exclaimed, ''That's it! That's it! That's what you're going to wear.'' Ehrlichman requested that the group wear the sweaters on their upcoming television appearance on The Ed Sullivan Show. After they did, the sales of Aran sweaters rose by 700  according to Liam Clancy, and they soon became the Clancy Brothers and Tommy Makem's trademark costume. On 12 March 1961, the Clancy Brothers and Tommy Makem performed for around fifteen minutes in front of a television audience of forty million people for the first time on The Ed Sullivan Show. A previously scheduled artist did not appear that night, and the Clancy Brothers and Tommy Makem were given the newly available time slot on the show, in addition to the two songs they had initially planned to do. The televised performance and the success of the Clancys' and Makem's nightclub performances attracted the attention of John Hammond of Columbia Records. The group was offered a five-year contract with an advance of  100,000, a huge sum in 1961. For their first album with Columbia, A Spontaneous Performance Recording, they enlisted Pete Seeger, one of the leaders of the American Folk Revival, as backup banjo player. The record included songs that would soon become classics for the group, such as ''Brennan on the Moor'', ''Jug of Punch'', ''Reilly's Daughter'', ''Finnegan's Wake'', ''Haul Away Joe'', ''Roddy McCorley'', ''Portlairge'' and ''The Moonshiner''. The album was nominated for a Grammy Award in 1962. Around the same time that they recorded A Spontaneous Performance, the Clancy Brothers and Tommy Makem cut their final, eponymous album with Tradition Records. By the end of 1962, they released a second album with Columbia, Hearty and Hellish! A Live Nightclub Performance, and they played an acclaimed concert at Carnegie Hall. Additionally, they were making appearances on major radio and television talk-shows in America. Meanwhile, after taking the rest of 1976 off, Paddy and Tom made plans to bring back the Clancy Brothers. They asked Bobby Clancy to return to the group. Tom was at the height of his new career in Hollywood and Paddy was busy with his farm, so it was ultimately decided to tour on a part-time basis and only in the United States. Their recently deceased sister Cait's son, Robbie O'Connell, was an up-and-coming musician in the US and in Ireland; he was also helping manage, along with Bobby, the inn that Cait had opened up years before. They asked him to take on the role Liam had vacated in the group. He played the guitar and occasionally the mandolin, while Bobby played the banjo, guitar, harmonica, and bodhran. Paddy continued to play the lead harmonica. Beginning in 1977, the Clancy Brothers and Robbie O'Connell toured three months a year in March, August, and November. Tom would fly over a few days before each tour and rehearse material, mostly oldies from their 1960s albums but some new ones as well. Robbie was a songwriter, composing several numbers the group sang regularly, such as ''Bobby's Britches'', ''Ferrybank Piper'', ''There Were Roses'' and ''You're Not Irish''. He also included songs written by others, such as ''Dear Boss'', ''Sister Josephine'', ''John O'Dreams'', and what is possibly his signature song, ''Killkelly''. Bobby also sang numbers new to the group, including ''Love of the North'', ''Song for Ireland'', and ''Anne Boleyn''. In America, the Clancy Brothers continued where they had left off the previous year, still packing Carnegie Hall. Reviews cited Robbie as a fresh addition to the group with his original compositions. Over the next several years, Paddy and Tom brought in some new material too. ''The Green Fields of France'', also known as ''Willie McBride'', by Eric Bogle had become a hit with a recording by the Clancys' old back-up musicians, the Furey Brothers, in the early 1980s. Soon numerous Irish groups were singing it, including the Clancy Brothers and Makem and Clancy. It became a staple in Tom's repertoire. He also sang ''Logger Lover''. The group added new lyrics to the old Irish ballad, ''She Didn't Dance'', and reworked old classics, such as ''As I Roved Out'', ''Beer, Beer, Beer'', and ''Rebellion 1916 Medley''. Some of these songs appeared on the Clancy Brothers' first album in nine years, The Clancy Brothers with Robbie O'Connell Live! (1982). In the summer of 1983, the group travelled to their hometown in Ireland to film a 20-minute special on sea songs, sung on location on the fishing ships in the area. It was called Songs of the Sea. Directed by Irish filmmaker David Donaghy, it was broadcast on the BBC Northern Ireland. Tom tried on many occasions to put it on videocassette but the plans fell through. CANNOTANSWER

\begin{figure}[t] \small \begin{tcolorbox}[boxsep=0pt,left=5pt,right=0pt,top=2pt,colback = yellow!5] \begin{dialogue}
 \small 
 \speak{Student}{\bf When dd Robbie O'Connell join the group? }
\speak{Teacher}\colorbox{pink!25}{ $\bar{\hookrightarrow}$}
{ ``'' (1977, ) }
\\
\speak{Student}{\bf Why did they decide to add him? }
\speak{Teacher}\colorbox{pink!25}{ $\bar{\hookrightarrow}$}
{ ``'' (They asked him to take on the role Liam had vacated in the group. ) }
\\
\speak{Student}{\bf Why did Liam leave? }
\speak{Teacher}\colorbox{pink!25}{$\not\hookrightarrow$}
{ ``'' (CANNOTANSWER ) }
\\
\speak{Student}{\bf did they record an album with the new lineup? }
\speak{Teacher}\colorbox{pink!25}{ $\bar{\hookrightarrow}$}
\colorbox{red!25}{Yes,}
{ ``'' (Robbie was a songwriter, composing several numbers the group sang regularly, ) }
\\
\speak{Student}{\bf Did they tour with Robbie O'Connell? }
\speak{Teacher}\colorbox{pink!25}{$\hookrightarrow$}
\colorbox{red!25}{Yes,}
{ ``'' (in 1977, the Clancy Brothers and Robbie O'Connell toured three months a year in March, August, and November. ) }
\\
\speak{Student}{\bf Where did they tour? }
\speak{Teacher}\colorbox{pink!25}{$\hookrightarrow$}
{ ``'' (CANNOTANSWER ) }
\\
\speak{Student}{\bf Did they tour elsewhere? }
\speak{Teacher}\colorbox{pink!25}{ $\bar{\hookrightarrow}$}
{ ``'' (CANNOTANSWER ) }
\\
\speak{Student}{\bf Did the addition of O'Connell change their sound or style? }
\speak{Teacher}\colorbox{pink!25}{$\not\hookrightarrow$}
\colorbox{red!25}{Yes,}
{ ``'' (Reviews cited Robbie as a fresh addition to the group with his original compositions. ) }
\\
 \end{dialogue}\end{tcolorbox}\end{figure}

\end{document}

