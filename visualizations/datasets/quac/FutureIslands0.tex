\documentclass[11pt,a4paper, onecolumn]{article}
\usepackage{times}
\usepackage{latexsym}
\usepackage{url}
\usepackage{textcomp}
\usepackage{bbm}
\usepackage{amsmath}
\usepackage{booktabs}
\usepackage{tabularx}
\usepackage{graphicx}
\usepackage{dialogue}
\usepackage{mathtools}
\usepackage{hyperref}
%\hypersetup{draft}

\usepackage{multirow}
\usepackage{mdframed}
\usepackage{tcolorbox}

\usepackage{xcolor,pifont}
%\newcommand{\cmark}{\ding{51}}
%\newcommand{\xmark}{\ding{55}}

\setcounter{topnumber}{2}
\setcounter{bottomnumber}{2}
\setcounter{totalnumber}{4}
\renewcommand{\topfraction}{0.75}
\renewcommand{\bottomfraction}{0.75}
\renewcommand{\textfraction}{0.05}
\renewcommand{\floatpagefraction}{0.6}

\newcommand\cmark {\textcolor{green}{\ding{52}}}
\newcommand\xmark {\textcolor{red}{\ding{55}}}
\mdfdefinestyle{dialogue}{
    backgroundcolor=yellow!20,
    innermargin=5pt
}
\usepackage{amssymb}
\usepackage{soul}
\makeatletter

\begin{document}

\hspace{15pt}{\textbf{Section}:Future Islands0\\}
\\ Context: Sam Herring and Gerrit Welmers grew up in Morehead City, North Carolina two streets away from each other, and attended the same middle school in Newport, North Carolina. They became friends around 1998, when they were in 8th grade. Herring had started making hip-hop music when he was 13 or 14, while Gerrit was a skater with interests in metal and punk music who bought his first guitar at age 14. Having different musical backgrounds, they did not consider making music together during high-school. William Cashion started playing guitar when he was around 13, having had a couple of bands as a teenager in Raleigh, where he commuted to High School from Wendell, North Carolina. In 2012 he enrolled in the painting and drawing program at ECU and had drawing classes with Sam Herring. The idea to form a band came while Cashion was helping Herring study for an art history exam. They invited local record shop personality Adam Beeby to play rhythmic keyboards and fellow art student Kymia Nawabi for percussion and backing vocals. After a tumultuous debut on Valentine's Day February 14, 2003 at Soccer Moms' House, Herring also invited Welmers to join the band. Only Cashion and Welmers already played a musical instrument--the guitar--but Cashion took the bass and Welmers the keyboards, for a Kraftwerk-inspired sound. Sam Herring played Locke Ernst-Frost an arrogant narcissistic artist from Germany, Ohio, dressed in a 70's-inspired white suit with slicked-back hair, and a heavy German accent. The character's name originally was meant to be Oarlock Ernest Frost but it got shortened as a reference to John Locke the religious poet, Max Ernst, the artist and Robert Frost, the American poet. The band quickly gained a local reputation and started touring the underground venues in the Southwest, playing shows with North Carolina acts like Valient Thorr and Baltimore artists such as Height, Videohippos, OCDJ, Nuclear Power Pants, Santa Dads, Ecstatic Sunshine, Blood Baby, Ponytail and electronic musician Dan Deacon whom they met during a show on May 26, 2004. Nawabi who was already a senior when Cashion, Herring and Welmers were freshmen, left the band to prepare for her graduation project in June-July 2003. When Adam Beeby had to leave Greenville in September 2005, the remaining members dissolved the band. When Art Lord & the Self Portraits disbanded in late 2005, its members forgot they had discussed with alt-country band The Texas Governor the possibility of touring together. Future Islands was formed in early 2006 to keep that commitment, with an original line-up consisting of Cashion, Herring, Welmers and Erick Murillo--bassist for The Kickass --who played an electronic drum kit. Already as Art Lord & the Self-Portraits, the band wanted to change their image and took this opportunity to do so. William Cashion stated: ''Me and Gerrit had been talking for a while about how we wanted to get rid of the gimmick. We wanted to be taken seriously. Our songs had outgrown the gimmick that the band was made on. The songs were starting to deal with bigger, personal, universal themes. We wanted to be taken seriously.'' The band played their first show on February 12, 2006 at an anti-Valentine's Day party in a venue called the Turducken house, opening for about a dozen bands. After writing 6-7 songs in only one week, they had to come up with a new name quickly, narrowing it down to two choices--Future Shoes and Already Islands--and combining them into one. Future Islands self-released the EP Little Advances on April 28, 2006 which they recorded in March 2006. A couple of months later, Herring dropped out college and left Greenville to deal with a substance abuse problem he had acquired: In June, I left town and didn't come back. It was just drug problems, man. I got sucked into the darkness of partying and shit college kids do. I came clean to my parents and said, 'Look, I have a problem and need your help.' I stayed at my parent's for about a month and then moved across the state to Asheville, North Carolina. It took about a year for me to get my act together. The band still continued and on January 6, 2007 they self-released a split CD with Welmers' solo project Moss of Aura, recorded in December 2006. In July 2007, Future Islands recorded their debut album Wave Like Home with Chester Endersby Gwazda at Backdoor Skateshop in Greenville. As Cashion describes: ''When we did Wave Like Home, we were working with a really tight schedule. Sam lived in Asheville and could only be in Greenville to record for a week or so, and we had to work very fast. We recorded the whole album in 3 days, and we spent about a month mixing it.'' After a Halloween party in 2007, Erick Murillo quit the band. Having finished his degree, Cashion moved back to Raleigh: ''We were scattered across North Carolina. I was living in Raleigh on friends' couches, Gerrit was in Greenville and Sam was in Asheville, which was five hours away.'' Between November 2007 and June 2008, Future Islands--encouraged by Dan Deacon and Benny Boeldt from Baltimore band Adventure--relocated to Baltimore. Cashion moved in November, Herring in January and finally Welmers. There, they could have access to cheap rent, be part of a supportive community and be closer to cities like New York and Washington, which allowed them to tour more extensively. During the first half of 2008, the band added another drummer, Sam Ortiz from the Baltimore band Thrust Lab, who left weeks before the start of their first national tour in late July. On August 5, 2008, the band released the track ''Follow You (Pangea Version)'' as part of a split 7'' with Dan Deacon, through the Durham label 307 Knox Records. Future Islands' track on the EP ''Follow You (Pangea version)'' was recorded in April 2006 at the Bonque house in Greenville, NC during the Pangea sessions: the band's first proper session with Chester Endersby Gwazda. London-based label Upset The Rhythm released Wave Like Home on August 25, 2008 which made sales difficult in the US due to the import costs. The cover art was designed by Kymia Nawabi, a former member of Art Lord & the Self-Portraits. She also designed the cover art of the Feathers and Hallways 7'' which was recorded in Oakland, California, on July 21, 2008 during their first U.S. tour. Produced by Chester Endersby Gwazda, it was released on April 15, 2009 by Upset The Rhythm. This single was their first release as a focused three-piece: We have definitely talked about adding a drummer at some point, when the time is right, but right now it just makes sense to be a three piece if, for nothing else, the fact that it is really easy to tour as a three piece. We really have very little gear. We really just have PA speakers for the keyboard and a bass amp. CANNOTANSWER

\begin{figure}[t] \small \begin{tcolorbox}[boxsep=0pt,left=5pt,right=0pt,top=2pt,colback = yellow!5] \begin{dialogue}
 \small 
 \speak{Student}{\bf what is Wave like home? }
\speak{Teacher}\colorbox{pink!25}{ $\bar{\hookrightarrow}$}
{ ``'' (debut album ) }
\\
\speak{Student}{\bf What is a single? }
\speak{Teacher}\colorbox{pink!25}{ $\bar{\hookrightarrow}$}
{ ``'' (Upset The Rhythm. ) }
\\
\speak{Student}{\bf who was the producer? }
\speak{Teacher}\colorbox{pink!25}{ $\bar{\hookrightarrow}$}
{ ``'' (Chester Endersby Gwazda, ) }
\\
\speak{Student}{\bf did they ever tour? }
\speak{Teacher}\colorbox{pink!25}{$\not\hookrightarrow$}
\colorbox{red!25}{Yes,}
{ ``'' (during their first U.S. tour. ) }
\\
\speak{Student}{\bf did anyone ever leave the band? }
\speak{Teacher}\colorbox{pink!25}{ $\bar{\hookrightarrow}$}
\colorbox{red!25}{Yes,}
{ ``'' (Herring and Welmers were freshmen, left the band to prepare for her graduation project ) }
\\
\speak{Student}{\bf who were some of the other band mates? }
\speak{Teacher}\colorbox{pink!25}{ $\bar{\hookrightarrow}$}
{ ``'' (Sam Herring and Gerrit Welmers ) }
\\
 \end{dialogue}\end{tcolorbox}\end{figure}

\end{document}

