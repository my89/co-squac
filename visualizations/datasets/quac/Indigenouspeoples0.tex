\documentclass[11pt,a4paper, onecolumn]{article}
\usepackage{times}
\usepackage{latexsym}
\usepackage{url}
\usepackage{textcomp}
\usepackage{bbm}
\usepackage{amsmath}
\usepackage{booktabs}
\usepackage{tabularx}
\usepackage{graphicx}
\usepackage{dialogue}
\usepackage{mathtools}
\usepackage{hyperref}
%\hypersetup{draft}

\usepackage{multirow}
\usepackage{mdframed}
\usepackage{tcolorbox}

\usepackage{xcolor,pifont}
%\newcommand{\cmark}{\ding{51}}
%\newcommand{\xmark}{\ding{55}}

\setcounter{topnumber}{2}
\setcounter{bottomnumber}{2}
\setcounter{totalnumber}{4}
\renewcommand{\topfraction}{0.75}
\renewcommand{\bottomfraction}{0.75}
\renewcommand{\textfraction}{0.05}
\renewcommand{\floatpagefraction}{0.6}

\newcommand\cmark {\textcolor{green}{\ding{52}}}
\newcommand\xmark {\textcolor{red}{\ding{55}}}
\mdfdefinestyle{dialogue}{
    backgroundcolor=yellow!20,
    innermargin=5pt
}
\usepackage{amssymb}
\usepackage{soul}
\makeatletter

\begin{document}

\hspace{15pt}{\textbf{Section}:Indigenous peoples0\\}
\\ Context: Throughout history, different states designate the groups within their boundaries that are recognized as indigenous peoples according to international or national legislation by different terms. Indigenous people also include people indigenous based on their descent from populations that inhabited the country when non-indigenous religions and cultures arrived--or at the establishment of present state boundaries--who retain some or all of their own social, economic, cultural and political institutions, but who may have been displaced from their traditional domains or who may have resettled outside their ancestral domains. The status of the indigenous groups in the subjugated relationship can be characterized in most instances as an effectively marginalized, isolated or minimally participative one, in comparison to majority groups or the nation-state as a whole. Their ability to influence and participate in the external policies that may exercise jurisdiction over their traditional lands and practices is very frequently limited. This situation can persist even in the case where the indigenous population outnumbers that of the other inhabitants of the region or state; the defining notion here is one of separation from decision and regulatory processes that have some, at least titular, influence over aspects of their community and land rights. In a ground-breaking 1997 decision involving the Ainu people of Japan, the Japanese courts recognised their claim in law, stating that ''If one minority group lived in an area prior to being ruled over by a majority group and preserved its distinct ethnic culture even after being ruled over by the majority group, while another came to live in an area ruled over by a majority after consenting to the majority rule, it must be recognised that it is only natural that the distinct ethnic culture of the former group requires greater consideration.'' The presence of external laws, claims and cultural mores either potentially or actually act to variously constrain the practices and observances of an indigenous society. These constraints can be observed even when the indigenous society is regulated largely by its own tradition and custom. They may be purposefully imposed, or arise as unintended consequence of trans-cultural interaction. They may have a measurable effect, even where countered by other external influences and actions deemed beneficial or that promote indigenous rights and interests. CANNOTANSWER

\begin{figure}[t] \small \begin{tcolorbox}[boxsep=0pt,left=5pt,right=0pt,top=2pt,colback = yellow!5] \begin{dialogue}
 \small 
 \speak{Student}{\bf What are some national definitions? }
\speak{Teacher}\colorbox{pink!25}{$\hookrightarrow$}
{ ``'' (Throughout history, different states designate the groups within their boundaries that are recognized as indigenous peoples according to international or national legislation by different terms. ) }
\\
\speak{Student}{\bf What are some of those terms? }
\speak{Teacher}\colorbox{pink!25}{$\not\hookrightarrow$}
{ ``'' (cultural and political institutions, but who may have been displaced from their traditional domains or who ) }
\\
\speak{Student}{\bf Are there any other interesting aspects about this article? }
\speak{Teacher}\colorbox{pink!25}{$\hookrightarrow$}
\colorbox{red!25}{Yes,}
{ ``'' (The status of the indigenous groups in the subjugated relationship can be characterized in most instances as an effectively marginalized, ) }
\\
\speak{Student}{\bf Has anything changed during the years? }
\speak{Teacher}\colorbox{pink!25}{$\hookrightarrow$}
\colorbox{red!25}{Yes,}
{ ``'' (isolated or minimally participative one, in comparison to majority groups or the nation-state as a whole. ) }
\\
\speak{Student}{\bf What are these national definitions used for? }
\speak{Teacher}\colorbox{pink!25}{$\hookrightarrow$}
{ ``'' (In a ground-breaking 1997 decision involving the Ainu people of Japan, the Japanese courts recognised their claim in law, ) }
\\
\speak{Student}{\bf How was this law viewed by people? }
\speak{Teacher}\colorbox{pink!25}{$\hookrightarrow$}
{ ``'' (rule, it must be recognised that it is only natural that the distinct ethnic culture of the former group requires greater consideration.'' ) }
\\
\speak{Student}{\bf Does this rulling keep on being legit today? }
\speak{Teacher}\colorbox{pink!25}{$\not\hookrightarrow$}
{ ``'' (CANNOTANSWER ) }
\\
 \end{dialogue}\end{tcolorbox}\end{figure}

\end{document}

