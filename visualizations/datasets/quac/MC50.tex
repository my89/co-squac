\documentclass[11pt,a4paper, onecolumn]{article}
\usepackage{times}
\usepackage{latexsym}
\usepackage{url}
\usepackage{textcomp}
\usepackage{bbm}
\usepackage{amsmath}
\usepackage{booktabs}
\usepackage{tabularx}
\usepackage{graphicx}
\usepackage{dialogue}
\usepackage{mathtools}
\usepackage{hyperref}
%\hypersetup{draft}

\usepackage{multirow}
\usepackage{mdframed}
\usepackage{tcolorbox}

\usepackage{xcolor,pifont}
%\newcommand{\cmark}{\ding{51}}
%\newcommand{\xmark}{\ding{55}}

\setcounter{topnumber}{2}
\setcounter{bottomnumber}{2}
\setcounter{totalnumber}{4}
\renewcommand{\topfraction}{0.75}
\renewcommand{\bottomfraction}{0.75}
\renewcommand{\textfraction}{0.05}
\renewcommand{\floatpagefraction}{0.6}

\newcommand\cmark {\textcolor{green}{\ding{52}}}
\newcommand\xmark {\textcolor{red}{\ding{55}}}
\mdfdefinestyle{dialogue}{
    backgroundcolor=yellow!20,
    innermargin=5pt
}
\usepackage{amssymb}
\usepackage{soul}
\makeatletter

\begin{document}

\hspace{15pt}{\textbf{Section}:MC50\\}
\\ Context: The origins of MC5 can be traced to the friendship between guitarists Wayne Kramer and Fred Smith. Friends since their teen years, they were both fans of R&B music, blues, Chuck Berry, Dick Dale, The Ventures, and what would later be called garage rock: they adored any music with speed, energy and a rebellious attitude. Each guitarist/singer formed and led a rock group (Smith's Vibratones and Kramer's Bounty Hunters). As members of both groups left for college or straight jobs, the most committed members eventually united (under Kramer's leadership and the ''Bounty Hunters'' name) with Billy Vargo on guitar and Leo LeDuc on drums (at this point Smith played bass), and were popular and successful enough in and around Detroit that the musicians were able to quit their day jobs and make a living from the group. Kramer felt they needed a manager, which led him to Rob Derminer, a few years older than the others, and deeply involved in Detroit's hipster and left-wing political scenes. Derminer originally auditioned as a bass guitarist (a role which he held briefly in 1964, with Smith switching to guitar to replace Vargo and with Bob Gaspar replacing LeDuc), though they quickly realized that his talents could be better used as a lead singer: Though not conventionally attractive and rather paunchy by traditional frontman standards, he nonetheless had a commanding stage presence, and a booming baritone voice that evidenced his abiding love of American soul and gospel music. Derminer renamed himself Rob Tyner (after Coltrane's pianist McCoy Tyner). Tyner also invented their new name, MC5: it reflected their Detroit roots (it was short for ''Motor City Five'). In some ways the group was similar to other garage bands of the period, composing soon-to-be historic workouts such as ''Black to Comm'' during their mid-teens in the basement of the home of Kramer's mother. Upon Tyner's switch from bassist to vocalist, he was initially replaced by Patrick Burrows, however the lineup was stabilised in 1965 by the arrival of Michael Davis and Dennis Thompson to replace Burrows and Gaspar respectively. The music also reflected Smith and Kramer's increasing interest in free jazz--the guitarists were inspired by the likes of Albert Ayler, Archie Shepp, Sun Ra and late period John Coltrane, and tried to imitate the ecstatic sounds of the squealing, high-pitched saxophonists they adored. MC5 even later opened for a few U.S. midwest shows for Sun Ra, whose influence is obvious in ''Starship''. Kramer and Smith were also deeply inspired by Sonny Sharrock, one of the few electric guitarists working in free jazz, and they eventually developed a unique interlocking style that was like little heard before: Kramer's solos often used a heavy, irregular vibrato, while Smith's rhythms contained an uncommon explosive energy, including patterns that conveyed great excitement, as evidenced in ''Black to Comm'' and many other songs. Playing almost nightly any place they could in and around Detroit, MC5 quickly earned a reputation for their high-energy live performances and had a sizeable local following, regularly drawing sellout audiences of 1000 or more. Contemporary rock writer Robert Bixby stated that the sound of MC5 was like ''a catastrophic force of nature the band was barely able to control'', while Don McLeese notes that fans compared the aftermath of an MC5 performance to the delirious exhaustion experienced after ''a street rumble or an orgy''. (McLeese, 57) Having released a cover of Them's ''I Can Only Give You Everything'' backed with original composition ''One of the Guys'' on the tiny AMG label over a year earlier, in early 1968 their second single was released by Trans-Love Energies on A-Square records (though without the knowledge of that label's owner Jeep Holland). Housed in a striking picture sleeve, it comprised two original songs: ''Borderline'' and ''Looking at You''. The first pressing sold out in a few weeks, and by year's end it had gone through more pressings totaling several thousand copies. A third single that coupled ''I Can Only Give You Everything'' with the original ''I Just Don't Know'' appeared at about the same time on the AMG label, as well. That summer MC5 toured the U.S. east coast, which generated an enormous response, with the group often overshadowing the more famous acts they opened up for: McLeese writes that when opening for Big Brother and the Holding Company audiences regularly demanded multiple encores of MC5, and at a memorable series of concerts, Cream -- one of the leading hard rock groups of the era -- ''left the stage vanquished''. (McLeese, 65) This same east coast tour led to the rapturous aforementioned Rolling Stone cover story that praised MC5 with nearly evangelistic zeal, and also to an association with the radical group Up Against the Wall Motherfuckers. MC5 became the leading band in a burgeoning hard rock scene, serving as mentors to fellow South-Eastern Michigan bands The Stooges and The Up, and major record labels expressed an interest in the group. As related in the notes for reissued editions of the Stooges' debut album, Danny Fields of Elektra Records came to Detroit to see MC5. At Kramer's recommendation, he went to see The Stooges. Fields was so impressed that he ended up offering contracts to both bands in September 1968. They were the first hard rock groups signed to the fledgling Elektra. MC5 earned national attention with their first album, Kick Out the Jams, recorded live on October 30 and 31, 1968, at Detroit's Grande Ballroom. Elektra executive Jac Holzman and producer Bruce Botnick recognized that MC5 were at their best when playing for a receptive audience. Containing such songs as the proto-punk classics ''Kick Out the Jams'' and ''Rama Lama Fa Fa Fa'', the spaced-out ''Starship'' (co-credited to Sun Ra because the lyrics were partly cribbed from one of Ra's poems), and an extended cover of John Lee Hooker's ''Motor City is Burning'' wherein Tyner praises the role of Black Panther snipers during the Detroit Insurrection of 1967. Critic Mark Deming writes that Kick out the Jams ''is one of the most powerfully energetic live albums ever made ... this is an album that refuses to be played quietly.'' The album caused some controversy due to Sinclair's inflammatory liner notes and the title track's rallying cry of ''Kick out the jams, motherfucker!'' According to Kramer, the band recorded this as ''Kick out the jams, brothers and sisters!'' for the single released for radio play; Tyner claimed this was done without group consensus (Thompson, 2000). The edited version also appeared in some LP copies, which also withdrew Sinclair's excitable comments. The album was released in January 1969; reviews were mixed, but the album was relatively successful, quickly selling over 100,000 copies and peaking at #30 on the Billboard album chart in May 1969 during a 23-week stay. When Hudson's, a Detroit-based department store chain, refused to stock Kick Out the Jams due to the obscenity, MC5 responded with a full page advertisement in the local underground magazine Fifth Estate saying ''Stick Alive with the MC5, and Fuck Hudson's!'', prominently including the logo of MC5's label, Elektra Records, in the ad. Hudson's pulled all Elektra records from their stores, and in the ensuing controversy, Jac Holzman, the head of Elektra, dropped the band from their contract. MC5 then signed with Atlantic Records. CANNOTANSWER

\begin{figure}[t] \small \begin{tcolorbox}[boxsep=0pt,left=5pt,right=0pt,top=2pt,colback = yellow!5] \begin{dialogue}
 \small 
 \speak{Student}{\bf what year was kick out the jams released? }
\speak{Teacher}\colorbox{pink!25}{$\hookrightarrow$}
{ ``'' (Jams, recorded live on October 30 and 31, 1968, ) }
\\
\speak{Student}{\bf was it well received? }
\speak{Teacher}\colorbox{pink!25}{$\hookrightarrow$}
\colorbox{red!25}{Yes,}
{ ``'' (Critic Mark Deming writes that Kick out the Jams ''is one of the most powerfully energetic live albums ever made ... ) }
\\
\speak{Student}{\bf Did it sell a lot of records? }
\speak{Teacher}\colorbox{pink!25}{$\hookrightarrow$}
{ ``'' (selling over 100,000 copies ) }
\\
\speak{Student}{\bf what are some of the tracks on the album? }
\speak{Teacher}\colorbox{pink!25}{$\hookrightarrow$}
{ ``'' (''Rama Lama Fa Fa Fa'', ) }
\\
\speak{Student}{\bf Are there any other notable tracks? }
\speak{Teacher}\colorbox{pink!25}{$\hookrightarrow$}
\colorbox{red!25}{Yes,}
{ ``'' (''Kick Out the Jams'' ) }
\\
\speak{Student}{\bf what else is important about this album? }
\speak{Teacher}\colorbox{pink!25}{$\hookrightarrow$}
{ ``'' (The origins of MC5 can be traced to the friendship between guitarists ) }
\\
\speak{Student}{\bf who were the guitarists that were friends? }
\speak{Teacher}\colorbox{pink!25}{$\hookrightarrow$}
{ ``'' (Wayne Kramer and Fred Smith. ) }
\\
\speak{Student}{\bf what number album was kick out the jams for them? }
\speak{Teacher}\colorbox{pink!25}{$\hookrightarrow$}
{ ``'' (MC5 earned national attention with their first album, Kick Out the Jams, ) }
\\
 \end{dialogue}\end{tcolorbox}\end{figure}

\end{document}

