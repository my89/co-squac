\documentclass[11pt,a4paper, onecolumn]{article}
\usepackage{times}
\usepackage{latexsym}
\usepackage{url}
\usepackage{textcomp}
\usepackage{bbm}
\usepackage{amsmath}
\usepackage{booktabs}
\usepackage{tabularx}
\usepackage{graphicx}
\usepackage{dialogue}
\usepackage{mathtools}
\usepackage{hyperref}
%\hypersetup{draft}

\usepackage{multirow}
\usepackage{mdframed}
\usepackage{tcolorbox}

\usepackage{xcolor,pifont}
%\newcommand{\cmark}{\ding{51}}
%\newcommand{\xmark}{\ding{55}}

\setcounter{topnumber}{2}
\setcounter{bottomnumber}{2}
\setcounter{totalnumber}{4}
\renewcommand{\topfraction}{0.75}
\renewcommand{\bottomfraction}{0.75}
\renewcommand{\textfraction}{0.05}
\renewcommand{\floatpagefraction}{0.6}

\newcommand\cmark {\textcolor{green}{\ding{52}}}
\newcommand\xmark {\textcolor{red}{\ding{55}}}
\mdfdefinestyle{dialogue}{
    backgroundcolor=yellow!20,
    innermargin=5pt
}
\usepackage{amssymb}
\usepackage{soul}
\makeatletter

\begin{document}

\hspace{15pt}{\textbf{Section}:The Orb0\\}
\\ Context: In 2001, Alex Paterson formed the record label Badorb.com as an outlet for Orb members' side projects. To promote both Badorb.com and Cydonia, the Orb toured internationally, including their first visit to the United States in four years. NME described the Orb's tour as ''charming'' and that they were ''freed from the Floydian pretensions that dogged the band throughout the mid-'90s''. The Orb, now composed of Paterson, Phillips, and Fehlmann, with guest John Roome, accepted an invitation to join the Area:One concert tour with Moby, Paul Oakenfold, New Order and other alternative and electronic artists. Though the Orb were paired with more mainstream artists during the tour such as Incubus, Paterson and Fehlmann made their next releases a series of several low-key EPs for German label Kompakt in 2002. The Orb found critical success on Kompakt; but Badorb.com collapsed soon after releasing the compilation Bless You. Badorb.com had released fourteen records over the course of fourteen months from artists including Guy Pratt (Conduit), Ayumi Hamasaki, and Takayuki Shiraishi, as well as the Orb's three-track Daleth of Elphame EP. Though Badorb.com was an internet-based record label, they only sold vinyl releases (with one exception, the Orb EP), which Paterson later remarked was a poor idea because ''not many people... have record players''. Though their musical style had changed somewhat since the 1990s, the Orb continued to use their odd synthetic sounds on 2004's Bicycles & Tricycles, to mixed reviews. The Daily Telegraph praised Bicycles & Tricycles as being ''inclusive, exploratory, and an enjoyable journey''; other publications dismissed it as ''stoner dub'' and irrelevant to current electronic music. Like Cydonia, Bicycles & Tricycles featured vocals, including female rapper MC Soom-T who added a hip hop contribution to the album. The Orb left Island Records and released the album on Cooking Vinyl and Sanctuary Records. To promote the album, the band began a UK tour with dub artist Mad Professor. Though the Orb still pulled in large crowds, The Guardian noted that they lacked the intensity found in their earlier performances. CANNOTANSWER

\begin{figure}[t] \small \begin{tcolorbox}[boxsep=0pt,left=5pt,right=0pt,top=2pt,colback = yellow!5] \begin{dialogue}
 \small 
 \speak{Student}{\bf When was this releaed }
\speak{Teacher}\colorbox{pink!25}{$\hookrightarrow$}
{ ``'' (Paterson and Fehlmann made their next releases a series of several low-key EPs for German label Kompakt in 2002. ) }
\\
\speak{Student}{\bf Did they release any other hits? }
\speak{Teacher}\colorbox{pink!25}{$\not\hookrightarrow$}
{ ``'' (CANNOTANSWER ) }
\\
\speak{Student}{\bf Are there any other interesting aspects about this article? }
\speak{Teacher}\colorbox{pink!25}{$\not\hookrightarrow$}
{ ``'' (To promote both Badorb.com and Cydonia, the Orb toured internationally, including their first visit to the United States in four years. ) }
\\
\speak{Student}{\bf Did they tour any where else? }
\speak{Teacher}\colorbox{pink!25}{$\not\hookrightarrow$}
\colorbox{red!25}{Yes,}
{ ``'' (released the album on Cooking Vinyl and Sanctuary Records. To promote the album, the band began a UK tour with dub artist Mad Professor. ) }
\\
\speak{Student}{\bf Did they release another album during this time? }
\speak{Teacher}\colorbox{pink!25}{$\not\hookrightarrow$}
\colorbox{red!25}{Yes,}
{ ``'' (2004's Bicycles & Tricycles, ) }
\\
 \end{dialogue}\end{tcolorbox}\end{figure}

\end{document}

