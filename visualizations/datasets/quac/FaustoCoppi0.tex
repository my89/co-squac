\documentclass[11pt,a4paper, onecolumn]{article}
\usepackage{times}
\usepackage{latexsym}
\usepackage{url}
\usepackage{textcomp}
\usepackage{bbm}
\usepackage{amsmath}
\usepackage{booktabs}
\usepackage{tabularx}
\usepackage{graphicx}
\usepackage{dialogue}
\usepackage{mathtools}
\usepackage{hyperref}
%\hypersetup{draft}

\usepackage{multirow}
\usepackage{mdframed}
\usepackage{tcolorbox}

\usepackage{xcolor,pifont}
%\newcommand{\cmark}{\ding{51}}
%\newcommand{\xmark}{\ding{55}}

\setcounter{topnumber}{2}
\setcounter{bottomnumber}{2}
\setcounter{totalnumber}{4}
\renewcommand{\topfraction}{0.75}
\renewcommand{\bottomfraction}{0.75}
\renewcommand{\textfraction}{0.05}
\renewcommand{\floatpagefraction}{0.6}

\newcommand\cmark {\textcolor{green}{\ding{52}}}
\newcommand\xmark {\textcolor{red}{\ding{55}}}
\mdfdefinestyle{dialogue}{
    backgroundcolor=yellow!20,
    innermargin=5pt
}
\usepackage{amssymb}
\usepackage{soul}
\makeatletter

\begin{document}

\hspace{15pt}{\textbf{Section}:Fausto Coppi0\\}
\\ Context: Coppi's racing days are generally referred to as the beginning of the golden years of cycle racing. A factor is the competition between Coppi and Gino Bartali. Italian tifosi (fans) divided into coppiani and bartaliani. Bartali's rivalry with Coppi divided Italy. Bartali, conservative, religious, was venerated in the rural, agrarian south, while Coppi, more worldly, secular, innovative in diet and training, was hero of the industrial north. The writer Curzio Malaparte said: ''Bartali belongs to those who believe in tradition... he is a metaphysical man protected by the saints. Coppi has nobody in heaven to take care of him. His manager, his masseur, have no wings. He is alone, alone on a bicycle... Bartali prays while he is pedalling: the rational Cartesian and sceptical Coppi is filled with doubts, believes only in his body, his motor''. Their lives came together on 7 January 1940 when Eberardo Pavesi, head of the Legnano team, took on Coppi to ride for Bartali. Their rivalry started when Coppi, the helping hand, won the Giro and Bartali, the star, marshalled the team to chase. By the 1948 world championship at Valkenburg, Limburg in the Netherlands, both climbed off rather than help the other. The Italian cycling association said: ''They have forgotten to honour the Italian prestige they represent. Thinking only of their personal rivalry, they abandoned the race, to the approbation of all sportsmen''. They were suspended for three months. The thaw partly broke when the pair shared a bottle on the Col d'Izoard in the 1952 Tour but the two fell out over who had offered it. ''I did'', Bartali insisted. ''He never gave me anything''. Their rivalry was the subject of intense coverage and resulted in epic races. CANNOTANSWER

\begin{figure}[t] \small \begin{tcolorbox}[boxsep=0pt,left=5pt,right=0pt,top=2pt,colback = yellow!5] \begin{dialogue}
 \small 
 \speak{Student}{\bf When did the Rivalry with Bartali start? }
\speak{Teacher}\colorbox{pink!25}{$\hookrightarrow$}
{ ``'' (Bartali, conservative, religious, was venerated in the rural, agrarian south, while Coppi, more worldly, secular, innovative ) }
\\
\speak{Student}{\bf Why were they rivals? }
\speak{Teacher}\colorbox{pink!25}{$\hookrightarrow$}
{ ``'' (while Coppi, more worldly, secular, innovative in diet and training, was hero of the industrial north. ) }
\\
\speak{Student}{\bf Did Coppi address the rivalry? }
\speak{Teacher}\colorbox{pink!25}{$\hookrightarrow$}
{ ``'' (CANNOTANSWER ) }
\\
\speak{Student}{\bf Who won the race on that day? }
\speak{Teacher}\colorbox{pink!25}{$\hookrightarrow$}
{ ``'' (Their rivalry started when Coppi, the helping hand, won the Giro and Bartali, the star, marshalled the team to chase. ) }
\\
\speak{Student}{\bf Did they race each other again after that day? }
\speak{Teacher}\colorbox{pink!25}{$\hookrightarrow$}
\colorbox{red!25}{Yes,}
{ ``'' (Thinking only of their personal rivalry, they abandoned the race, to the approbation of all sportsmen''. They were suspended for three months. ) }
\\
\speak{Student}{\bf Did the rivalry continue after the suspension? }
\speak{Teacher}\colorbox{pink!25}{ $\bar{\hookrightarrow}$}
\colorbox{red!25}{Yes,}
{ ``'' ('', Bartali insisted. ''He never gave me anything''. Their rivalry was the subject of intense coverage and resulted in epic races. ) }
\\
\speak{Student}{\bf Did they become friends after the rivalry ended? }
\speak{Teacher}\colorbox{pink!25}{$\not\hookrightarrow$}
{ ``'' (CANNOTANSWER ) }
\\
\speak{Student}{\bf Is there anything else of note about the rivalry? }
\speak{Teacher}\colorbox{pink!25}{ $\bar{\hookrightarrow}$}
\colorbox{red!25}{Yes,}
{ ``'' (Coppi's racing days are generally referred to as the beginning of the golden years of cycle racing. A factor is the competition between Coppi and Gino Bartali. ) }
\\
 \end{dialogue}\end{tcolorbox}\end{figure}

\end{document}

