\documentclass[11pt,a4paper, onecolumn]{article}
\usepackage{times}
\usepackage{latexsym}
\usepackage{url}
\usepackage{textcomp}
\usepackage{bbm}
\usepackage{amsmath}
\usepackage{booktabs}
\usepackage{tabularx}
\usepackage{graphicx}
\usepackage{dialogue}
\usepackage{mathtools}
\usepackage{hyperref}
%\hypersetup{draft}

\usepackage{multirow}
\usepackage{mdframed}
\usepackage{tcolorbox}

\usepackage{xcolor,pifont}
%\newcommand{\cmark}{\ding{51}}
%\newcommand{\xmark}{\ding{55}}

\setcounter{topnumber}{2}
\setcounter{bottomnumber}{2}
\setcounter{totalnumber}{4}
\renewcommand{\topfraction}{0.75}
\renewcommand{\bottomfraction}{0.75}
\renewcommand{\textfraction}{0.05}
\renewcommand{\floatpagefraction}{0.6}

\newcommand\cmark {\textcolor{green}{\ding{52}}}
\newcommand\xmark {\textcolor{red}{\ding{55}}}
\mdfdefinestyle{dialogue}{
    backgroundcolor=yellow!20,
    innermargin=5pt
}
\usepackage{amssymb}
\usepackage{soul}
\makeatletter

\begin{document}

\hspace{15pt}{\textbf{Section}:Johnny Cash0\\}
\\ Context: On July 18, 1951, while in Air Force training, Cash met 17-year-old Vivian Liberto at a roller skating rink in her native San Antonio, Texas. They dated for three weeks until Cash was deployed to Germany for a three-year tour. During that time, the couple exchanged hundreds of pages of love letters. On August 7, 1954, one month after his discharge, they were married at St. Ann's Roman Catholic Church in San Antonio. The ceremony was performed by her uncle, Vincent Liberto. They had four daughters: Rosanne, Kathy, Cindy, and Tara. In 1961, Johnny moved his family to a hilltop home overlooking Casitas Springs, California, a small town south of Ojai on Highway 33. He had previously moved his parents to the area to run a small trailer park called The Johnny Cash Trailer Park. Johnny's drinking led to several run-ins with local law enforcement. Liberto later said that she had filed for divorce in 1966 because of Cash's severe drug and alcohol abuse, as well as constant touring, affairs with other women, and his close relationship with June Carter. Their four daughters were then raised by their mother. Cash met singer June Carter, of the famed Carter Family while on tour, and the two became infatuated with each other. In 1968, 13 years after they first met backstage at the Grand Ole Opry, Cash proposed to June, during a live performance in London, Ontario. The couple married on March 1, 1968, in Franklin, Kentucky. They had one child together, John Carter Cash, born March 3, 1970. Cash and Carter continued to work, raising their child, create music, and tour together for 35 years until June's death in May 2003. Throughout their marriage, June attempted to keep Cash off of amphetamines, often taking his drugs and flushing them down the toilet. June remained with him even throughout his multiple admissions for rehab treatment and years of drug abuse. After June's death, Cash believed that his only reason for living was his music. He died four months after her. Cash began performing concerts at prisons starting in the late 1950s. He played his first famous prison concert on January 1, 1958, at San Quentin State Prison. These performances led to a pair of highly successful live albums, Johnny Cash at Folsom Prison (1968) and Johnny Cash at San Quentin (1969). Both live albums reached number 1 on Billboard country album music and the latter crossed over to reach the top of the Billboard pop album chart. In 1969 Cash became an international hit when he eclipsed even the Beatles by selling 6.5 million albums. In comparison, the prison concerts were much more successful than his later live albums such as Strawberry Cake recorded in London and Live at Madison Square Garden, which peaked at #33 and #39 on the album charts respectively. The Folsom Prison record was introduced by a rendition of his ''Folsom Prison Blues,'' while the San Quentin record included the crossover hit single ''A Boy Named Sue,'' a Shel Silverstein-penned novelty song that reached No. 1 on the country charts and No. 2 on the U.S. Top Ten pop charts. The AM versions of the latter contained profanities which were edited out of the aired version. The modern CD versions are unedited thus making them longer than the original vinyl albums, though they retain the audience reaction overdubs of the originals. Cash performed at the Osteraker Prison in Sweden in 1972. The live album Pa Osteraker (''At Osteraker'') was released in 1973. ''San Quentin'' was recorded with Cash replacing ''San Quentin'' with ''Osteraker''. In 1976, a further prison concert, this time at Tennessee Prison, was videotaped for TV broadcast and received a belated CD release after Cash's death as A Concert Behind Prison Walls. From 1969 to 1971, Cash starred in his own television show, The Johnny Cash Show, on the ABC network. The show was performed at the Ryman Auditorium in Nashville. The Statler Brothers opened up for him in every episode; the Carter Family and rockabilly legend Carl Perkins were also part of the regular show entourage. Cash also enjoyed booking mainstream performers as guests; including Neil Young, Louis Armstrong, Neil Diamond, Kenny Rogers and The First Edition (who appeared four times), James Taylor, Ray Charles, Roger Miller, Roy Orbison, Derek and the Dominos, and Bob Dylan. During the same period, he contributed the title song and other songs to the film Little Fauss and Big Halsey, which starred Robert Redford, Michael J. Pollard, and Lauren Hutton. The title song, ''The Ballad of Little Fauss and Big Halsey,'' written by Carl Perkins, was nominated for a Golden Globe award. Cash had met with Dylan in the mid-1960s and became closer friends when they were neighbors in the late 1960s in Woodstock, New York. Cash was enthusiastic about reintroducing the reclusive Dylan to his audience. Cash sang a duet with Dylan on Dylan's country album Nashville Skyline and also wrote the album's Grammy-winning liner notes. Another artist who received a major career boost from The Johnny Cash Show was Kris Kristofferson, who was beginning to make a name for himself as a singer-songwriter. During a live performance of Kristofferson's ''Sunday Mornin' Comin' Down,'' Cash refused to change the lyrics to suit network executives, singing the song with its references to marijuana intact: The closing program of the Johnny Cash show was a gospel music special. Guests included the Blackwood Brothers, Mahalia Jackson, Stuart Hamblen and Billy Graham. CANNOTANSWER

\begin{figure}[t] \small \begin{tcolorbox}[boxsep=0pt,left=5pt,right=0pt,top=2pt,colback = yellow!5] \begin{dialogue}
 \small 
 \speak{Student}{\bf What was his first show? }
\speak{Teacher}\colorbox{pink!25}{$\hookrightarrow$}
{ ``'' (Cash began performing concerts at prisons starting in the late 1950s. ) }
\\
\speak{Student}{\bf How did his early career go }
\speak{Teacher}\colorbox{pink!25}{$\hookrightarrow$}
{ ``'' (These performances led to a pair of highly successful live albums, ) }
\\
\speak{Student}{\bf What were the early albums }
\speak{Teacher}\colorbox{pink!25}{$\hookrightarrow$}
{ ``'' (Johnny Cash at Folsom Prison (1968) and Johnny Cash at San Quentin (1969). ) }
\\
\speak{Student}{\bf Were the albums popular }
\speak{Teacher}\colorbox{pink!25}{$\hookrightarrow$}
\colorbox{red!25}{Yes,}
{ ``'' (Both live albums reached number 1 on Billboard country album music and the latter crossed over to reach the top of the Billboard pop album chart. ) }
\\
\speak{Student}{\bf Did he win any awards }
\speak{Teacher}\colorbox{pink!25}{$\not\hookrightarrow$}
\colorbox{red!25}{Yes,}
{ ``'' (reached No. 1 on the country charts and No. 2 on the U.S. Top Ten pop charts. ) }
\\
 \end{dialogue}\end{tcolorbox}\end{figure}

\end{document}

