\documentclass[11pt,a4paper, onecolumn]{article}
\usepackage{times}
\usepackage{latexsym}
\usepackage{url}
\usepackage{textcomp}
\usepackage{bbm}
\usepackage{amsmath}
\usepackage{booktabs}
\usepackage{tabularx}
\usepackage{graphicx}
\usepackage{dialogue}
\usepackage{mathtools}
\usepackage{hyperref}
%\hypersetup{draft}

\usepackage{multirow}
\usepackage{mdframed}
\usepackage{tcolorbox}

\usepackage{xcolor,pifont}
%\newcommand{\cmark}{\ding{51}}
%\newcommand{\xmark}{\ding{55}}

\setcounter{topnumber}{2}
\setcounter{bottomnumber}{2}
\setcounter{totalnumber}{4}
\renewcommand{\topfraction}{0.75}
\renewcommand{\bottomfraction}{0.75}
\renewcommand{\textfraction}{0.05}
\renewcommand{\floatpagefraction}{0.6}

\newcommand\cmark {\textcolor{green}{\ding{52}}}
\newcommand\xmark {\textcolor{red}{\ding{55}}}
\mdfdefinestyle{dialogue}{
    backgroundcolor=yellow!20,
    innermargin=5pt
}
\usepackage{amssymb}
\usepackage{soul}
\makeatletter

\begin{document}

\hspace{15pt}{\textbf{Section}:Sun Yat-sen0\\}
\\ Context: At the age of 10, Sun Yat-sen began seeking schooling. It is also at this point where he met childhood friend Lu Haodong. By age 13 in 1878 after receiving a few years of local schooling, Sun went to live with his elder brother, Sun Mei (Sun Mei ) in Honolulu. Sun Mei financed Sun Yat-sen's education and would later be a major contributor for the overthrow of the Manchus. During his stay in Honolulu, Sun Yat-sen went to `Iolani School where he studied English, British history, mathematics, science, and Christianity. While he was originally unable to speak English, Sun Yat-sen quickly picked up the language and received a prize for academic achievement from King David Kalakaua before graduating in 1882. He then attended Oahu College (now known as Punahou School) for one semester. In 1883 he was soon sent home to China as his brother was becoming worried that Sun Yat-sen was beginning to embrace Christianity. When he returned to China in 1883 at age 17, Sun met up with his childhood friend Lu Haodong again at Beijidian (Bei Ji Dian ), a temple in Cuiheng Village. They saw many villagers worshipping the Beiji (literally North Pole) Emperor-God in the temple, and were dissatisfied with their ancient healing methods. They broke the statue, incurring the wrath of fellow villagers, and escaped to Hong Kong. While in Hong Kong in 1883 he studied at the Diocesan Boys' School, and from 1884 to 1886 he was at The Government Central School. In 1886 Sun studied medicine at the Guangzhou Boji Hospital under the Christian missionary John G. Kerr. Ultimately, he earned the license of Christian practice as a medical doctor from the Hong Kong College of Medicine for Chinese (the forerunner of The University of Hong Kong) in 1892. Notably, of his class of 12 students, Sun was one of only two who graduated. CANNOTANSWER

\begin{figure}[t] \small \begin{tcolorbox}[boxsep=0pt,left=5pt,right=0pt,top=2pt,colback = yellow!5] \begin{dialogue}
 \small 
 \speak{Student}{\bf When did he start his education? }
\speak{Teacher}\colorbox{pink!25}{$\hookrightarrow$}
{ ``'' (At the age of 10, Sun Yat-sen began seeking schooling. ) }
\\
\speak{Student}{\bf Where did he go to school? }
\speak{Teacher}\colorbox{pink!25}{$\hookrightarrow$}
{ ``'' (Sun Yat-sen went to `Iolani School ) }
\\
\speak{Student}{\bf What did he learn at school? }
\speak{Teacher}\colorbox{pink!25}{$\hookrightarrow$}
{ ``'' (he studied English, British history, mathematics, science, and Christianity. ) }
\\
\speak{Student}{\bf Did he study anything else? }
\speak{Teacher}\colorbox{pink!25}{$\hookrightarrow$}
{ ``'' (CANNOTANSWER ) }
\\
\speak{Student}{\bf Did Sun have any accomplishments in school? }
\speak{Teacher}\colorbox{pink!25}{$\hookrightarrow$}
\colorbox{red!25}{Yes,}
{ ``'' (Ultimately, he earned the license of Christian practice as a medical doctor ) }
\\
\speak{Student}{\bf Did he go to college? }
\speak{Teacher}\colorbox{pink!25}{$\hookrightarrow$}
\colorbox{red!25}{Yes,}
{ ``'' (Hong Kong College of Medicine for Chinese (the forerunner of The University of Hong Kong) ) }
\\
 \end{dialogue}\end{tcolorbox}\end{figure}

\end{document}

