\documentclass[11pt,a4paper, onecolumn]{article}
\usepackage{times}
\usepackage{latexsym}
\usepackage{url}
\usepackage{textcomp}
\usepackage{bbm}
\usepackage{amsmath}
\usepackage{booktabs}
\usepackage{tabularx}
\usepackage{graphicx}
\usepackage{dialogue}
\usepackage{mathtools}
\usepackage{hyperref}
%\hypersetup{draft}

\usepackage{multirow}
\usepackage{mdframed}
\usepackage{tcolorbox}

\usepackage{xcolor,pifont}
%\newcommand{\cmark}{\ding{51}}
%\newcommand{\xmark}{\ding{55}}

\setcounter{topnumber}{2}
\setcounter{bottomnumber}{2}
\setcounter{totalnumber}{4}
\renewcommand{\topfraction}{0.75}
\renewcommand{\bottomfraction}{0.75}
\renewcommand{\textfraction}{0.05}
\renewcommand{\floatpagefraction}{0.6}

\newcommand\cmark {\textcolor{green}{\ding{52}}}
\newcommand\xmark {\textcolor{red}{\ding{55}}}
\mdfdefinestyle{dialogue}{
    backgroundcolor=yellow!20,
    innermargin=5pt
}
\usepackage{amssymb}
\usepackage{soul}
\makeatletter

\begin{document}

\hspace{15pt}{\textbf{Section}:Marcus Garvey0\\}
\\ Context: On 4 October 1916, the Daily Gleaner in Kingston published a letter written by Raphael Morgan, a Jamaican-American priest of the Ecumenical Patriarchate, together with over a dozen other like-minded Jamaican Americans, who wrote in to protest against Garvey's lectures. Garvey's views on Jamaica, they felt, were damaging to both the reputation of their homeland and its people, enumerating several objections to Garvey's stated preference for the prejudice of the American whites over that of English whites. Garvey's response was published a month later: he called the letter a conspiratorial fabrication meant to undermine the success and favour he had gained while in Jamaica and in the United States. While W. E. B. Du Bois felt that the Black Star Line was ''original and promising'', he added that ''Marcus Garvey is, without doubt, the most dangerous enemy of the Negro race in America and in the world. He is either a lunatic or a traitor.'' Du Bois considered Garvey's program of complete separation a capitulation to white supremacy; a tacit admission that Blacks could never be equal to Whites. Noting how popular the idea was with racist thinkers and politicians, Du Bois feared that Garvey threatened the gains made by his own movement. Garvey suspected that Du Bois was prejudiced against him because he was a Caribbean native with darker skin. Du Bois once described Garvey as ''a little, fat black man; ugly, but with intelligent eyes and a big head''. Garvey called Du Bois ''purely and simply a white man's nigger'' and ''a little Dutch, a little French, a little Negro ... a mulatto ... a monstrosity''. This led to an acrimonious relationship between Garvey and the NAACP. In addition, Garvey accused Du Bois of paying conspirators to sabotage the Black Star Line in order to destroy his reputation. Garvey recognized the influence of the Ku Klux Klan and, after the Black Star Line was closed, sought to engage the South in his activism, since the UNIA now lacked a specific program. In early 1922, he went to Atlanta for a conference with KKK imperial giant Edward Young Clarke, seeking to advance his organization in the South. Garvey made a number of incendiary speeches in the months leading up to that meeting; in some, he thanked the whites for Jim Crow. Garvey once stated: ''I regard the Klan, the Anglo-Saxon clubs and White American societies, as far as the Negro is concerned, as better friends of the race than all other groups of hypocritical whites put together. I like honesty and fair play. You may call me a Klansman if you will, but, potentially, every white man is a Klansman as far as the Negro in competition with whites socially, economically and politically is concerned, and there is no use lying.'' After Garvey's entente with the Klan, a number of African-American leaders appealed to U.S. Attorney General Harry M. Daugherty to have Garvey incarcerated. CANNOTANSWER

\begin{figure}[t] \small \begin{tcolorbox}[boxsep=0pt,left=5pt,right=0pt,top=2pt,colback = yellow!5] \begin{dialogue}
 \small 
 \speak{Student}{\bf What conflicts did he have with Du Bois? }
\speak{Teacher}\colorbox{pink!25}{$\hookrightarrow$}
{ ``'' (Du Bois considered Garvey's program of complete separation a capitulation to white supremacy; ) }
\\
\speak{Student}{\bf Can you elaborate? }
\speak{Teacher}\colorbox{pink!25}{$\hookrightarrow$}
\colorbox{red!25}{Yes,}
{ ``'' (a tacit admission that Blacks could never be equal to Whites. ) }
\\
\speak{Student}{\bf How did he approach the situation? }
\speak{Teacher}\colorbox{pink!25}{$\hookrightarrow$}
{ ``'' (Noting how popular the idea was with racist thinkers and politicians, Du Bois feared that Garvey threatened the gains made by his own movement. ) }
\\
\speak{Student}{\bf Did they confront one another? }
\speak{Teacher}\colorbox{pink!25}{$\not\hookrightarrow$}
{ ``'' (CANNOTANSWER ) }
\\
\speak{Student}{\bf Are there any other interesting aspects about this article? }
\speak{Teacher}\colorbox{pink!25}{$\hookrightarrow$}
\colorbox{red!25}{Yes,}
{ ``'' (Garvey recognized the influence of the Ku Klux Klan ) }
\\
\speak{Student}{\bf What was their influence? }
\speak{Teacher}\colorbox{pink!25}{ $\bar{\hookrightarrow}$}
{ ``'' (Klan, the Anglo-Saxon clubs and White American societies, as far as the Negro is concerned, as better friends of the race than all other groups of hypocritical whites put together. ) }
\\
\speak{Student}{\bf Why is that? }
\speak{Teacher}\colorbox{pink!25}{ $\bar{\hookrightarrow}$}
{ ``'' (every white man is a Klansman as far as the Negro in competition with whites socially, economically and politically is concerned, and there is no use lying. ) }
\\
\speak{Student}{\bf Who else did he have conflict with? }
\speak{Teacher}\colorbox{pink!25}{ $\bar{\hookrightarrow}$}
{ ``'' (After Garvey's entente with the Klan, a number of African-American leaders appealed to U.S. Attorney General Harry M. Daugherty to have Garvey incarcerated. ) }
\\
\speak{Student}{\bf For what reason? }
\speak{Teacher}\colorbox{pink!25}{$\not\hookrightarrow$}
{ ``'' (CANNOTANSWER ) }
\\
\speak{Student}{\bf Around what year was this? }
\speak{Teacher}\colorbox{pink!25}{$\not\hookrightarrow$}
{ ``'' (CANNOTANSWER ) }
\\
 \end{dialogue}\end{tcolorbox}\end{figure}

\end{document}

