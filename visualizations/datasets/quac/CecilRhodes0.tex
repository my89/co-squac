\documentclass[11pt,a4paper, onecolumn]{article}
\usepackage{times}
\usepackage{latexsym}
\usepackage{url}
\usepackage{textcomp}
\usepackage{bbm}
\usepackage{amsmath}
\usepackage{booktabs}
\usepackage{tabularx}
\usepackage{graphicx}
\usepackage{dialogue}
\usepackage{mathtools}
\usepackage{hyperref}
%\hypersetup{draft}

\usepackage{multirow}
\usepackage{mdframed}
\usepackage{tcolorbox}

\usepackage{xcolor,pifont}
%\newcommand{\cmark}{\ding{51}}
%\newcommand{\xmark}{\ding{55}}

\setcounter{topnumber}{2}
\setcounter{bottomnumber}{2}
\setcounter{totalnumber}{4}
\renewcommand{\topfraction}{0.75}
\renewcommand{\bottomfraction}{0.75}
\renewcommand{\textfraction}{0.05}
\renewcommand{\floatpagefraction}{0.6}

\newcommand\cmark {\textcolor{green}{\ding{52}}}
\newcommand\xmark {\textcolor{red}{\ding{55}}}
\mdfdefinestyle{dialogue}{
    backgroundcolor=yellow!20,
    innermargin=5pt
}
\usepackage{amssymb}
\usepackage{soul}
\makeatletter

\begin{document}

\hspace{15pt}{\textbf{Section}:Cecil Rhodes0\\}
\\ Context: When he first came to Africa, Rhodes lived on money lent by his aunt Sophia. After a brief stay with the Surveyor-General of Natal, Dr. P.C. Sutherland, in Pietermaritzburg, Rhodes took an interest in agriculture. He joined his brother Herbert on his cotton farm in the Umkomazi valley in Natal. The land was unsuitable for cotton, and the venture failed. In October 1871, 18-year-old Rhodes and his brother Herbert left the colony for the diamond fields of Kimberley. Financed by N M Rothschild & Sons, Rhodes succeeded over the next 17 years in buying up all the smaller diamond mining operations in the Kimberley area. In 1873, he returned to Britain to study at Oxford, but stayed there for only one term after which he went back to South Africa. His monopoly of the world's diamond supply was sealed in 1890 through a strategic partnership with the London-based Diamond Syndicate. They agreed to control world supply to maintain high prices. Rhodes supervised the working of his brother's claim and speculated on his behalf. Among his associates in the early days were John X. Merriman and Charles Rudd, who later became his partner in the De Beers Mining Company and the Niger Oil Company. During the 1880s, Cape vineyards had been devastated by a phylloxera epidemic. The diseased vineyards were dug up and replanted, and farmers were looking for alternatives to wine. In 1892, Rhodes financed The Pioneer Fruit Growing Company at Nooitgedacht, a venture created by Harry Pickstone, an Englishman who had experience with fruit-growing in California. The shipping magnate Percy Molteno had just undertaken the first successful refrigerated export to Europe and in 1896, after consulting with Molteno, Rhodes began to pay more attention to export fruit farming and bought farms in Groot Drakenstein, Wellington and Stellenbosch. A year later, he bought Rhone and Boschendal and commissioned Sir Herbert Baker to build him a cottage there. The successful operation soon expanded into Rhodes Fruit Farms, and formed a cornerstone of the modern-day Cape fruit industry. During his years at Oxford, Rhodes continued to prosper in Kimberley. Before his departure for Oxford, he and C.D. Rudd had moved from the Kimberley Mine to invest in the more costly claims of what was known as old De Beers (Vooruitzicht). It was named after Johannes Nicolaas de Beer and his brother, Diederik Arnoldus, who occupied the farm. After purchasing the land in 1839 from David Danser, a Koranna chief in the area, David Stephanus Fourie, Claudine Fourie-Grosvenor's forebearer, had allowed the de Beers and various other Afrikaner families to cultivate the land. The region extended from the Modder River via the Vet River up to the Vaal River. In 1874 and 1875, the diamond fields were in the grip of depression, but Rhodes and Rudd were among those who stayed to consolidate their interests. They believed that diamonds would be numerous in the hard blue ground that had been exposed after the softer, yellow layer near the surface had been worked out. During this time, the technical problem of clearing out the water that was flooding the mines became serious. Rhodes and Rudd obtained the contract for pumping water out of the three main mines. After Rhodes returned from his first term at Oxford he lived with Robert Dundas Graham, who later became a mining partner with Rudd and Rhodes. On 13 March 1888, Rhodes and Rudd launched De Beers Consolidated Mines after the amalgamation of a number of individual claims. With PS200,000 of capital, the company, of which Rhodes was secretary, owned the largest interest in the mine (PS200,000 in 1880 = PS12.9m in 2004 =  22.5m USD). Rhodes was named the chairman of De Beers at the company's founding in 1888. De Beers was established with funding from N M Rothschild & Sons Limited in 1887. Rhodes had already tried and failed to get a mining concession from Lobengula, king of the Ndebele of Matabeleland. In 1888 he tried again. He sent John Moffat, son of the missionary Robert Moffat, who was trusted by Lobengula, to persuade the latter to sign a treaty of friendship with Britain, and to look favourably on Rhodes' proposals. His associate Charles Rudd, together with Francis Thompson and Rochfort Maguire, assured Lobengula that no more than ten white men would mine in Matabeleland. This limitation was left out of the document, known as the Rudd Concession, which Lobengula signed. Furthermore, it stated that the mining companies could do anything necessary to their operations. When Lobengula discovered later the true effects of the concession, he tried to renounce it, but the British Government ignored him. During the Company's early days, Rhodes and his associates set themselves up to make millions (hundreds of millions in current pounds) over the coming years through what has been described as a ''suppressio veri ... which must be regarded as one of Rhodes's least creditable actions''. Contrary to what the British government and the public had been allowed to think, the Rudd Concession was not vested in the British South Africa Company, but in a short-lived ancillary concern of Rhodes, Rudd and a few others called the Central Search Association, which was quietly formed in London in 1889. This entity renamed itself the United Concessions Company in 1890, and soon after sold the Rudd Concession to the Chartered Company for 1,000,000 shares. When Colonial Office functionaries discovered this chicanery in 1891, they advised Secretary of State for the Colonies Knutsford to consider revoking the concession, but no action was taken. Armed with the Rudd Concession, in 1889 Rhodes obtained a charter from the British Government for his British South Africa Company (BSAC) to rule, police, and make new treaties and concessions from the Limpopo River to the great lakes of Central Africa. He obtained further concessions and treaties north of the Zambezi, such as those in Barotseland (the Lochner Concession with King Lewanika in 1890, which was similar to the Rudd Concession); and in the Lake Mweru area (Alfred Sharpe's 1890 Kazembe concession). Rhodes also sent Sharpe to get a concession over mineral-rich Katanga, but met his match in ruthlessness: when Sharpe was rebuffed by its ruler Msiri, King Leopold II of Belgium obtained a concession over Msiri's dead body for his Congo Free State. Rhodes also wanted Bechuanaland Protectorate (now Botswana) incorporated in the BSAC charter. But three Tswana kings, including Khama III, travelled to Britain and won over British public opinion for it to remain governed by the British Colonial Office in London. Rhodes commented: ''It is humiliating to be utterly beaten by these niggers.'' The British Colonial Office also decided to administer British Central Africa (Nyasaland, today's Malawi) owing to the activism of Scots missionaries trying to end the slave trade. Rhodes paid much of the cost so that the British Central Africa Commissioner Sir Harry Johnston, and his successor Alfred Sharpe, would assist with security for Rhodes in the BSAC's north-eastern territories. Johnston shared Rhodes' expansionist views, but he and his successors were not as pro-settler as Rhodes, and disagreed on dealings with Africans. CANNOTANSWER

\begin{figure}[t] \small \begin{tcolorbox}[boxsep=0pt,left=5pt,right=0pt,top=2pt,colback = yellow!5] \begin{dialogue}
 \small 
 \speak{Student}{\bf What treaties did they have }
\speak{Teacher}\colorbox{pink!25}{$\hookrightarrow$}
{ ``'' (He sent John Moffat, son of the missionary Robert Moffat, who was trusted by Lobengula, to persuade the latter to sign a treaty of friendship with Britain, ) }
\\
\speak{Student}{\bf what did the treaty say }
\speak{Teacher}\colorbox{pink!25}{$\hookrightarrow$}
{ ``'' (and Rochfort Maguire, assured Lobengula that no more than ten white men would mine in Matabeleland. This limitation was left out of the document, ) }
\\
\speak{Student}{\bf did it say anything else }
\speak{Teacher}\colorbox{pink!25}{$\hookrightarrow$}
\colorbox{red!25}{Yes,}
{ ``'' (Furthermore, it stated that the mining companies could do anything necessary to their operations. ) }
\\
\speak{Student}{\bf What happened after the treaty }
\speak{Teacher}\colorbox{pink!25}{$\hookrightarrow$}
{ ``'' (During the Company's early days, Rhodes and his associates set themselves up to make millions (hundreds of millions in current pounds) over the coming years ) }
\\
\speak{Student}{\bf How did they do that }
\speak{Teacher}\colorbox{pink!25}{$\hookrightarrow$}
{ ``'' (through what has been described as a ''suppressio veri ... which must be regarded as one of Rhodes's least creditable actions''. ) }
\\
\speak{Student}{\bf What does that mean }
\speak{Teacher}\colorbox{pink!25}{$\hookrightarrow$}
{ ``'' (Contrary to what the British government and the public had been allowed to think, the Rudd Concession was not vested in the British South Africa Company, ) }
\\
\speak{Student}{\bf So then what happened }
\speak{Teacher}\colorbox{pink!25}{$\hookrightarrow$}
{ ``'' (Rhodes obtained a charter from the British Government for his British South Africa Company (BSAC) to rule, police, and make new treaties and concessions from the Limpopo River to ) }
\\
\speak{Student}{\bf And after the Charter? }
\speak{Teacher}\colorbox{pink!25}{$\hookrightarrow$}
{ ``'' (He obtained further concessions and treaties north of the Zambezi, such as those in Barotseland (the Lochner Concession with King Lewanika in 1890, ) }
\\
\speak{Student}{\bf Did this help him make money }
\speak{Teacher}\colorbox{pink!25}{$\not\hookrightarrow$}
{ ``'' (CANNOTANSWER ) }
\\
 \end{dialogue}\end{tcolorbox}\end{figure}

\end{document}

