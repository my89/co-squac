\documentclass[11pt,a4paper, onecolumn]{article}
\usepackage{times}
\usepackage{latexsym}
\usepackage{url}
\usepackage{textcomp}
\usepackage{bbm}
\usepackage{amsmath}
\usepackage{booktabs}
\usepackage{tabularx}
\usepackage{graphicx}
\usepackage{dialogue}
\usepackage{mathtools}
\usepackage{hyperref}
%\hypersetup{draft}

\usepackage{multirow}
\usepackage{mdframed}
\usepackage{tcolorbox}

\usepackage{xcolor,pifont}
%\newcommand{\cmark}{\ding{51}}
%\newcommand{\xmark}{\ding{55}}

\setcounter{topnumber}{2}
\setcounter{bottomnumber}{2}
\setcounter{totalnumber}{4}
\renewcommand{\topfraction}{0.75}
\renewcommand{\bottomfraction}{0.75}
\renewcommand{\textfraction}{0.05}
\renewcommand{\floatpagefraction}{0.6}

\newcommand\cmark {\textcolor{green}{\ding{52}}}
\newcommand\xmark {\textcolor{red}{\ding{55}}}
\mdfdefinestyle{dialogue}{
    backgroundcolor=yellow!20,
    innermargin=5pt
}
\usepackage{amssymb}
\usepackage{soul}
\makeatletter

\begin{document}

\hspace{15pt}{\textbf{Section}:Skyhooks (band)0\\}
\\ Context: Greg Macainsh and Imants ''Freddie'' Strauks both attended Norwood High School in the Melbourne suburb of Ringwood and formed Spare Parts in 1966 with Macainsh on bass guitar and Strauks on lead vocals. Spare Parts was followed by Sound Pump in 1968, Macainsh formed Reuben Tice in Eltham, with Tony Williams on vocals. By 1970 Macainsh was back with Strauks, now on drums, first in Claptrap and by 1971 in Frame which had Graeme ''Shirley'' Strachan as lead vocalist. Frame also included Pat O'Brien on guitar and Cynthio Ooms on guitar. Strachan had befriended Strauks earlier--he sang with Strauks on the way to parties--and was asked to join Claptrap which was renamed as Frame. Strachan stayed in Frame for about 18 months but left for a career in carpentry and a hobby of surfing in Phillip Island. Skyhooks formed in March 1973 in Melbourne with Steve Hill on vocals (ex-Lillee), Peter Ingliss on guitar (The Captain Matchbox Whoopee Band), Macainsh on bass guitar and backing vocals, Peter Starkie on guitar and backing vocals (Lipp & the Double Dekker Brothers) and Strauks on drums and backing vocals. The name, Skyhooks, came from a fictional organisation in the 1956 film Earth vs. the Flying Saucers. Their first gig was on 16 April 1973 at St Jude's Church hall in Carlton. At a later gig, former Daddy Cool frontman, Ross Wilson was playing in his group Mighty Kong with Skyhooks as a support act. Wilson was impressed with the fledgling band and signed Macainsh to a publishing deal. In August, Bob ''Bongo'' Starkie (Mary Jane Union) on guitar replaced his older brother Peter (later in Jo Jo Zep & The Falcons) and Ingliss was replaced by Red Symons (Scumbag) on guitar, vocals and keyboards. The two new members added a touch of theatre and humour to the band's visual presence. By late 1973, Wilson had convinced Michael Gudinski to sign the band to his booking agency, Australian Entertainment Exchange, and eventually to Gudinski's label, Mushroom Records. Skyhooks gained a cult following around Melbourne including university intelligentsia and pub rockers, but a poorly received show at the January 1974 Sunbury Pop Festival saw the group booed off stage. Two tracks from their live set, ''Hey What's the Matter?'' and ''Love on the Radio'' appeared on Mushroom's Highlights of Sunbury '74. After seeing his performance on TV, Hill phoned Macainsh and resigned. To replace Hill, in March, Macainsh recruited occasional singer, surfer and carpenter Strachan from his Frame era. Strachan had been dubbed ''Shirley'' by fellow surfers due to his curly blond hair a la Shirley Temple. For Skyhooks, the replacement of Hill by Strachan was a pivotal moment, as Strachan had remarkable vocal skills, and a magnetic stage and screen presence. Alongside Macainsh's lyrics, another facet of the group was the twin-guitar sound of Starkie and Symons. Adopting elements of glam rock in their presentation, and lyrics that presented frank depictions of the social life of young Australia in the 1970s, the band shocked conservative middle Australia with their outrageous (for the time) costumes, make-up, lyrics, and on-stage activities. A 1.2 metre (4 ft) high mushroom-shaped phallus was confiscated by Adelaide police after a performance. Six of the ten tracks on their debut album, Living in the 70's, were banned by the Federation of Australian Commercial Broadcasters for their sex and drug references, ''Toorak Cowboy'', ''Whatever Happened to the Revolution?'', ''You Just Like Me Cos I'm Good in Bed'', ''Hey What's the Matter'', ''Motorcycle Bitch'' and ''Smut''. Much of the group's success derived from its distinctive repertoire, mostly penned by bass guitarist Macainsh, with an occasional additional song from Symons--who wrote ''Smut'' and performed its lead vocals. Although Skyhooks were not the first Australian rock band to write songs in a local setting--rather than ditties about love or songs about New York or other foreign lands--they were the first to become commercially successful doing so. Skyhooks songs addressed teenage issues including buying drugs (''Carlton (Lygon Street Limbo)''), suburban sex (''Balwyn Calling''), the gay scene (''Toorak Cowboy'') and loss of girlfriends (''Somewhere in Sydney'') by namechecking Australian locales. Radio personality, Billy Pinnell described the importance of their lyrics in tackling Australia's cultural cringe: [Macainsh] broke down all the barriers [...] opening the door for Australian rock 'n' roll songwriters to write about local places and events. He legitimised Australian songwriting and it meant that Australians became themselves. The first Skyhooks single, ''Living in the 70's'', was released in August, ahead of the album, and peaked at #7 on the Australian Kent Music Report Singles Charts. Living in the 70's initially charted only in Melbourne upon its release on 28 October 1974. It went on to spend 16 weeks at the top of the Australian Kent Music Report Albums Charts from February to June 1975. The album was produced by Wilson, and became the best selling Australian album, to that time, with 226,000 copies sold in Australia. Skyhooks returned to the Sunbury Pop Festival in January 1975. They were declared the best performers by Rolling Stone Australia and The Age reviewers, and Gudinski now took over their management. The second single, ''Horror Movie'', reached #1 for two weeks in March. The band's success was credited by Gudinski with saving his struggling Mushroom Records and enabled it to develop into the most successful Australian label of its time. The success of the album was also due to support by a new pop music television show Countdown on national public broadcaster ABC Television, rather than promotion by commercial radio. ''Horror Movie'' was the first song played on the first colour transmission of Countdown in early 1975. Despite the radio ban, the ABC's newly established 24-hour rock music station Double Jay chose the album's fifth track, the provocatively titled ''You Just Like Me Cos I'm Good in Bed'', as its first ever broadcast on 19 January. After completing their 1976 US tour, the band remained in San Francisco and recorded their third album with Wilson producing, Straight in a Gay Gay World--called Living in the 70's for US release with ''Living in the 70's'' replacing ''The Girl Says She's Bored''--which appeared in August and peaked at #3 on the Australian album charts. In July, upon return to Australia they launched The Brats Are Back Tour with a single, ''This is My City'', which reached the Top 20. ''Blue Jeans'' followed in August and peaked at #13 on the singles chart. By October, Strachan provided his debut solo single, ''Every Little Bit Hurts'' (a cover of Brenda Holloway's 1964 hit), which reached #3. In February 1977, Symons left the band and was replaced on guitar by Bob Spencer from the band Finch. With Symons' departure the band dropped the glam rock look and used a more straight forward hard rock approach. During 1977 Skyhooks toured nationally three times, while their first single with Spencer, ''Party to End All Parties'', entered the top 30 in May. Strachan released his second solo single, a cover of Smokey Robinson's ''Tracks of My Tears'', which reached the top 20 in July. Meanwhile, Mushroom released a singles anthology, The Skyhooks Tapes, which entered the top 50 in September. The band's mass popularity had declined although they still kept their live performances exciting and irreverent. In January 1978 they toured New Zealand and performed at the Nambassa festival. In February their next single, ''Women in Uniform'', was issued and peaked at #8, while its album Guilty Until Proven Insane followed in March and reached #6. The album was produced by Americans Eddie Leonetti and Jack Douglas. The second single from the album, ''Megalomania'' issued in May, did not enter the top 40. Strachan told band members he intended to leave--but it was not officially announced for six months--he continued regular shows until his final gig with Skyhooks on 29 July. Strachan released further solo singles, ''Mr Summer'' in October and ''Nothing but the Best'' in January 1979, but neither charted in the top 50. Strachan's replacement in Skyhooks, on lead vocals, was Tony Williams (ex-Reuben Tice with Macainsh). Williams' first single for Skyhooks, ''Over the Border'', a political song about the state of the Queensland Police Force at the time, reached the top 40 in April, and their fifth studio album, Hot for the Orient, appeared in May 1980, but failed to enter the top 50. From 1975 to 1977, Skyhooks were--alongside Sherbet--the most commercially successful group in Australia, but over the next few years, Skyhooks rapidly faded from the public eye with the departure of key members, and in 1980 the band announced its break-up in controversial circumstances. Ian ''Molly'' Meldrum, usually a supporter of Skyhooks, savaged Hot for the Orient on his ''Humdrum'' segment of Countdown--viewers demanded that the band appear on a following show to defend it. Poor reception of the album both by the public and reviewers led the band to take out a page-sized ad in the local music press declaring ''Why Don't You All Get Fu**ed'' (title of one of their songs) and they played their last performance on 8 June, not in their hometown of Melbourne, but in the mining town of Kalgoorlie in Western Australia. CANNOTANSWER

\begin{figure}[t] \small \begin{tcolorbox}[boxsep=0pt,left=5pt,right=0pt,top=2pt,colback = yellow!5] \begin{dialogue}
 \small 
 \speak{Student}{\bf What performances they did in later years? }
\speak{Teacher}\colorbox{pink!25}{$\hookrightarrow$}
{ ``'' (Their first gig was on 16 April 1973 at St Jude's Church hall in Carlton. ) }
\\
\speak{Student}{\bf After first gig what they did? }
\speak{Teacher}\colorbox{pink!25}{$\not\hookrightarrow$}
{ ``'' (By late 1973, Wilson had convinced Michael Gudinski to sign the band to his booking agency, Australian Entertainment Exchange, and eventually to Gudinski's label, Mushroom Records. ) }
\\
\speak{Student}{\bf When did they break up? }
\speak{Teacher}\colorbox{pink!25}{$\not\hookrightarrow$}
{ ``'' (they played their last performance on 8 June, not in their hometown of Melbourne, but in the mining town of Kalgoorlie in Western Australia. ) }
\\
\speak{Student}{\bf Before break up what happened to them? }
\speak{Teacher}\colorbox{pink!25}{$\not\hookrightarrow$}
{ ``'' (Skyhooks rapidly faded from the public eye with the departure of key members, and in 1980 the band announced its break-up in controversial circumstances. ) }
\\
\speak{Student}{\bf What performances they had in later years? }
\speak{Teacher}\colorbox{pink!25}{$\hookrightarrow$}
{ ``'' (In January 1978 they toured New Zealand and performed at the Nambassa festival. In February their next single, ''Women in Uniform'', was issued and peaked at #8, ) }
\\
\speak{Student}{\bf Did they tour other countries? }
\speak{Teacher}\colorbox{pink!25}{$\hookrightarrow$}
{ ``'' (After completing their 1976 US tour, the band remained in San Francisco and recorded their third album with Wilson producing, ) }
\\
\speak{Student}{\bf After US tour what country they visited? }
\speak{Teacher}\colorbox{pink!25}{$\hookrightarrow$}
{ ``'' (In July, upon return to Australia they launched The Brats Are Back Tour with a single, ''This is My City'', which reached the Top 20. ) }
\\
 \end{dialogue}\end{tcolorbox}\end{figure}

\end{document}

