\documentclass[11pt,a4paper, onecolumn]{article}
\usepackage{times}
\usepackage{latexsym}
\usepackage{url}
\usepackage{textcomp}
\usepackage{bbm}
\usepackage{amsmath}
\usepackage{booktabs}
\usepackage{tabularx}
\usepackage{graphicx}
\usepackage{dialogue}
\usepackage{mathtools}
\usepackage{hyperref}
%\hypersetup{draft}

\usepackage{multirow}
\usepackage{mdframed}
\usepackage{tcolorbox}

\usepackage{xcolor,pifont}
%\newcommand{\cmark}{\ding{51}}
%\newcommand{\xmark}{\ding{55}}

\setcounter{topnumber}{2}
\setcounter{bottomnumber}{2}
\setcounter{totalnumber}{4}
\renewcommand{\topfraction}{0.75}
\renewcommand{\bottomfraction}{0.75}
\renewcommand{\textfraction}{0.05}
\renewcommand{\floatpagefraction}{0.6}

\newcommand\cmark {\textcolor{green}{\ding{52}}}
\newcommand\xmark {\textcolor{red}{\ding{55}}}
\mdfdefinestyle{dialogue}{
    backgroundcolor=yellow!20,
    innermargin=5pt
}
\usepackage{amssymb}
\usepackage{soul}
\makeatletter

\begin{document}

\hspace{15pt}{\textbf{Section}:Louis Jordan0\\}
\\ Context: In 1951, Jordan assembled a short-lived big band that included Pee Wee Moore and others, at a time when big bands were declining in popularity. This is considered the beginning of his commercial decline, even though he reverted to the Tympany Five format within a year. By the mid-1950s, Jordan's records were not selling as well as before, and he left Decca Records. The next label to sign Jordan was Aladdin Records, for which Jordan recorded 21 songs in early 1954. Nine singles were released from these sessions; three of the songs were not released. In 1955, Jordan recorded with RCA's ''independent'' subsidiary ''X'' Records, which changed its name to Vik Records while Jordan was with them. Three singles were released under the ''X'' imprint and one under the Vik imprint; four tracks were not released. In these sessions Jordan intensified his sound to compete with rock and roll. In 1956, Mercury Records signed Jordan and released two LPs and a handful of singles. Jordan's first LP for Mercury, Somebody Up There Digs Me (1956), showcased updated rock-and-roll versions of previous hits such as ''Ain't Nobody Here but Us Chickens'', ''Caldonia'', ''Choo Choo Ch'Boogie'', ''Salt Pork, West Virginia'', and ''Beware!'' Its follow-up, Man, We're Wailin' (1957), featured a more laid-back, ''late-night'' sound. Mercury intended this to be a comeback for Jordan, but it was not commercially successful, and the label let him go in 1958. He recorded sporadically in the 1960s for Warwick (1960), Black Lion (1962), Tangerine (1962-1965), and Pzazz (1968) and in the early 1970s for Black and Blue (1973), Blues Spectrum (1973), and JSP (1974). In 1962 he appeared on the album Louis Jordan Sings, by the British trumpeter and bandleader Chris Barber. Speaking in 2012, Barber recalled seeing Jordan in the early 1960s at the Apollo Theater in New York, with the intention of bringing him to the UK to record There for the first time: ''playing with him was just frightening. It's a bit like an amateur guitar player from a back street who has just bought a Spanish guitar, working with Segovia. He didn't make you feel small, but he was just so perfect in what he did. ... I still remember watching him singing, but he would accompany himself on the alto, and you were convinced he was playing the alto while he was singing. ... the breath hadn't gone from his last word before he was playing his alto and it seemed to be simultaneous. ... He got a very raw deal from history... In the Chick Webb band there were two regular singers - Ella [Fitzgerald] and Louis Jordan. And yet really history has consigned him to just being a comedy vocal thing with a bit of rock and roll, and the first alto ... but he was such a consumately good singer that it's sad that he wasn't known more for it.'' CANNOTANSWER

\begin{figure}[t] \small \begin{tcolorbox}[boxsep=0pt,left=5pt,right=0pt,top=2pt,colback = yellow!5] \begin{dialogue}
 \small 
 \speak{Student}{\bf When did the decline start? }
\speak{Teacher}\colorbox{pink!25}{$\hookrightarrow$}
{ ``'' (By the mid-1950s, Jordan's records were not selling as well as before, and he left Decca Records. ) }
\\
\speak{Student}{\bf What did he do after leaving Deca records? }
\speak{Teacher}\colorbox{pink!25}{$\hookrightarrow$}
{ ``'' (The next label to sign Jordan was Aladdin Records, ) }
\\
\speak{Student}{\bf Did he release an album with Aladdin Records? }
\speak{Teacher}\colorbox{pink!25}{ $\bar{\hookrightarrow}$}
\colorbox{red!25}{No,}
{ ``'' (Nine singles were released ) }
\\
\speak{Student}{\bf What were some of the singles titles? }
\speak{Teacher}\colorbox{pink!25}{$\not\hookrightarrow$}
{ ``'' (CANNOTANSWER ) }
\\
\speak{Student}{\bf Are there any other interesting aspects about this article? }
\speak{Teacher}\colorbox{pink!25}{$\hookrightarrow$}
\colorbox{red!25}{Yes,}
{ ``'' (In 1956, Mercury Records signed Jordan and released two LPs and a handful of singles. ) }
\\
\speak{Student}{\bf How did these singles do? }
\speak{Teacher}\colorbox{pink!25}{ $\bar{\hookrightarrow}$}
{ ``'' (it was not commercially successful, ) }
\\
\speak{Student}{\bf Did he have any other success? }
\speak{Teacher}\colorbox{pink!25}{$\not\hookrightarrow$}
{ ``'' (In 1962 he appeared on the album Louis Jordan Sings, by the British trumpeter and bandleader Chris Barber. ) }
\\
\speak{Student}{\bf Did he record with any other artists? }
\speak{Teacher}\colorbox{pink!25}{ $\bar{\hookrightarrow}$}
{ ``'' (In 1951, Jordan assembled a short-lived big band that included Pee Wee Moore and others, ) }
\\
\speak{Student}{\bf What happened to this band? }
\speak{Teacher}\colorbox{pink!25}{$\not\hookrightarrow$}
{ ``'' (This is considered the beginning of his commercial decline, even though he reverted to the Tympany Five format within a year. ) }
\\
 \end{dialogue}\end{tcolorbox}\end{figure}

\end{document}

