\documentclass[11pt,a4paper, onecolumn]{article}
\usepackage{times}
\usepackage{latexsym}
\usepackage{url}
\usepackage{textcomp}
\usepackage{bbm}
\usepackage{amsmath}
\usepackage{booktabs}
\usepackage{tabularx}
\usepackage{graphicx}
\usepackage{dialogue}
\usepackage{mathtools}
\usepackage{hyperref}
%\hypersetup{draft}

\usepackage{multirow}
\usepackage{mdframed}
\usepackage{tcolorbox}

\usepackage{xcolor,pifont}
%\newcommand{\cmark}{\ding{51}}
%\newcommand{\xmark}{\ding{55}}

\setcounter{topnumber}{2}
\setcounter{bottomnumber}{2}
\setcounter{totalnumber}{4}
\renewcommand{\topfraction}{0.75}
\renewcommand{\bottomfraction}{0.75}
\renewcommand{\textfraction}{0.05}
\renewcommand{\floatpagefraction}{0.6}

\newcommand\cmark {\textcolor{green}{\ding{52}}}
\newcommand\xmark {\textcolor{red}{\ding{55}}}
\mdfdefinestyle{dialogue}{
    backgroundcolor=yellow!20,
    innermargin=5pt
}
\usepackage{amssymb}
\usepackage{soul}
\makeatletter

\begin{document}

\hspace{15pt}{\textbf{Section}:Gujarati language.txt0\\}
\\ Context: Gujarati is an Indo-Aryan language native to the Indian state of Gujarat. It is part of the greater Indo-European language family. Gujarati is descended from Old Gujarati (''circa'' 1100–1500 AD). In India, it is the official language in the state of Gujarat, as well as an official language in the union territories of Daman and Diu and Dadra and Nagar Haveli. Gujarati is the language of the Gujjars, who had ruled Rajputana and Punjab. According to the Central Intelligence Agency (CIA), 4.5  of the Indian population (1.21 billion according to the 2011 census) speaks Gujarati, which amounts to 46 million speakers in India. There are about 50 million speakers of Gujarati worldwide, making it the 26th-most-spoken native language in the world. Gujarati was the first language of Mahatma Gandhi and Muhammad Ali Jinnah. Gujarati (also sometimes spelled ''Gujerati'', ''Gujarathi'', ''Guzratee'', ''Guujaratee'', ''Gujarati'', ''Gujrathi'', and ''Gujerathi'') is a modern IA (Indo-Aryan) language evolved from Sanskrit. The traditional practice is to differentiate the IA languages on the basis of three historical stages: Another view postulates successive family tree splits, in which Gujarati is assumed to have separated from other IA languages in four stages: The principal changes from Sanskrit are the following: Gujarati is then customarily divided into the following three historical stages: CANNOTANSWER

\begin{figure}[t] \small \begin{tcolorbox}[boxsep=0pt,left=5pt,right=0pt,top=2pt,colback = yellow!5] \begin{dialogue}
 \small 
 \speak{Student}{\bf What is descended from an old language? }
\speak{Teacher}\colorbox{pink!25}{$\hookrightarrow$}
{ ``Gujarati'' (Gujarati ) }
\\
\speak{Student}{\bf What is the old language called? }
\speak{Teacher}\colorbox{pink!25}{$\hookrightarrow$}
{ ``Old Gujarati'' (Old Gujarati ) }
\\
\speak{Student}{\bf When was it around? }
\speak{Teacher}\colorbox{pink!25}{$\hookrightarrow$}
{ ``1100–1500 AD'' (1100–1500 AD) ) }
\\
\speak{Student}{\bf Where is it the official language? }
\speak{Teacher}\colorbox{pink!25}{$\hookrightarrow$}
{ ``it is the official language in the state of Gujarat'' (it is the official language in the state of Gujarat ) }
\\
\speak{Student}{\bf Where else? }
\speak{Teacher}\colorbox{pink!25}{$\hookrightarrow$}
{ ``in the union territories of Daman and Diu and Dadra and Nagar Haveli'' (in the union territories of Daman and Diu and Dadra and Nagar Haveli ) }
\\
\speak{Student}{\bf Whta kind of language is it? }
\speak{Teacher}\colorbox{pink!25}{$\hookrightarrow$}
{ ``it is an Indo-Aryan language'' (is an Indo-Aryan language ) }
\\
\speak{Student}{\bf What language family is it a part of? }
\speak{Teacher}\colorbox{pink!25}{$\hookrightarrow$}
{ ``Indo-European language family.'' (Indo-European language family. ) }
\\
\speak{Student}{\bf Who spoke this language? }
\speak{Teacher}\colorbox{pink!25}{$\hookrightarrow$}
{ ``the Gujjars'' (Gujjars, ) }
\\
\speak{Student}{\bf What percent of the indian population speaks it? }
\speak{Teacher}\colorbox{pink!25}{$\hookrightarrow$}
{ ``4.5 '' (4.5  ) }
\\
\speak{Student}{\bf According to who? }
\speak{Teacher}\colorbox{pink!25}{$\hookrightarrow$}
{ ``the CIA'' ((CIA ) }
\\
\speak{Student}{\bf What does that stand for? }
\speak{Teacher}\colorbox{pink!25}{$\hookrightarrow$}
{ ``Central Intelligence Agency'' (Central Intelligence Agency ) }
\\
\speak{Student}{\bf How else is Gujarati spelled? }
\speak{Teacher}\colorbox{pink!25}{$\hookrightarrow$}
{ ``Gujarathi'' (Gujarathi ) }
\\
\speak{Student}{\bf What number of people speak it? }
\speak{Teacher}\colorbox{pink!25}{$\hookrightarrow$}
{ ``46 million speakers'' (46 million speakers ) }
\\
\speak{Student}{\bf According to what census? }
\speak{Teacher}\colorbox{pink!25}{$\hookrightarrow$}
{ ``according to the 2011 census'' (according to the 2011 census) ) }
\\
\speak{Student}{\bf What is the population of India? }
\speak{Teacher}\colorbox{pink!25}{$\hookrightarrow$}
{ ``1.21 billion'' ((1.21 billion ) }
\\
 \end{dialogue}\end{tcolorbox}\end{figure}\begin{figure}[t] \small \begin{tcolorbox}[boxsep=0pt,left=5pt,right=0pt,top=2pt,colback = yellow!5] \begin{dialogue}
 \small 
 \speak{Student}{\bf How many speakers of it are there worldwide? }
\speak{Teacher}\colorbox{pink!25}{$\hookrightarrow$}
{ ``about 50 million'' (about 50 million ) }
\\
\speak{Student}{\bf Where does it rank in most spoken language? }
\speak{Teacher}\colorbox{pink!25}{$\hookrightarrow$}
{ ``26th'' (it ) }
\\
\speak{Student}{\bf Who specifically spoke it? }
\speak{Teacher}\colorbox{pink!25}{$\hookrightarrow$}
{ ``Mahatma Gandhi'' (Mahatma Gandhi ) }
\\
\speak{Student}{\bf Anyone else? }
\speak{Teacher}\colorbox{pink!25}{$\hookrightarrow$}
{ ``Muhammad Ali Jinnah.'' (Muhammad Ali Jinnah. ) }
\\
\speak{Student}{\bf How many historical stages is it divided into? }
\speak{Teacher}\colorbox{pink!25}{$\hookrightarrow$}
{ ``three'' (three ) }
\\
 \end{dialogue}\end{tcolorbox}\end{figure}

\end{document}

