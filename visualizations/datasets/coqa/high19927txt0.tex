\documentclass[11pt,a4paper, onecolumn]{article}
\usepackage{times}
\usepackage{latexsym}
\usepackage{url}
\usepackage{textcomp}
\usepackage{bbm}
\usepackage{amsmath}
\usepackage{booktabs}
\usepackage{tabularx}
\usepackage{graphicx}
\usepackage{dialogue}
\usepackage{mathtools}
\usepackage{hyperref}
%\hypersetup{draft}

\usepackage{multirow}
\usepackage{mdframed}
\usepackage{tcolorbox}

\usepackage{xcolor,pifont}
%\newcommand{\cmark}{\ding{51}}
%\newcommand{\xmark}{\ding{55}}

\setcounter{topnumber}{2}
\setcounter{bottomnumber}{2}
\setcounter{totalnumber}{4}
\renewcommand{\topfraction}{0.75}
\renewcommand{\bottomfraction}{0.75}
\renewcommand{\textfraction}{0.05}
\renewcommand{\floatpagefraction}{0.6}

\newcommand\cmark {\textcolor{green}{\ding{52}}}
\newcommand\xmark {\textcolor{red}{\ding{55}}}
\mdfdefinestyle{dialogue}{
    backgroundcolor=yellow!20,
    innermargin=5pt
}
\usepackage{amssymb}
\usepackage{soul}
\makeatletter

\begin{document}

\hspace{15pt}{\textbf{Section}:high19927.txt0\\}
\\ Context: Israel was happy, very happy. The news of a deal to bring home the kidnapped soldier Gilad Shalit , a young man held prisoner by Hamas for five years,spread. But the happiness was hardened by the reality of the price Israelis had paid to set him free. The 1,027 Palestinian prisoners to be exchanged for the single Israeli corporal turned out to include men and women convicted of some of the worst terrorist attacks in the country. ''Ambivalent,'' says Aya Ilouz, of her feelings on the matter. Strolling in downtown Jerusalem with her husband Liron and their 5-month-old daughter Yael, the couple is so in sync on the question of the day that they finish each other's thoughts. ''Yes,'' says Liron, ''we are very happy and excited to see Gilad meet his family. And on the other hand--'' ''We are very concerned,'' says Aya. ''About what happens next,'' Liron explains. ''When the next terrorist blows himself up, someone will have to answer.'' Just around the corner, on King George Street, Alan Bauer had been walking home with his son on March 21, 2002, when a Palestinian man named Mohammad Hashaika exploded a suicide vest packed with metal scraps. Eighty-four people were wounded that day. Of the three killed, one was a woman pregnant with twins. Though the bomber of course died, Israeli courts convicted the two women who drove him to the site of the bombing, easing his way past the Israeli checkpoint by buying flowers to carry in the Mother's Day crowd. ''These women, as I speak, are being released,'' Bauer says. Specifics have a way of weakening the joy of Shalit's release. Among the 477 prisoners released on Tuesday, in the first phase of the exchange, are an organizer of the 2002 Passover bombing that killed 30 people; a woman who developed an online relationship with a lovesick Israeli youth she then had murdered when he came to meet her; and the man who proudly displayed his bloody hands to the mob gathered outside the Ramallah building where two Israeli soldiers were beaten to death after making a wrong turn on Oct. 12, 2000. When the list became public, s of terrorism victims appealed, without success, to Israel's supreme court to prevent the prisoner exchange. The court hearing was interrupted repeatedly by upset survivors, including Shvuel Schijveschuurder, who lost five of his family members in a 2001 attack at a Jerusalem Sbarro. To protest the release of the woman who drove the suicide bomber to the pizza restaurant, Schijveschuurder poured paint on a memorial to Yitzhak Rabin, the Prime Minister slain by an Israeli extremist for signing the Oslo Accords. ''When we say 1,027 prisoners will be released, it's abstract, it doesn't mean anything,'' says Eliad Moreh Rosenberg, who was wounded in the 2002 terrorism bombing at the Hebrew University cafeteria. ''But for victims of terror, it's a reality.'' Israeli officials calculate that 60  of those released resume terrorism attacks. To help prevent that resumption this time around, Israel insisted that most of the prisoners liberated be sent either to the Gaza Strip -- which is sealed off from Israel and under the control of Hamas, which says it continues to observe a cease-fire -- or into exile in Turkey, Qatar or Syria. About 100 arrived in the West Bank, where the government led by Palestinian Authority President Mahmoud Abbas works diligently to suppress terrorism, cooperating with Israeli intelligence and military. With the future uncertain, on Tuesday, Jewish Israelis stopped and stared at televisions wherever they came upon them. On the sidewalk at midmorning outside the 24-hour Hillel Market, 50 people were gathered under a flat screen to catch the first images of Shalit, looking painfully thin . ''It was moving. It was very exciting,'' says Anat Rubin, 42. ''I just saw photos of him getting out of the car. It gave me chills.'' But she says she heard Hamas say that, learning from success, it was keen to kidnap more Israelis in order to win freedom for the 6,000 Palestinians still in Israeli prisons. ''I don't want to see the photos of them doing the V for victory,'' she says. ''Like they won. They are really releasing murderers. I'm happy and sad all together.'' CANNOTANSWER

\begin{figure}[t] \small \begin{tcolorbox}[boxsep=0pt,left=5pt,right=0pt,top=2pt,colback = yellow!5] \begin{dialogue}
 \small 
 \speak{Student}{\bf Who was kidnapped? }
\speak{Teacher}\colorbox{pink!25}{$\hookrightarrow$}
{ ``Gilad Shalit'' (Gilad Shalit ) }
\\
\speak{Student}{\bf how long? }
\speak{Teacher}\colorbox{pink!25}{$\hookrightarrow$}
{ ``five years,'' (five years, ) }
\\
\speak{Student}{\bf How many prisoners exchanged? }
\speak{Teacher}\colorbox{pink!25}{$\hookrightarrow$}
{ ``1,027'' (1,027 ) }
\\
\speak{Student}{\bf What did they include? }
\speak{Teacher}\colorbox{pink!25}{$\hookrightarrow$}
{ ``men and women convicted of some of the worst terrorist attacks in the country.'' (men and women convicted of some of the worst terrorist attacks in the country. ) }
\\
\speak{Student}{\bf did Liron have a child? }
\speak{Teacher}\colorbox{pink!25}{$\hookrightarrow$}
\colorbox{red!25}{Yes,}
{ ``yes'' (men and women convicted of some of the worst terrorist attacks in the country. ) }
\\
\speak{Student}{\bf how old? }
\speak{Teacher}\colorbox{pink!25}{$\hookrightarrow$}
{ ``5-months'' (5-month-old ) }
\\
\speak{Student}{\bf name? }
\speak{Teacher}\colorbox{pink!25}{$\hookrightarrow$}
{ ``Yael'' (Yael ) }
\\
\speak{Student}{\bf Where was was Alan  walking? }
\speak{Teacher}\colorbox{pink!25}{$\hookrightarrow$}
{ ``on King George Street'' (on King George Street, ) }
\\
\speak{Student}{\bf with who? }
\speak{Teacher}\colorbox{pink!25}{$\hookrightarrow$}
{ ``with his son'' (with his son ) }
\\
\speak{Student}{\bf when? }
\speak{Teacher}\colorbox{pink!25}{$\hookrightarrow$}
{ ``on March 21, 2002'' (on March 21, 2002 ) }
\\
\speak{Student}{\bf prisoners released what day? }
\speak{Teacher}\colorbox{pink!25}{$\hookrightarrow$}
{ ``Tuesday,'' (Tuesday, ) }
\\
\speak{Student}{\bf how many? }
\speak{Teacher}\colorbox{pink!25}{$\hookrightarrow$}
{ ``477'' (477 ) }
\\
\speak{Student}{\bf Passover bombing killed how many? }
\speak{Teacher}\colorbox{pink!25}{$\hookrightarrow$}
{ ``30'' (30 ) }
\\
 \end{dialogue}\end{tcolorbox}\end{figure}

\end{document}

