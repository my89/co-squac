\documentclass[11pt,a4paper, onecolumn]{article}
\usepackage{times}
\usepackage{latexsym}
\usepackage{url}
\usepackage{textcomp}
\usepackage{bbm}
\usepackage{amsmath}
\usepackage{booktabs}
\usepackage{tabularx}
\usepackage{graphicx}
\usepackage{dialogue}
\usepackage{mathtools}
\usepackage{hyperref}
%\hypersetup{draft}

\usepackage{multirow}
\usepackage{mdframed}
\usepackage{tcolorbox}

\usepackage{xcolor,pifont}
%\newcommand{\cmark}{\ding{51}}
%\newcommand{\xmark}{\ding{55}}

\setcounter{topnumber}{2}
\setcounter{bottomnumber}{2}
\setcounter{totalnumber}{4}
\renewcommand{\topfraction}{0.75}
\renewcommand{\bottomfraction}{0.75}
\renewcommand{\textfraction}{0.05}
\renewcommand{\floatpagefraction}{0.6}

\newcommand\cmark {\textcolor{green}{\ding{52}}}
\newcommand\xmark {\textcolor{red}{\ding{55}}}
\mdfdefinestyle{dialogue}{
    backgroundcolor=yellow!20,
    innermargin=5pt
}
\usepackage{amssymb}
\usepackage{soul}
\makeatletter

\begin{document}

\hspace{15pt}{\textbf{Section}:data/gutenberg/txt/Edward Phillips Oppenheim   The New Tenant.txt/CHAPTER XXIII 166d45eb26535d5b1cfa273c2cabe24aebb475cb2939afadce71a2c0\\}
\\ Context: CHAPTER XXIII LOVERS Bernard Maddison kept his engagement that evening, and dined alone with Lady Thurwell and Helen. There had been some talk of going to the opera afterwards, but no one seemed to care about it, and so it dropped through. ''For my part,'' Lady Thurwell said, as they sat lingering over their dessert, ''I shall quite enjoy an evening's rest. You literary men, Mr. Maddison, talk a good deal about being overworked, but you know nothing of the life of a chaperon in the season. I tell Helen that she is sadly wanting in gratitude. We do everything worth doing--picture galleries, matinées, shopping, afternoon calls, dinners, dances, receptions--why, there's no slavery like it.'' Helen laughed softly. ''We do a great deal too much, aunt,'' she said. ''I am almost coming round to my father's opinion. You know, Mr. Maddison, he very seldom comes to London, and then only when he wants to pay a visit to his gunmaker, or to renew his hunting kit, or something of that sort. London life does not suit him at all.'' ''I think your father a very wise man,'' he answered. ''He seeks his pleasures in a more wholesome manner.'' She looked thoughtful. ''Yes, I suppose, ethically, the life of a man about town is on a very low level. That is why one meets so few who interest one, as a rule. Don't you think all this society life very frivolous, Mr. Maddison?'' ''I am not willing to be its judge,'' he answered. ''Yet it is a moral axiom that the higher we seek for our pleasures the greater happiness we attain to. I am an uncompromising enemy to what is known as fashionable society, so I will draw no conclusions.'' CANNOTANSWER

\begin{figure}[t] \small \begin{tcolorbox}[boxsep=0pt,left=5pt,right=0pt,top=2pt,colback = yellow!5] \begin{dialogue}
 \small 
 \speak{Student}{\bf Who does city life not favor? }
\speak{Teacher}\colorbox{pink!25}{$\hookrightarrow$}
{ ``Helen's father'' (him ) }
\\
\speak{Student}{\bf Does he visit there often? }
\speak{Teacher}\colorbox{pink!25}{$\hookrightarrow$}
\colorbox{red!25}{No,}
{ ``no'' (him ) }
\\
\speak{Student}{\bf Why does he come then? }
\speak{Teacher}\colorbox{pink!25}{$\hookrightarrow$}
{ ``when he wants to pay a visit to his gunmaker'' (when he wants to pay a visit to his gunmaker ) }
\\
\speak{Student}{\bf Any other reason? }
\speak{Teacher}\colorbox{pink!25}{$\hookrightarrow$}
{ ``or to renew his hunting kit'' (or to renew his hunting ) }
\\
\speak{Student}{\bf Who thinks that's a smart move? }
\speak{Teacher}\colorbox{pink!25}{$\hookrightarrow$}
{ ``Mr. Maddison'' (Mr. Maddison ) }
\\
\speak{Student}{\bf Why's that? }
\speak{Teacher}\colorbox{pink!25}{$\hookrightarrow$}
{ ``He seeks his pleasures in a more wholesome manner'' (He seeks his pleasures in a more wholesome manner ) }
\\
\speak{Student}{\bf Did she immediately dismiss the thought? }
\speak{Teacher}\colorbox{pink!25}{$\hookrightarrow$}
\colorbox{red!25}{No,}
{ ``no'' (He seeks his pleasures in a more wholesome manner ) }
\\
\speak{Student}{\bf How many interesting people does she feel one tends to meet? }
\speak{Teacher}\colorbox{pink!25}{$\hookrightarrow$}
{ ``few'' (few ) }
\\
\speak{Student}{\bf Does Maddison find it all shallow? }
\speak{Teacher}\colorbox{pink!25}{$\hookrightarrow$}
{ ``he draws no conclusions'' (no conclusions ) }
\\
\speak{Student}{\bf Is he a fan then? }
\speak{Teacher}\colorbox{pink!25}{$\hookrightarrow$}
\colorbox{red!25}{No,}
{ ``unknown'' (CANNOTANSWER ) }
\\
\speak{Student}{\bf How many people are eating together? }
\speak{Teacher}\colorbox{pink!25}{$\hookrightarrow$}
{ ``Three'' (and ) }
\\
\speak{Student}{\bf What are their names? }
\speak{Teacher}\colorbox{pink!25}{$\hookrightarrow$}
{ ``Bernard Maddison, Lady Thurwell, and Helen'' (Lady Thurwell and Helen ) }
\\
\speak{Student}{\bf Are they heading out to a performance soon? }
\speak{Teacher}\colorbox{pink!25}{$\hookrightarrow$}
\colorbox{red!25}{No,}
{ ``no'' (Lady Thurwell and Helen ) }
\\
\speak{Student}{\bf What part of the meal are they on? }
\speak{Teacher}\colorbox{pink!25}{$\hookrightarrow$}
{ ``dessert'' (dessert ) }
\\
\speak{Student}{\bf What kind of a man is he? }
\speak{Teacher}\colorbox{pink!25}{$\hookrightarrow$}
{ ``a literary man'' (literary ) }
\\
 \end{dialogue}\end{tcolorbox}\end{figure}\begin{figure}[t] \small \begin{tcolorbox}[boxsep=0pt,left=5pt,right=0pt,top=2pt,colback = yellow!5] \begin{dialogue}
 \small 
 \speak{Student}{\bf Is Thurwell looking forward to an active night? }
\speak{Teacher}\colorbox{pink!25}{$\hookrightarrow$}
\colorbox{red!25}{No,}
{ ``no'' (literary ) }
\\
\speak{Student}{\bf In what capacity is she acting for the dinner? }
\speak{Teacher}\colorbox{pink!25}{$\hookrightarrow$}
{ ``chaperon'' (chaperon ) }
\\
\speak{Student}{\bf For whom? }
\speak{Teacher}\colorbox{pink!25}{$\hookrightarrow$}
{ ``Helen'' (Helen ) }
\\
\speak{Student}{\bf Is Helen grateful? }
\speak{Teacher}\colorbox{pink!25}{$\hookrightarrow$}
\colorbox{red!25}{No,}
{ ``no'' (Helen ) }
\\
\speak{Student}{\bf What kind of things do they do together? }
\speak{Teacher}\colorbox{pink!25}{$\hookrightarrow$}
{ ``picture galleries, matinées, shopping'' (picture galleries, matinées, shopping ) }
\\
 \end{dialogue}\end{tcolorbox}\end{figure}

\end{document}

