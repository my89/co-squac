\documentclass[11pt,a4paper, onecolumn]{article}
\usepackage{times}
\usepackage{latexsym}
\usepackage{url}
\usepackage{textcomp}
\usepackage{bbm}
\usepackage{amsmath}
\usepackage{booktabs}
\usepackage{tabularx}
\usepackage{graphicx}
\usepackage{dialogue}
\usepackage{mathtools}
\usepackage{hyperref}
%\hypersetup{draft}

\usepackage{multirow}
\usepackage{mdframed}
\usepackage{tcolorbox}

\usepackage{xcolor,pifont}
%\newcommand{\cmark}{\ding{51}}
%\newcommand{\xmark}{\ding{55}}

\setcounter{topnumber}{2}
\setcounter{bottomnumber}{2}
\setcounter{totalnumber}{4}
\renewcommand{\topfraction}{0.75}
\renewcommand{\bottomfraction}{0.75}
\renewcommand{\textfraction}{0.05}
\renewcommand{\floatpagefraction}{0.6}

\newcommand\cmark {\textcolor{green}{\ding{52}}}
\newcommand\xmark {\textcolor{red}{\ding{55}}}
\mdfdefinestyle{dialogue}{
    backgroundcolor=yellow!20,
    innermargin=5pt
}
\usepackage{amssymb}
\usepackage{soul}
\makeatletter

\begin{document}

\hspace{15pt}{\textbf{Section}:Private school21\\}
\\ Context: Private schools generally prefer to be called independent schools, because of their freedom to operate outside of government and local government control. Some of these are also known as public schools. Preparatory schools in the UK prepare pupils aged up to 13 years old to enter public schools. The name ''public school'' is based on the fact that the schools were open to pupils from anywhere, and not merely to those from a certain locality, and of any religion or occupation. According to The Good Schools Guide approximately 9 per cent of children being educated in the UK are doing so at fee-paying schools at GSCE level and 13 per cent at A-level.[citation needed] Many independent schools are single-sex (though this is becoming less common). Fees range from under £3,000 to £21,000 and above per year for day pupils, rising to £27,000+ per year for boarders. For details in Scotland, see ''Meeting the Cost''. CANNOTANSWER

\begin{figure}[t] \small \begin{tcolorbox}[boxsep=0pt,left=5pt,right=0pt,top=2pt,colback = yellow!5] \begin{dialogue}
 \small 
 \speak{Student}{\bf Up to what age do students in the United Kingdom attend preparatory schools? }
\speak{Teacher}\colorbox{pink!25}{$\hookrightarrow$}
{ ``'' (13 ) }
\\
\speak{Student}{\bf What schools do preparatory schools prepare British children to attend? }
\speak{Teacher}\colorbox{pink!25}{$\hookrightarrow$}
{ ``'' (public ) }
\\
\speak{Student}{\bf What percentage of British children are educated at GSCE level in fee-paying schools? }
\speak{Teacher}\colorbox{pink!25}{$\hookrightarrow$}
{ ``'' (9 ) }
\\
\speak{Student}{\bf At A-level, what percentage of British students attend fee-paying schools? }
\speak{Teacher}\colorbox{pink!25}{$\hookrightarrow$}
{ ``'' (13 ) }
\\
\speak{Student}{\bf What is the upper range of annual fees for non-boarding students in British public schools? }
\speak{Teacher}\colorbox{pink!25}{$\hookrightarrow$}
{ ``'' (£21,000 ) }
\\
\speak{Student}{\bf Who do preparatory schools in Scotland prepare to enter public schools? }
\speak{Teacher}\colorbox{pink!25}{$\hookrightarrow$}
{ ``'' (CANNOTANSWER ) }
\\
\speak{Student}{\bf What percentage of children educated in Scotland are at independent schools? }
\speak{Teacher}\colorbox{pink!25}{$\hookrightarrow$}
{ ``'' (CANNOTANSWER ) }
\\
\speak{Student}{\bf At what level are 13% of children in Scottish independent schools? }
\speak{Teacher}\colorbox{pink!25}{$\hookrightarrow$}
{ ``'' (CANNOTANSWER ) }
\\
\speak{Student}{\bf What type of school is becoming less common in Scotland? }
\speak{Teacher}\colorbox{pink!25}{$\hookrightarrow$}
{ ``'' (CANNOTANSWER ) }
\\
\speak{Student}{\bf What is the highest range a student would pay when boarding  in a Scottish school? }
\speak{Teacher}\colorbox{pink!25}{$\hookrightarrow$}
{ ``'' (CANNOTANSWER ) }
\\
 \end{dialogue}\end{tcolorbox}\end{figure}

\end{document}

