\documentclass[11pt,a4paper, onecolumn]{article}
\usepackage{times}
\usepackage{latexsym}
\usepackage{url}
\usepackage{textcomp}
\usepackage{bbm}
\usepackage{amsmath}
\usepackage{booktabs}
\usepackage{tabularx}
\usepackage{graphicx}
\usepackage{dialogue}
\usepackage{mathtools}
\usepackage{hyperref}
%\hypersetup{draft}

\usepackage{multirow}
\usepackage{mdframed}
\usepackage{tcolorbox}

\usepackage{xcolor,pifont}
%\newcommand{\cmark}{\ding{51}}
%\newcommand{\xmark}{\ding{55}}

\setcounter{topnumber}{2}
\setcounter{bottomnumber}{2}
\setcounter{totalnumber}{4}
\renewcommand{\topfraction}{0.75}
\renewcommand{\bottomfraction}{0.75}
\renewcommand{\textfraction}{0.05}
\renewcommand{\floatpagefraction}{0.6}

\newcommand\cmark {\textcolor{green}{\ding{52}}}
\newcommand\xmark {\textcolor{red}{\ding{55}}}
\mdfdefinestyle{dialogue}{
    backgroundcolor=yellow!20,
    innermargin=5pt
}
\usepackage{amssymb}
\usepackage{soul}
\makeatletter

\begin{document}

\hspace{15pt}{\textbf{Section}:Rhine41\\}
\\ Context: At the end of World War I, the Rhineland was subject to the Treaty of Versailles. This decreed that it would be occupied by the allies, until 1935 and after that, it would be a demilitarised zone, with the German army forbidden to enter. The Treaty of Versailles and this particular provision, in general, caused much resentment in Germany and is often cited as helping Adolf Hitler's rise to power. The allies left the Rhineland, in 1930 and the German army re-occupied it in 1936, which was enormously popular in Germany. Although the allies could probably have prevented the re-occupation, Britain and France were not inclined to do so, a feature of their policy of appeasement to Hitler. CANNOTANSWER

\begin{figure}[t] \small \begin{tcolorbox}[boxsep=0pt,left=5pt,right=0pt,top=2pt,colback = yellow!5] \begin{dialogue}
 \small 
 \speak{Student}{\bf When was Rhineland subject to the Treaty of Versailles? }
\speak{Teacher}\colorbox{pink!25}{$\hookrightarrow$}
{ ``'' (end of World War I ) }
\\
\speak{Student}{\bf When would the occupation of allies leave Rhineland? }
\speak{Teacher}\colorbox{pink!25}{$\hookrightarrow$}
{ ``'' (1935 ) }
\\
\speak{Student}{\bf After 1935, who would be forbidden to enter Rhineland? }
\speak{Teacher}\colorbox{pink!25}{$\hookrightarrow$}
{ ``'' (German army ) }
\\
\speak{Student}{\bf What do some believe the Treaty of Versailles assisted in? }
\speak{Teacher}\colorbox{pink!25}{$\hookrightarrow$}
{ ``'' (Adolf Hitler's rise to power ) }
\\
\speak{Student}{\bf When did the German army reoccupy Rhineland? }
\speak{Teacher}\colorbox{pink!25}{$\hookrightarrow$}
{ ``'' (1936 ) }
\\
\speak{Student}{\bf When was the Treaty of Versailles written? }
\speak{Teacher}\colorbox{pink!25}{$\hookrightarrow$}
{ ``'' (CANNOTANSWER ) }
\\
\speak{Student}{\bf When did the Germany army enter Rhineland? }
\speak{Teacher}\colorbox{pink!25}{$\hookrightarrow$}
{ ``'' (CANNOTANSWER ) }
\\
\speak{Student}{\bf What year did Adolf Hitler's rise to power? }
\speak{Teacher}\colorbox{pink!25}{$\hookrightarrow$}
{ ``'' (CANNOTANSWER ) }
\\
\speak{Student}{\bf When did the the German army re-occupy Britain and France? }
\speak{Teacher}\colorbox{pink!25}{$\hookrightarrow$}
{ ``'' (CANNOTANSWER ) }
\\
 \end{dialogue}\end{tcolorbox}\end{figure}

\end{document}

