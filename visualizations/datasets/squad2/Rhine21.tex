\documentclass[11pt,a4paper, onecolumn]{article}
\usepackage{times}
\usepackage{latexsym}
\usepackage{url}
\usepackage{textcomp}
\usepackage{bbm}
\usepackage{amsmath}
\usepackage{booktabs}
\usepackage{tabularx}
\usepackage{graphicx}
\usepackage{dialogue}
\usepackage{mathtools}
\usepackage{hyperref}
%\hypersetup{draft}

\usepackage{multirow}
\usepackage{mdframed}
\usepackage{tcolorbox}

\usepackage{xcolor,pifont}
%\newcommand{\cmark}{\ding{51}}
%\newcommand{\xmark}{\ding{55}}

\setcounter{topnumber}{2}
\setcounter{bottomnumber}{2}
\setcounter{totalnumber}{4}
\renewcommand{\topfraction}{0.75}
\renewcommand{\bottomfraction}{0.75}
\renewcommand{\textfraction}{0.05}
\renewcommand{\floatpagefraction}{0.6}

\newcommand\cmark {\textcolor{green}{\ding{52}}}
\newcommand\xmark {\textcolor{red}{\ding{55}}}
\mdfdefinestyle{dialogue}{
    backgroundcolor=yellow!20,
    innermargin=5pt
}
\usepackage{amssymb}
\usepackage{soul}
\makeatletter

\begin{document}

\hspace{15pt}{\textbf{Section}:Rhine21\\}
\\ Context: Before the St. Elizabeth's flood (1421), the Meuse flowed just south of today's line Merwede-Oude Maas to the North Sea and formed an archipelago-like estuary with Waal and Lek. This system of numerous bays, estuary-like extended rivers, many islands and constant changes of the coastline, is hard to imagine today. From 1421 to 1904, the Meuse and Waal merged further upstream at Gorinchem to form Merwede. For flood protection reasons, the Meuse was separated from the Waal through a lock and diverted into a new outlet called ''Bergse Maas'', then Amer and then flows into the former bay Hollands Diep. CANNOTANSWER

\begin{figure}[t] \small \begin{tcolorbox}[boxsep=0pt,left=5pt,right=0pt,top=2pt,colback = yellow!5] \begin{dialogue}
 \small 
 \speak{Student}{\bf What flood impacted the Meuse? }
\speak{Teacher}\colorbox{pink!25}{$\hookrightarrow$}
{ ``'' (St. Elizabeth's ) }
\\
\speak{Student}{\bf What year did the flood that impacted the Meuse take place? }
\speak{Teacher}\colorbox{pink!25}{$\hookrightarrow$}
{ ``'' (1421 ) }
\\
\speak{Student}{\bf Where did the Meuse flow before the flood?  }
\speak{Teacher}\colorbox{pink!25}{$\hookrightarrow$}
{ ``'' (Merwede-Oude Maas ) }
\\
\speak{Student}{\bf What did the Merwede-Oude Maas form with Waal and Lek? }
\speak{Teacher}\colorbox{pink!25}{$\hookrightarrow$}
{ ``'' (archipelago-like estuary ) }
\\
\speak{Student}{\bf When did the Meuse and Waal merge? }
\speak{Teacher}\colorbox{pink!25}{$\hookrightarrow$}
{ ``'' (1421 to 1904 ) }
\\
\speak{Student}{\bf When did the Meuse and Waal diverge further upstream at Gorinchem? }
\speak{Teacher}\colorbox{pink!25}{$\hookrightarrow$}
{ ``'' (CANNOTANSWER ) }
\\
\speak{Student}{\bf What year was St. Elizabeth born? }
\speak{Teacher}\colorbox{pink!25}{$\hookrightarrow$}
{ ``'' (CANNOTANSWER ) }
\\
\speak{Student}{\bf What year was it when St. Elizabeth flooded part of the North Sea? }
\speak{Teacher}\colorbox{pink!25}{$\hookrightarrow$}
{ ``'' (CANNOTANSWER ) }
\\
\speak{Student}{\bf What flows out of the former bay Hollands Diep? }
\speak{Teacher}\colorbox{pink!25}{$\hookrightarrow$}
{ ``'' (CANNOTANSWER ) }
\\
\speak{Student}{\bf From what years did St. Elizabeth live in the Hollands Diep bay? }
\speak{Teacher}\colorbox{pink!25}{$\hookrightarrow$}
{ ``'' (CANNOTANSWER ) }
\\
 \end{dialogue}\end{tcolorbox}\end{figure}

\end{document}

