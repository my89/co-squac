\documentclass[11pt,a4paper, onecolumn]{article}
\usepackage{times}
\usepackage{latexsym}
\usepackage{url}
\usepackage{textcomp}
\usepackage{bbm}
\usepackage{amsmath}
\usepackage{booktabs}
\usepackage{tabularx}
\usepackage{graphicx}
\usepackage{dialogue}
\usepackage{mathtools}
\usepackage{hyperref}
%\hypersetup{draft}

\usepackage{multirow}
\usepackage{mdframed}
\usepackage{tcolorbox}

\usepackage{xcolor,pifont}
%\newcommand{\cmark}{\ding{51}}
%\newcommand{\xmark}{\ding{55}}

\setcounter{topnumber}{2}
\setcounter{bottomnumber}{2}
\setcounter{totalnumber}{4}
\renewcommand{\topfraction}{0.75}
\renewcommand{\bottomfraction}{0.75}
\renewcommand{\textfraction}{0.05}
\renewcommand{\floatpagefraction}{0.6}

\newcommand\cmark {\textcolor{green}{\ding{52}}}
\newcommand\xmark {\textcolor{red}{\ding{55}}}
\mdfdefinestyle{dialogue}{
    backgroundcolor=yellow!20,
    innermargin=5pt
}
\usepackage{amssymb}
\usepackage{soul}
\makeatletter

\begin{document}

\hspace{15pt}{\textbf{Section}:Yuan dynasty44\\}
\\ Context: The reason for the order of the classes and the reason why people were placed in a certain class was the date they surrendered to the Mongols, and had nothing to do with their ethnicity. The earlier they surrendered to the Mongols, the higher they were placed, the more the held out, the lower they were ranked. The Northern Chinese were ranked higher and Southern Chinese were ranked lower because southern China withstood and fought to the last before caving in. Major commerce during this era gave rise to favorable conditions for private southern Chinese manufacturers and merchants. CANNOTANSWER

\begin{figure}[t] \small \begin{tcolorbox}[boxsep=0pt,left=5pt,right=0pt,top=2pt,colback = yellow!5] \begin{dialogue}
 \small 
 \speak{Student}{\bf Which part of China had people ranked higher in the class system? }
\speak{Teacher}\colorbox{pink!25}{$\hookrightarrow$}
{ ``'' (Northern ) }
\\
\speak{Student}{\bf Which part of China had people ranked lower in the class system? }
\speak{Teacher}\colorbox{pink!25}{$\hookrightarrow$}
{ ``'' (Southern ) }
\\
\speak{Student}{\bf Why were Southern Chinese ranked lower? }
\speak{Teacher}\colorbox{pink!25}{$\hookrightarrow$}
{ ``'' (southern China withstood and fought to the last ) }
\\
\speak{Student}{\bf Why were Northern Chinese ranked higher? }
\speak{Teacher}\colorbox{pink!25}{$\hookrightarrow$}
{ ``'' (The earlier they surrendered to the Mongols, the higher they were placed ) }
\\
\speak{Student}{\bf Who did the Yuan's increase in commerce help? }
\speak{Teacher}\colorbox{pink!25}{$\hookrightarrow$}
{ ``'' (private southern Chinese manufacturers and merchants ) }
\\
\speak{Student}{\bf Which part of Japan had people ranked higher in the class system? }
\speak{Teacher}\colorbox{pink!25}{$\hookrightarrow$}
{ ``'' (CANNOTANSWER ) }
\\
\speak{Student}{\bf Which part of Japan had people ranked lower in the class system? }
\speak{Teacher}\colorbox{pink!25}{$\hookrightarrow$}
{ ``'' (CANNOTANSWER ) }
\\
\speak{Student}{\bf  Why were Eastern Chinese ranked lower? }
\speak{Teacher}\colorbox{pink!25}{$\hookrightarrow$}
{ ``'' (CANNOTANSWER ) }
\\
\speak{Student}{\bf Why were Western Chinese ranked higher? }
\speak{Teacher}\colorbox{pink!25}{$\hookrightarrow$}
{ ``'' (CANNOTANSWER ) }
\\
\speak{Student}{\bf  Who did the Yuan's decrease in commerce help? }
\speak{Teacher}\colorbox{pink!25}{$\hookrightarrow$}
{ ``'' (CANNOTANSWER ) }
\\
 \end{dialogue}\end{tcolorbox}\end{figure}

\end{document}

