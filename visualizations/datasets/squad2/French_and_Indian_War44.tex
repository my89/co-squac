\documentclass[11pt,a4paper, onecolumn]{article}
\usepackage{times}
\usepackage{latexsym}
\usepackage{url}
\usepackage{textcomp}
\usepackage{bbm}
\usepackage{amsmath}
\usepackage{booktabs}
\usepackage{tabularx}
\usepackage{graphicx}
\usepackage{dialogue}
\usepackage{mathtools}
\usepackage{hyperref}
%\hypersetup{draft}

\usepackage{multirow}
\usepackage{mdframed}
\usepackage{tcolorbox}

\usepackage{xcolor,pifont}
%\newcommand{\cmark}{\ding{51}}
%\newcommand{\xmark}{\ding{55}}

\setcounter{topnumber}{2}
\setcounter{bottomnumber}{2}
\setcounter{totalnumber}{4}
\renewcommand{\topfraction}{0.75}
\renewcommand{\bottomfraction}{0.75}
\renewcommand{\textfraction}{0.05}
\renewcommand{\floatpagefraction}{0.6}

\newcommand\cmark {\textcolor{green}{\ding{52}}}
\newcommand\xmark {\textcolor{red}{\ding{55}}}
\mdfdefinestyle{dialogue}{
    backgroundcolor=yellow!20,
    innermargin=5pt
}
\usepackage{amssymb}
\usepackage{soul}
\makeatletter

\begin{document}

\hspace{15pt}{\textbf{Section}:French and Indian War44\\}
\\ Context: Following the treaty, King George III issued the Royal Proclamation of 1763 on October 7, 1763, which outlined the division and administration of the newly conquered territory, and to some extent continues to govern relations between the government of modern Canada and the First Nations. Included in its provisions was the reservation of lands west of the Appalachian Mountains to its Indian population, a demarcation that was at best a temporary impediment to a rising tide of westward-bound settlers. The proclamation also contained provisions that prevented civic participation by the Roman Catholic Canadians. When accommodations were made in the Quebec Act in 1774 to address this and other issues, religious concerns were raised in the largely Protestant Thirteen Colonies over the advance of ''popery''; the Act maintained French Civil law, including the seigneurial system, a medieval code soon to be removed from France within a generation by the French Revolution. CANNOTANSWER

\begin{figure}[t] \small \begin{tcolorbox}[boxsep=0pt,left=5pt,right=0pt,top=2pt,colback = yellow!5] \begin{dialogue}
 \small 
 \speak{Student}{\bf Who issued the Royal Proclamation of 1763? }
\speak{Teacher}\colorbox{pink!25}{$\hookrightarrow$}
{ ``'' (King George III ) }
\\
\speak{Student}{\bf What was the objective of Royal Proclamation of 1763? }
\speak{Teacher}\colorbox{pink!25}{$\hookrightarrow$}
{ ``'' (outlined the division and administration of the newly conquered territory ) }
\\
\speak{Student}{\bf What lands were reserved for natives? }
\speak{Teacher}\colorbox{pink!25}{$\hookrightarrow$}
{ ``'' (west of the Appalachian Mountains ) }
\\
\speak{Student}{\bf Who never issued the Royal Proclamation of 1763? }
\speak{Teacher}\colorbox{pink!25}{$\hookrightarrow$}
{ ``'' (CANNOTANSWER ) }
\\
\speak{Student}{\bf Who issued the Royal Proclamation of 1736? }
\speak{Teacher}\colorbox{pink!25}{$\hookrightarrow$}
{ ``'' (CANNOTANSWER ) }
\\
\speak{Student}{\bf What was the objective of Royal Proclamation of 1736? }
\speak{Teacher}\colorbox{pink!25}{$\hookrightarrow$}
{ ``'' (CANNOTANSWER ) }
\\
\speak{Student}{\bf What lands weren't reserved for natives? }
\speak{Teacher}\colorbox{pink!25}{$\hookrightarrow$}
{ ``'' (CANNOTANSWER ) }
\\
\speak{Student}{\bf What lands were reserved for the French? }
\speak{Teacher}\colorbox{pink!25}{$\hookrightarrow$}
{ ``'' (CANNOTANSWER ) }
\\
 \end{dialogue}\end{tcolorbox}\end{figure}

\end{document}

