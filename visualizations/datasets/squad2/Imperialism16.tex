\documentclass[11pt,a4paper, onecolumn]{article}
\usepackage{times}
\usepackage{latexsym}
\usepackage{url}
\usepackage{textcomp}
\usepackage{bbm}
\usepackage{amsmath}
\usepackage{booktabs}
\usepackage{tabularx}
\usepackage{graphicx}
\usepackage{dialogue}
\usepackage{mathtools}
\usepackage{hyperref}
%\hypersetup{draft}

\usepackage{multirow}
\usepackage{mdframed}
\usepackage{tcolorbox}

\usepackage{xcolor,pifont}
%\newcommand{\cmark}{\ding{51}}
%\newcommand{\xmark}{\ding{55}}

\setcounter{topnumber}{2}
\setcounter{bottomnumber}{2}
\setcounter{totalnumber}{4}
\renewcommand{\topfraction}{0.75}
\renewcommand{\bottomfraction}{0.75}
\renewcommand{\textfraction}{0.05}
\renewcommand{\floatpagefraction}{0.6}

\newcommand\cmark {\textcolor{green}{\ding{52}}}
\newcommand\xmark {\textcolor{red}{\ding{55}}}
\mdfdefinestyle{dialogue}{
    backgroundcolor=yellow!20,
    innermargin=5pt
}
\usepackage{amssymb}
\usepackage{soul}
\makeatletter

\begin{document}

\hspace{15pt}{\textbf{Section}:Imperialism16\\}
\\ Context: Europe's expansion into territorial imperialism was largely focused on economic growth by collecting resources from colonies, in combination with assuming political control by military and political means. The colonization of India in the mid-18th century offers an example of this focus: there, the ''British exploited the political weakness of the Mughal state, and, while military activity was important at various times, the economic and administrative incorporation of local elites was also of crucial significance'' for the establishment of control over the subcontinent's resources, markets, and manpower. Although a substantial number of colonies had been designed to provide economic profit and to ship resources to home ports in the seventeenth and eighteenth centuries, Fieldhouse suggests that in the nineteenth and twentieth centuries in places such as Africa and Asia, this idea is not necessarily valid: CANNOTANSWER

\begin{figure}[t] \small \begin{tcolorbox}[boxsep=0pt,left=5pt,right=0pt,top=2pt,colback = yellow!5] \begin{dialogue}
 \small 
 \speak{Student}{\bf European imperialism was focused on what? }
\speak{Teacher}\colorbox{pink!25}{$\hookrightarrow$}
{ ``'' (economic growth ) }
\\
\speak{Student}{\bf What did European empires rely on to supply them with resources? }
\speak{Teacher}\colorbox{pink!25}{$\hookrightarrow$}
{ ``'' (colonies ) }
\\
\speak{Student}{\bf When did the colonization of India occur? }
\speak{Teacher}\colorbox{pink!25}{$\hookrightarrow$}
{ ``'' (mid-18th century ) }
\\
\speak{Student}{\bf Who did Britain exploit in India? }
\speak{Teacher}\colorbox{pink!25}{$\hookrightarrow$}
{ ``'' (the Mughal state ) }
\\
\speak{Student}{\bf European imperialism was never focused on what? }
\speak{Teacher}\colorbox{pink!25}{$\hookrightarrow$}
{ ``'' (CANNOTANSWER ) }
\\
\speak{Student}{\bf  What did European empires not rely on to supply them with resources? }
\speak{Teacher}\colorbox{pink!25}{$\hookrightarrow$}
{ ``'' (CANNOTANSWER ) }
\\
\speak{Student}{\bf  When did the colonization of India not occur? }
\speak{Teacher}\colorbox{pink!25}{$\hookrightarrow$}
{ ``'' (CANNOTANSWER ) }
\\
\speak{Student}{\bf  Who did Britain not exploit in India? }
\speak{Teacher}\colorbox{pink!25}{$\hookrightarrow$}
{ ``'' (CANNOTANSWER ) }
\\
\speak{Student}{\bf  What was made valid in the late 19th and 20th centuries? }
\speak{Teacher}\colorbox{pink!25}{$\hookrightarrow$}
{ ``'' (CANNOTANSWER ) }
\\
 \end{dialogue}\end{tcolorbox}\end{figure}

\end{document}

