\documentclass[11pt,a4paper, onecolumn]{article}
\usepackage{times}
\usepackage{latexsym}
\usepackage{url}
\usepackage{textcomp}
\usepackage{bbm}
\usepackage{amsmath}
\usepackage{booktabs}
\usepackage{tabularx}
\usepackage{graphicx}
\usepackage{dialogue}
\usepackage{mathtools}
\usepackage{hyperref}
%\hypersetup{draft}

\usepackage{multirow}
\usepackage{mdframed}
\usepackage{tcolorbox}

\usepackage{xcolor,pifont}
%\newcommand{\cmark}{\ding{51}}
%\newcommand{\xmark}{\ding{55}}

\setcounter{topnumber}{2}
\setcounter{bottomnumber}{2}
\setcounter{totalnumber}{4}
\renewcommand{\topfraction}{0.75}
\renewcommand{\bottomfraction}{0.75}
\renewcommand{\textfraction}{0.05}
\renewcommand{\floatpagefraction}{0.6}

\newcommand\cmark {\textcolor{green}{\ding{52}}}
\newcommand\xmark {\textcolor{red}{\ding{55}}}
\mdfdefinestyle{dialogue}{
    backgroundcolor=yellow!20,
    innermargin=5pt
}
\usepackage{amssymb}
\usepackage{soul}
\makeatletter

\begin{document}

\hspace{15pt}{\textbf{Section}:Force26\\}
\\ Context: It was only the orbit of the planet Mercury that Newton's Law of Gravitation seemed not to fully explain. Some astrophysicists predicted the existence of another planet (Vulcan) that would explain the discrepancies; however, despite some early indications, no such planet could be found. When Albert Einstein formulated his theory of general relativity (GR) he turned his attention to the problem of Mercury's orbit and found that his theory added a correction, which could account for the discrepancy. This was the first time that Newton's Theory of Gravity had been shown to be less correct than an alternative. CANNOTANSWER

\begin{figure}[t] \small \begin{tcolorbox}[boxsep=0pt,left=5pt,right=0pt,top=2pt,colback = yellow!5] \begin{dialogue}
 \small 
 \speak{Student}{\bf What planet seemed to buck Newton's gravitational laws? }
\speak{Teacher}\colorbox{pink!25}{$\hookrightarrow$}
{ ``'' (Mercury ) }
\\
\speak{Student}{\bf What planet did astrophysisist predict to explain the problems with Mercury? }
\speak{Teacher}\colorbox{pink!25}{$\hookrightarrow$}
{ ``'' (Vulcan ) }
\\
\speak{Student}{\bf What theory accounted for the Mercury problem? }
\speak{Teacher}\colorbox{pink!25}{$\hookrightarrow$}
{ ``'' (theory of general relativity ) }
\\
\speak{Student}{\bf Who came up with the theory of relativity? }
\speak{Teacher}\colorbox{pink!25}{$\hookrightarrow$}
{ ``'' (Albert Einstein ) }
\\
\speak{Student}{\bf Who first showed that Newton's Theory of Gravity was not as correct as another theory? }
\speak{Teacher}\colorbox{pink!25}{$\hookrightarrow$}
{ ``'' (Albert Einstein ) }
\\
\speak{Student}{\bf Whose law did not explain the orbit of the planet Saturn? }
\speak{Teacher}\colorbox{pink!25}{$\hookrightarrow$}
{ ``'' (CANNOTANSWER ) }
\\
\speak{Student}{\bf Who predicted the existence of many other planets? }
\speak{Teacher}\colorbox{pink!25}{$\hookrightarrow$}
{ ``'' (CANNOTANSWER ) }
\\
\speak{Student}{\bf Albert Einstein formulated what law? }
\speak{Teacher}\colorbox{pink!25}{$\hookrightarrow$}
{ ``'' (CANNOTANSWER ) }
\\
\speak{Student}{\bf The planet Vulcan was predicted to explain the what with planet Saturn? }
\speak{Teacher}\colorbox{pink!25}{$\hookrightarrow$}
{ ``'' (CANNOTANSWER ) }
\\
 \end{dialogue}\end{tcolorbox}\end{figure}

\end{document}

