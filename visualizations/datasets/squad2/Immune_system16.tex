\documentclass[11pt,a4paper, onecolumn]{article}
\usepackage{times}
\usepackage{latexsym}
\usepackage{url}
\usepackage{textcomp}
\usepackage{bbm}
\usepackage{amsmath}
\usepackage{booktabs}
\usepackage{tabularx}
\usepackage{graphicx}
\usepackage{dialogue}
\usepackage{mathtools}
\usepackage{hyperref}
%\hypersetup{draft}

\usepackage{multirow}
\usepackage{mdframed}
\usepackage{tcolorbox}

\usepackage{xcolor,pifont}
%\newcommand{\cmark}{\ding{51}}
%\newcommand{\xmark}{\ding{55}}

\setcounter{topnumber}{2}
\setcounter{bottomnumber}{2}
\setcounter{totalnumber}{4}
\renewcommand{\topfraction}{0.75}
\renewcommand{\bottomfraction}{0.75}
\renewcommand{\textfraction}{0.05}
\renewcommand{\floatpagefraction}{0.6}

\newcommand\cmark {\textcolor{green}{\ding{52}}}
\newcommand\xmark {\textcolor{red}{\ding{55}}}
\mdfdefinestyle{dialogue}{
    backgroundcolor=yellow!20,
    innermargin=5pt
}
\usepackage{amssymb}
\usepackage{soul}
\makeatletter

\begin{document}

\hspace{15pt}{\textbf{Section}:Immune system16\\}
\\ Context: Both B cells and T cells carry receptor molecules that recognize specific targets. T cells recognize a ''non-self'' target, such as a pathogen, only after antigens (small fragments of the pathogen) have been processed and presented in combination with a ''self'' receptor called a major histocompatibility complex (MHC) molecule. There are two major subtypes of T cells: the killer T cell and the helper T cell. In addition there are regulatory T cells which have a role in modulating immune response. Killer T cells only recognize antigens coupled to Class I MHC molecules, while helper T cells and regulatory T cells only recognize antigens coupled to Class II MHC molecules. These two mechanisms of antigen presentation reflect the different roles of the two types of T cell. A third, minor subtype are the γδ T cells that recognize intact antigens that are not bound to MHC receptors. CANNOTANSWER

\begin{figure}[t] \small \begin{tcolorbox}[boxsep=0pt,left=5pt,right=0pt,top=2pt,colback = yellow!5] \begin{dialogue}
 \small 
 \speak{Student}{\bf What are the two major subtypes of T cells? }
\speak{Teacher}\colorbox{pink!25}{$\hookrightarrow$}
{ ``'' (killer T cell and the helper T cell ) }
\\
\speak{Student}{\bf What kind of T cells have the purpose of modulating the immune response? }
\speak{Teacher}\colorbox{pink!25}{$\hookrightarrow$}
{ ``'' (regulatory T cells ) }
\\
\speak{Student}{\bf Killer T cells can only recognize antigens coupled to what kind of molecules? }
\speak{Teacher}\colorbox{pink!25}{$\hookrightarrow$}
{ ``'' (Class I MHC molecules ) }
\\
\speak{Student}{\bf Helper and regulatory T cells can only recognize antigens coupled to what kind of molecules? }
\speak{Teacher}\colorbox{pink!25}{$\hookrightarrow$}
{ ``'' (Class II MHC molecules ) }
\\
\speak{Student}{\bf What class of T cells recognizes intact antigens that are not associated with MHC receptors? }
\speak{Teacher}\colorbox{pink!25}{$\hookrightarrow$}
{ ``'' (γδ T cells ) }
\\
\speak{Student}{\bf What cells do not carry receptor molecules? }
\speak{Teacher}\colorbox{pink!25}{$\hookrightarrow$}
{ ``'' (CANNOTANSWER ) }
\\
\speak{Student}{\bf What do T cells recognize before antigens have been processed? }
\speak{Teacher}\colorbox{pink!25}{$\hookrightarrow$}
{ ``'' (CANNOTANSWER ) }
\\
\speak{Student}{\bf How many subtypes of B cells exist? }
\speak{Teacher}\colorbox{pink!25}{$\hookrightarrow$}
{ ``'' (CANNOTANSWER ) }
\\
\speak{Student}{\bf How many roles do the types of B cell have? }
\speak{Teacher}\colorbox{pink!25}{$\hookrightarrow$}
{ ``'' (CANNOTANSWER ) }
\\
\speak{Student}{\bf What do killer B cells recognize? }
\speak{Teacher}\colorbox{pink!25}{$\hookrightarrow$}
{ ``'' (CANNOTANSWER ) }
\\
 \end{dialogue}\end{tcolorbox}\end{figure}

\end{document}

