\documentclass[11pt,a4paper, onecolumn]{article}
\usepackage{times}
\usepackage{latexsym}
\usepackage{url}
\usepackage{textcomp}
\usepackage{bbm}
\usepackage{amsmath}
\usepackage{booktabs}
\usepackage{tabularx}
\usepackage{graphicx}
\usepackage{dialogue}
\usepackage{mathtools}
\usepackage{hyperref}
%\hypersetup{draft}

\usepackage{multirow}
\usepackage{mdframed}
\usepackage{tcolorbox}

\usepackage{xcolor,pifont}
%\newcommand{\cmark}{\ding{51}}
%\newcommand{\xmark}{\ding{55}}

\setcounter{topnumber}{2}
\setcounter{bottomnumber}{2}
\setcounter{totalnumber}{4}
\renewcommand{\topfraction}{0.75}
\renewcommand{\bottomfraction}{0.75}
\renewcommand{\textfraction}{0.05}
\renewcommand{\floatpagefraction}{0.6}

\newcommand\cmark {\textcolor{green}{\ding{52}}}
\newcommand\xmark {\textcolor{red}{\ding{55}}}
\mdfdefinestyle{dialogue}{
    backgroundcolor=yellow!20,
    innermargin=5pt
}
\usepackage{amssymb}
\usepackage{soul}
\makeatletter

\begin{document}

\hspace{15pt}{\textbf{Section}:Steam engine7\\}
\\ Context: In 1781 James Watt patented a steam engine that produced continuous rotary motion. Watt's ten-horsepower engines enabled a wide range of manufacturing machinery to be powered. The engines could be sited anywhere that water and coal or wood fuel could be obtained. By 1883, engines that could provide 10,000 hp had become feasible. The stationary steam engine was a key component of the Industrial Revolution, allowing factories to locate where water power was unavailable. The atmospheric engines of Newcomen and Watt were large compared to the amount of power they produced, but high pressure steam engines were light enough to be applied to vehicles such as traction engines and the railway locomotives. CANNOTANSWER

\begin{figure}[t] \small \begin{tcolorbox}[boxsep=0pt,left=5pt,right=0pt,top=2pt,colback = yellow!5] \begin{dialogue}
 \small 
 \speak{Student}{\bf Who patented a steam engine in 1781? }
\speak{Teacher}\colorbox{pink!25}{$\hookrightarrow$}
{ ``'' (James Watt ) }
\\
\speak{Student}{\bf What sort of motion did Watt's steam engine continuously produce? }
\speak{Teacher}\colorbox{pink!25}{$\hookrightarrow$}
{ ``'' (rotary ) }
\\
\speak{Student}{\bf How many horsepower was Watt's engine? }
\speak{Teacher}\colorbox{pink!25}{$\hookrightarrow$}
{ ``'' (ten ) }
\\
\speak{Student}{\bf As of what year were 10000 horsepower engines available? }
\speak{Teacher}\colorbox{pink!25}{$\hookrightarrow$}
{ ``'' (1883 ) }
\\
\speak{Student}{\bf What was the steam engine an important component of? }
\speak{Teacher}\colorbox{pink!25}{$\hookrightarrow$}
{ ``'' (Industrial Revolution ) }
\\
\speak{Student}{\bf Who patented a steam engine in 1883? }
\speak{Teacher}\colorbox{pink!25}{$\hookrightarrow$}
{ ``'' (CANNOTANSWER ) }
\\
\speak{Student}{\bf What sort of motion did Newcomen's steam engine continuously produce? }
\speak{Teacher}\colorbox{pink!25}{$\hookrightarrow$}
{ ``'' (CANNOTANSWER ) }
\\
\speak{Student}{\bf How many horsepower was Newcomen's engine? }
\speak{Teacher}\colorbox{pink!25}{$\hookrightarrow$}
{ ``'' (CANNOTANSWER ) }
\\
\speak{Student}{\bf As of what year were 1700 horsepower engines available? }
\speak{Teacher}\colorbox{pink!25}{$\hookrightarrow$}
{ ``'' (CANNOTANSWER ) }
\\
\speak{Student}{\bf What was the high pressure engine an important component of? }
\speak{Teacher}\colorbox{pink!25}{$\hookrightarrow$}
{ ``'' (CANNOTANSWER ) }
\\
 \end{dialogue}\end{tcolorbox}\end{figure}

\end{document}

