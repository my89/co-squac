\documentclass[11pt,a4paper, onecolumn]{article}
\usepackage{times}
\usepackage{latexsym}
\usepackage{url}
\usepackage{textcomp}
\usepackage{bbm}
\usepackage{amsmath}
\usepackage{booktabs}
\usepackage{tabularx}
\usepackage{graphicx}
\usepackage{dialogue}
\usepackage{mathtools}
\usepackage{hyperref}
%\hypersetup{draft}

\usepackage{multirow}
\usepackage{mdframed}
\usepackage{tcolorbox}

\usepackage{xcolor,pifont}
%\newcommand{\cmark}{\ding{51}}
%\newcommand{\xmark}{\ding{55}}

\setcounter{topnumber}{2}
\setcounter{bottomnumber}{2}
\setcounter{totalnumber}{4}
\renewcommand{\topfraction}{0.75}
\renewcommand{\bottomfraction}{0.75}
\renewcommand{\textfraction}{0.05}
\renewcommand{\floatpagefraction}{0.6}

\newcommand\cmark {\textcolor{green}{\ding{52}}}
\newcommand\xmark {\textcolor{red}{\ding{55}}}
\mdfdefinestyle{dialogue}{
    backgroundcolor=yellow!20,
    innermargin=5pt
}
\usepackage{amssymb}
\usepackage{soul}
\makeatletter

\begin{document}

\hspace{15pt}{\textbf{Section}:Steam engine45\\}
\\ Context: The Rankine cycle is sometimes referred to as a practical Carnot cycle because, when an efficient turbine is used, the TS diagram begins to resemble the Carnot cycle. The main difference is that heat addition (in the boiler) and rejection (in the condenser) are isobaric (constant pressure) processes in the Rankine cycle and isothermal (constant temperature) processes in the theoretical Carnot cycle. In this cycle a pump is used to pressurize the working fluid which is received from the condenser as a liquid not as a gas. Pumping the working fluid in liquid form during the cycle requires a small fraction of the energy to transport it compared to the energy needed to compress the working fluid in gaseous form in a compressor (as in the Carnot cycle). The cycle of a reciprocating steam engine differs from that of turbines because of condensation and re-evaporation occurring in the cylinder or in the steam inlet passages. CANNOTANSWER

\begin{figure}[t] \small \begin{tcolorbox}[boxsep=0pt,left=5pt,right=0pt,top=2pt,colback = yellow!5] \begin{dialogue}
 \small 
 \speak{Student}{\bf What is the Rankine cycle sometimes called? }
\speak{Teacher}\colorbox{pink!25}{$\hookrightarrow$}
{ ``'' (practical Carnot cycle ) }
\\
\speak{Student}{\bf Where does heat rejection occur in the Rankine cycle? }
\speak{Teacher}\colorbox{pink!25}{$\hookrightarrow$}
{ ``'' (in the condenser ) }
\\
\speak{Student}{\bf What does isobaric mean? }
\speak{Teacher}\colorbox{pink!25}{$\hookrightarrow$}
{ ``'' (constant pressure ) }
\\
\speak{Student}{\bf What is a term that means constant temperature? }
\speak{Teacher}\colorbox{pink!25}{$\hookrightarrow$}
{ ``'' (isothermal ) }
\\
\speak{Student}{\bf In the Rankine cycle, in what state is the working fluid received in the condenser? }
\speak{Teacher}\colorbox{pink!25}{$\hookrightarrow$}
{ ``'' (liquid ) }
\\
\speak{Student}{\bf What is the cycle condenser sometimes called? }
\speak{Teacher}\colorbox{pink!25}{$\hookrightarrow$}
{ ``'' (CANNOTANSWER ) }
\\
\speak{Student}{\bf Where does heat rejection occur in the Rankine cycle? }
\speak{Teacher}\colorbox{pink!25}{$\hookrightarrow$}
{ ``'' (CANNOTANSWER ) }
\\
\speak{Student}{\bf What does Carnot mean? }
\speak{Teacher}\colorbox{pink!25}{$\hookrightarrow$}
{ ``'' (CANNOTANSWER ) }
\\
\speak{Student}{\bf What is a term that means constant energy? }
\speak{Teacher}\colorbox{pink!25}{$\hookrightarrow$}
{ ``'' (CANNOTANSWER ) }
\\
\speak{Student}{\bf In the Rankine cycle, in what state is the working fluid received in the steam? }
\speak{Teacher}\colorbox{pink!25}{$\hookrightarrow$}
{ ``'' (CANNOTANSWER ) }
\\
 \end{dialogue}\end{tcolorbox}\end{figure}

\end{document}

