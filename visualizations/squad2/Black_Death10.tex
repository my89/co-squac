\documentclass[11pt,a4paper, onecolumn]{article}
\usepackage{times}
\usepackage{latexsym}
\usepackage{url}
\usepackage{textcomp}
\usepackage{bbm}
\usepackage{amsmath}
\usepackage{booktabs}
\usepackage{tabularx}
\usepackage{graphicx}
\usepackage{dialogue}
\usepackage{mathtools}
\usepackage{hyperref}
%\hypersetup{draft}

\usepackage{multirow}
\usepackage{mdframed}
\usepackage{tcolorbox}

\usepackage{xcolor,pifont}
%\newcommand{\cmark}{\ding{51}}
%\newcommand{\xmark}{\ding{55}}

\setcounter{topnumber}{2}
\setcounter{bottomnumber}{2}
\setcounter{totalnumber}{4}
\renewcommand{\topfraction}{0.75}
\renewcommand{\bottomfraction}{0.75}
\renewcommand{\textfraction}{0.05}
\renewcommand{\floatpagefraction}{0.6}

\newcommand\cmark {\textcolor{green}{\ding{52}}}
\newcommand\xmark {\textcolor{red}{\ding{55}}}
\mdfdefinestyle{dialogue}{
    backgroundcolor=yellow!20,
    innermargin=5pt
}
\usepackage{amssymb}
\usepackage{soul}
\makeatletter

\begin{document}

\hspace{15pt}{\textbf{Section}:Black Death10\\}
\\ Context: In October 2010, the open-access scientific journal PLoS Pathogens published a paper by a multinational team who undertook a new investigation into the role of Yersinia pestis in the Black Death following the disputed identification by Drancourt and Raoult in 1998. They assessed the presence of DNA/RNA with Polymerase Chain Reaction (PCR) techniques for Y. pestis from the tooth sockets in human skeletons from mass graves in northern, central and southern Europe that were associated archaeologically with the Black Death and subsequent resurgences. The authors concluded that this new research, together with prior analyses from the south of France and Germany, ''. . . ends the debate about the etiology of the Black Death, and unambiguously demonstrates that Y. pestis was the causative agent of the epidemic plague that devastated Europe during the Middle Ages''. CANNOTANSWER

\begin{figure}[t] \small \begin{tcolorbox}[boxsep=0pt,left=5pt,right=0pt,top=2pt,colback = yellow!5] \begin{dialogue}
 \small 
 \speak{Student}{\bf When did the Plos Pathogens paper come out? }
\speak{Teacher}\colorbox{pink!25}{$\hookrightarrow$}
{ In October 2010 }
\\
\speak{Student}{\bf What was the Plos Pathogens paper about? }
\speak{Teacher}\colorbox{pink!25}{$\hookrightarrow$}
{ a new investigation into the role of Yersinia pestis in the Black Death }
\\
\speak{Student}{\bf How did scientists assess the DNA/RNA of yersinia pestis?  }
\speak{Teacher}\colorbox{pink!25}{$\hookrightarrow$}
{ with Polymerase Chain Reaction (PCR) }
\\
\speak{Student}{\bf Where did scientists find their Y. pestis sample?  }
\speak{Teacher}\colorbox{pink!25}{$\hookrightarrow$}
{ from the tooth sockets in human skeletons }
\\
\speak{Student}{\bf What does the plos pathogen paper claim? }
\speak{Teacher}\colorbox{pink!25}{$\hookrightarrow$}
{ unambiguously demonstrates that Y. pestis was the causative agent of the epidemic plague }
\\
\speak{Student}{\bf In what year was PloS Pathogens first published? }
\speak{Teacher}\colorbox{pink!25}{$\hookrightarrow$}
{ CANNOTANSWER }
\\
\speak{Student}{\bf In what country is PloS Pathogens headquartered? }
\speak{Teacher}\colorbox{pink!25}{$\hookrightarrow$}
{ CANNOTANSWER }
\\
\speak{Student}{\bf In what year were Polymerase Chain Reactions first used by researchers? }
\speak{Teacher}\colorbox{pink!25}{$\hookrightarrow$}
{ CANNOTANSWER }
 \end{dialogue}\end{tcolorbox}\end{figure}\begin{figure}[t] \small \begin{tcolorbox}[boxsep=0pt,left=5pt,right=0pt,top=2pt,colback = yellow!5] \begin{dialogue}
 \small 
 \speak{Student}{\bf What scientific journal in France published a prior analysis of the Black Death? }
\speak{Teacher}\colorbox{pink!25}{$\hookrightarrow$}
{ CANNOTANSWER }
\\
\speak{Student}{\bf In what month was Drancourt and Raoult's research published in 1998? }
\speak{Teacher}\colorbox{pink!25}{$\hookrightarrow$}
{ CANNOTANSWER }
\\
 \end{dialogue}\end{tcolorbox}\end{figure}

\end{document}

