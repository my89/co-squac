\documentclass[11pt,a4paper, onecolumn]{article}
\usepackage{times}
\usepackage{latexsym}
\usepackage{url}
\usepackage{textcomp}
\usepackage{bbm}
\usepackage{amsmath}
\usepackage{booktabs}
\usepackage{tabularx}
\usepackage{graphicx}
\usepackage{dialogue}
\usepackage{mathtools}
\usepackage{hyperref}
%\hypersetup{draft}

\usepackage{multirow}
\usepackage{mdframed}
\usepackage{tcolorbox}

\usepackage{xcolor,pifont}
%\newcommand{\cmark}{\ding{51}}
%\newcommand{\xmark}{\ding{55}}

\setcounter{topnumber}{2}
\setcounter{bottomnumber}{2}
\setcounter{totalnumber}{4}
\renewcommand{\topfraction}{0.75}
\renewcommand{\bottomfraction}{0.75}
\renewcommand{\textfraction}{0.05}
\renewcommand{\floatpagefraction}{0.6}

\newcommand\cmark {\textcolor{green}{\ding{52}}}
\newcommand\xmark {\textcolor{red}{\ding{55}}}
\mdfdefinestyle{dialogue}{
    backgroundcolor=yellow!20,
    innermargin=5pt
}
\usepackage{amssymb}
\usepackage{soul}
\makeatletter

\begin{document}

\hspace{15pt}{\textbf{Section}:Steam engine12\\}
\\ Context: The most useful instrument for analyzing the performance of steam engines is the steam engine indicator. Early versions were in use by 1851, but the most successful indicator was developed for the high speed engine inventor and manufacturer Charles Porter by Charles Richard and exhibited at London Exhibition in 1862. The steam engine indicator traces on paper the pressure in the cylinder throughout the cycle, which can be used to spot various problems and calculate developed horsepower. It was routinely used by engineers, mechanics and insurance inspectors. The engine indicator can also be used on internal combustion engines. See image of indicator diagram below (in Types of motor units section). CANNOTANSWER

\begin{figure}[t] \small \begin{tcolorbox}[boxsep=0pt,left=5pt,right=0pt,top=2pt,colback = yellow!5] \begin{dialogue}
 \small 
 \speak{Student}{\bf What instrument is used to examine steam engine performance? }
\speak{Teacher}\colorbox{pink!25}{$\hookrightarrow$}
{ steam engine indicator }
\\
\speak{Student}{\bf What year saw the earliest recorded use of the steam engine indicator? }
\speak{Teacher}\colorbox{pink!25}{$\hookrightarrow$}
{ 1851 }
\\
\speak{Student}{\bf What company developed the most successful steam engine indicator? }
\speak{Teacher}\colorbox{pink!25}{$\hookrightarrow$}
{ Charles Porter }
\\
\speak{Student}{\bf Who developed a successful steam engine indicator for Charles Porter? }
\speak{Teacher}\colorbox{pink!25}{$\hookrightarrow$}
{ Charles Richard }
\\
\speak{Student}{\bf Where was the Charles Porter steam engine indicator shown? }
\speak{Teacher}\colorbox{pink!25}{$\hookrightarrow$}
{ London Exhibition }
\\
\speak{Student}{\bf What instrument is used to examine diagram performance? }
\speak{Teacher}\colorbox{pink!25}{$\hookrightarrow$}
{ CANNOTANSWER }
\\
\speak{Student}{\bf What year saw the earliest recorded use of steam engines? }
\speak{Teacher}\colorbox{pink!25}{$\hookrightarrow$}
{ CANNOTANSWER }
\\
\speak{Student}{\bf What company developed the most successful steam engine? }
\speak{Teacher}\colorbox{pink!25}{$\hookrightarrow$}
{ CANNOTANSWER }
 \end{dialogue}\end{tcolorbox}\end{figure}\begin{figure}[t] \small \begin{tcolorbox}[boxsep=0pt,left=5pt,right=0pt,top=2pt,colback = yellow!5] \begin{dialogue}
 \small 
 \speak{Student}{\bf Who developed a successful steam engine for Charles Porter? }
\speak{Teacher}\colorbox{pink!25}{$\hookrightarrow$}
{ CANNOTANSWER }
\\
\speak{Student}{\bf Where was the Charles Porter steam engine shown? }
\speak{Teacher}\colorbox{pink!25}{$\hookrightarrow$}
{ CANNOTANSWER }
\\
 \end{dialogue}\end{tcolorbox}\end{figure}

\end{document}

