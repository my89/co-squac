\documentclass[11pt,a4paper, onecolumn]{article}
\usepackage{times}
\usepackage{latexsym}
\usepackage{url}
\usepackage{textcomp}
\usepackage{bbm}
\usepackage{amsmath}
\usepackage{booktabs}
\usepackage{tabularx}
\usepackage{graphicx}
\usepackage{dialogue}
\usepackage{mathtools}
\usepackage{hyperref}
%\hypersetup{draft}

\usepackage{multirow}
\usepackage{mdframed}
\usepackage{tcolorbox}

\usepackage{xcolor,pifont}
%\newcommand{\cmark}{\ding{51}}
%\newcommand{\xmark}{\ding{55}}

\setcounter{topnumber}{2}
\setcounter{bottomnumber}{2}
\setcounter{totalnumber}{4}
\renewcommand{\topfraction}{0.75}
\renewcommand{\bottomfraction}{0.75}
\renewcommand{\textfraction}{0.05}
\renewcommand{\floatpagefraction}{0.6}

\newcommand\cmark {\textcolor{green}{\ding{52}}}
\newcommand\xmark {\textcolor{red}{\ding{55}}}
\mdfdefinestyle{dialogue}{
    backgroundcolor=yellow!20,
    innermargin=5pt
}
\usepackage{amssymb}
\usepackage{soul}
\makeatletter

\begin{document}

\hspace{15pt}{\textbf{Section}:Oxygen24\\}
\\ Context: In the triplet form, O
2 molecules are paramagnetic. That is, they impart magnetic character to oxygen when it is in the presence of a magnetic field, because of the spin magnetic moments of the unpaired electrons in the molecule, and the negative exchange energy between neighboring O
2 molecules. Liquid oxygen is attracted to a magnet to a sufficient extent that, in laboratory demonstrations, a bridge of liquid oxygen may be supported against its own weight between the poles of a powerful magnet.[c] CANNOTANSWER

\begin{figure}[t] \small \begin{tcolorbox}[boxsep=0pt,left=5pt,right=0pt,top=2pt,colback = yellow!5] \begin{dialogue}
 \small 
 \speak{Student}{\bf What magnetic character do triplet O2 have? }
\speak{Teacher}\colorbox{pink!25}{$\hookrightarrow$}
{ paramagnetic }
\\
\speak{Student}{\bf In experiments, a bridge of what element can be built between poles of a magnet? }
\speak{Teacher}\colorbox{pink!25}{$\hookrightarrow$}
{ Liquid oxygen }
\\
\speak{Student}{\bf The spin of what can produce a magnetic effect to oxygen molecules? }
\speak{Teacher}\colorbox{pink!25}{$\hookrightarrow$}
{ unpaired electrons }
\\
\speak{Student}{\bf What kind of field is necessary to produce a magnet effect in oxygen molecules? }
\speak{Teacher}\colorbox{pink!25}{$\hookrightarrow$}
{ magnetic field }
\\
\speak{Student}{\bf What device is used to test the magnetic attractions involved in liquid oxygen? }
\speak{Teacher}\colorbox{pink!25}{$\hookrightarrow$}
{ powerful magnet }
\\
\speak{Student}{\bf What are O molecules in triplet form? }
\speak{Teacher}\colorbox{pink!25}{$\hookrightarrow$}
{ CANNOTANSWER }
\\
\speak{Student}{\bf Why are O molecules paramagnetic? }
\speak{Teacher}\colorbox{pink!25}{$\hookrightarrow$}
{ CANNOTANSWER }
\\
\speak{Student}{\bf What is attracted to the poles of a powerful magnet? }
\speak{Teacher}\colorbox{pink!25}{$\hookrightarrow$}
{ CANNOTANSWER }
 \end{dialogue}\end{tcolorbox}\end{figure}

\end{document}

