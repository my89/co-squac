\documentclass[11pt,a4paper, onecolumn]{article}
\usepackage{times}
\usepackage{latexsym}
\usepackage{url}
\usepackage{textcomp}
\usepackage{bbm}
\usepackage{amsmath}
\usepackage{booktabs}
\usepackage{tabularx}
\usepackage{graphicx}
\usepackage{dialogue}
\usepackage{mathtools}
\usepackage{hyperref}
%\hypersetup{draft}

\usepackage{multirow}
\usepackage{mdframed}
\usepackage{tcolorbox}

\usepackage{xcolor,pifont}
%\newcommand{\cmark}{\ding{51}}
%\newcommand{\xmark}{\ding{55}}

\setcounter{topnumber}{2}
\setcounter{bottomnumber}{2}
\setcounter{totalnumber}{4}
\renewcommand{\topfraction}{0.75}
\renewcommand{\bottomfraction}{0.75}
\renewcommand{\textfraction}{0.05}
\renewcommand{\floatpagefraction}{0.6}

\newcommand\cmark {\textcolor{green}{\ding{52}}}
\newcommand\xmark {\textcolor{red}{\ding{55}}}
\mdfdefinestyle{dialogue}{
    backgroundcolor=yellow!20,
    innermargin=5pt
}
\usepackage{amssymb}
\usepackage{soul}
\makeatletter

\begin{document}

\hspace{15pt}{\textbf{Section}:Construction7\\}
\\ Context: New techniques of building construction are being researched, made possible by advances in 3D printing technology. In a form of additive building construction, similar to the additive manufacturing techniques for manufactured parts, building printing is making it possible to flexibly construct small commercial buildings and private habitations in around 20 hours, with built-in plumbing and electrical facilities, in one continuous build, using large 3D printers. Working versions of 3D-printing building technology are already printing 2 metres (6 ft 7 in) of building material per hour as of January 2013[update], with the next-generation printers capable of 3.5 metres (11 ft) per hour, sufficient to complete a building in a week. Dutch architect Janjaap Ruijssenaars's performative architecture 3D-printed building is scheduled to be built in 2014. CANNOTANSWER

\begin{figure}[t] \small \begin{tcolorbox}[boxsep=0pt,left=5pt,right=0pt,top=2pt,colback = yellow!5] \begin{dialogue}
 \small 
 \speak{Student}{\bf New techniques of building construction are being researched, made possible by advances in what? }
\speak{Teacher}\colorbox{pink!25}{$\hookrightarrow$}
{ 3D printing technology }
\\
\speak{Student}{\bf Building printing is making it possible to flexibly construct small commercial buildings and private habitations in what amount of time? }
\speak{Teacher}\colorbox{pink!25}{$\hookrightarrow$}
{ around 20 hours }
\\
\speak{Student}{\bf Dutch architect Janjaap Ruijssenaars's performative architecture 3D-printed building is scheduled to be built when? }
\speak{Teacher}\colorbox{pink!25}{$\hookrightarrow$}
{ Working versions of 3D-printing building technology are already printing }
\\
\speak{Student}{\bf Working versions of 3D-printing building technology are already printing how much building material per hour? }
\speak{Teacher}\colorbox{pink!25}{$\hookrightarrow$}
{ 2 metres (6 ft 7 in) }
\\
\speak{Student}{\bf What is Janjaap Ruijssenaar going to build in 2013? }
\speak{Teacher}\colorbox{pink!25}{$\hookrightarrow$}
{ CANNOTANSWER }
\\
\speak{Student}{\bf What makes research in built-in plumbing possible? }
\speak{Teacher}\colorbox{pink!25}{$\hookrightarrow$}
{ CANNOTANSWER }
\\
\speak{Student}{\bf What is flexible construction similar to? }
\speak{Teacher}\colorbox{pink!25}{$\hookrightarrow$}
{ CANNOTANSWER }
\\
\speak{Student}{\bf How long does it take to build electrical facilities? }
\speak{Teacher}\colorbox{pink!25}{$\hookrightarrow$}
{ CANNOTANSWER }
 \end{dialogue}\end{tcolorbox}\end{figure}\begin{figure}[t] \small \begin{tcolorbox}[boxsep=0pt,left=5pt,right=0pt,top=2pt,colback = yellow!5] \begin{dialogue}
 \small 
 \speak{Student}{\bf As of 2014 how quickly is building material being printed? }
\speak{Teacher}\colorbox{pink!25}{$\hookrightarrow$}
{ CANNOTANSWER }
\\
 \end{dialogue}\end{tcolorbox}\end{figure}

\end{document}

