\documentclass[11pt,a4paper, onecolumn]{article}
\usepackage{times}
\usepackage{latexsym}
\usepackage{url}
\usepackage{textcomp}
\usepackage{bbm}
\usepackage{amsmath}
\usepackage{booktabs}
\usepackage{tabularx}
\usepackage{graphicx}
\usepackage{dialogue}
\usepackage{mathtools}
\usepackage{hyperref}
%\hypersetup{draft}

\usepackage{multirow}
\usepackage{mdframed}
\usepackage{tcolorbox}

\usepackage{xcolor,pifont}
%\newcommand{\cmark}{\ding{51}}
%\newcommand{\xmark}{\ding{55}}

\setcounter{topnumber}{2}
\setcounter{bottomnumber}{2}
\setcounter{totalnumber}{4}
\renewcommand{\topfraction}{0.75}
\renewcommand{\bottomfraction}{0.75}
\renewcommand{\textfraction}{0.05}
\renewcommand{\floatpagefraction}{0.6}

\newcommand\cmark {\textcolor{green}{\ding{52}}}
\newcommand\xmark {\textcolor{red}{\ding{55}}}
\mdfdefinestyle{dialogue}{
    backgroundcolor=yellow!20,
    innermargin=5pt
}
\usepackage{amssymb}
\usepackage{soul}
\makeatletter

\begin{document}

\hspace{15pt}{\textbf{Section}:Yuan dynasty36\\}
\\ Context: The Chinese medical tradition of the Yuan had ''Four Great Schools'' that the Yuan inherited from the Jin dynasty. All four schools were based on the same intellectual foundation, but advocated different theoretical approaches toward medicine. Under the Mongols, the practice of Chinese medicine spread to other parts of the empire. Chinese physicians were brought along military campaigns by the Mongols as they expanded towards the west. Chinese medical techniques such as acupuncture, moxibustion, pulse diagnosis, and various herbal drugs and elixirs were transmitted westward to the Middle East and the rest of the empire. Several medical advances were made in the Yuan period. The physician Wei Yilin (1277–1347) invented a suspension method for reducing dislocated joints, which he performed using anesthetics. The Mongol physician Hu Sihui described the importance of a healthy diet in a 1330 medical treatise. CANNOTANSWER

\begin{figure}[t] \small \begin{tcolorbox}[boxsep=0pt,left=5pt,right=0pt,top=2pt,colback = yellow!5] \begin{dialogue}
 \small 
 \speak{Student}{\bf How many schools of medicine were recognized in China? }
\speak{Teacher}\colorbox{pink!25}{$\hookrightarrow$}
{ four }
\\
\speak{Student}{\bf How did the Yuan come to have the 4 schools of medicine? }
\speak{Teacher}\colorbox{pink!25}{$\hookrightarrow$}
{ inherited from the Jin dynasty }
\\
\speak{Student}{\bf How did Chinese medicine spread? }
\speak{Teacher}\colorbox{pink!25}{$\hookrightarrow$}
{ Chinese physicians were brought along military campaigns by the Mongols }
\\
\speak{Student}{\bf What techniques did Chinese medicine include? }
\speak{Teacher}\colorbox{pink!25}{$\hookrightarrow$}
{ acupuncture, moxibustion, pulse diagnosis, and various herbal drugs and elixirs }
\\
\speak{Student}{\bf When did Wei Yilin die? }
\speak{Teacher}\colorbox{pink!25}{$\hookrightarrow$}
{ 1347 }
\\
\speak{Student}{\bf How many schools of medicine were recognized in Japan? }
\speak{Teacher}\colorbox{pink!25}{$\hookrightarrow$}
{ CANNOTANSWER }
\\
\speak{Student}{\bf  How did the Yuan come to have the 8 schools of medicine? }
\speak{Teacher}\colorbox{pink!25}{$\hookrightarrow$}
{ CANNOTANSWER }
\\
\speak{Student}{\bf  How did Chinese medicine stay in one place? }
\speak{Teacher}\colorbox{pink!25}{$\hookrightarrow$}
{ CANNOTANSWER }
 \end{dialogue}\end{tcolorbox}\end{figure}\begin{figure}[t] \small \begin{tcolorbox}[boxsep=0pt,left=5pt,right=0pt,top=2pt,colback = yellow!5] \begin{dialogue}
 \small 
 \speak{Student}{\bf What techniques did Chinese medicine never include? }
\speak{Teacher}\colorbox{pink!25}{$\hookrightarrow$}
{ CANNOTANSWER }
\\
 \end{dialogue}\end{tcolorbox}\end{figure}

\end{document}

