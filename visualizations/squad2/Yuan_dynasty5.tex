\documentclass[11pt,a4paper, onecolumn]{article}
\usepackage{times}
\usepackage{latexsym}
\usepackage{url}
\usepackage{textcomp}
\usepackage{bbm}
\usepackage{amsmath}
\usepackage{booktabs}
\usepackage{tabularx}
\usepackage{graphicx}
\usepackage{dialogue}
\usepackage{mathtools}
\usepackage{hyperref}
%\hypersetup{draft}

\usepackage{multirow}
\usepackage{mdframed}
\usepackage{tcolorbox}

\usepackage{xcolor,pifont}
%\newcommand{\cmark}{\ding{51}}
%\newcommand{\xmark}{\ding{55}}

\setcounter{topnumber}{2}
\setcounter{bottomnumber}{2}
\setcounter{totalnumber}{4}
\renewcommand{\topfraction}{0.75}
\renewcommand{\bottomfraction}{0.75}
\renewcommand{\textfraction}{0.05}
\renewcommand{\floatpagefraction}{0.6}

\newcommand\cmark {\textcolor{green}{\ding{52}}}
\newcommand\xmark {\textcolor{red}{\ding{55}}}
\mdfdefinestyle{dialogue}{
    backgroundcolor=yellow!20,
    innermargin=5pt
}
\usepackage{amssymb}
\usepackage{soul}
\makeatletter

\begin{document}

\hspace{15pt}{\textbf{Section}:Yuan dynasty5\\}
\\ Context: Shi Tianze was a Han Chinese who lived in the Jin dynasty. Interethnic marriage between Han and Jurchen became common at this time. His father was Shi Bingzhi (史秉直, Shih Ping-chih). Shi Bingzhi was married to a Jurchen woman (surname Na-ho) and a Han Chinese woman (surname Chang); it is unknown which of them was Shi Tianze's mother. Shi Tianze was married to two Jurchen women, a Han Chinese woman, and a Korean woman, and his son Shi Gang was born to one of his Jurchen wives. The surnames of his Jurchen wives were Mo-nien and Na-ho; the surname of his Korean wife was Li; and the surname of his Han Chinese wife was Shi. Shi Tianze defected to Mongol forces upon their invasion of the Jin dynasty. His son Shi Gang married a Kerait woman; the Kerait were Mongolified Turkic people and were considered part of the ''Mongol nation''. Shi Tianze (Shih T'ien-tse), Zhang Rou (Chang Jou, 張柔), and Yan Shi (Yen Shih, 嚴實) and other high ranking Chinese who served in the Jin dynasty and defected to the Mongols helped build the structure for the administration of the new state. Chagaan (Tsagaan) and Zhang Rou jointly launched an attack on the Song dynasty ordered by Töregene Khatun. CANNOTANSWER

\begin{figure}[t] \small \begin{tcolorbox}[boxsep=0pt,left=5pt,right=0pt,top=2pt,colback = yellow!5] \begin{dialogue}
 \small 
 \speak{Student}{\bf What ethnicity was Shi Tianze? }
\speak{Teacher}\colorbox{pink!25}{$\hookrightarrow$}
{ Han Chinese }
\\
\speak{Student}{\bf In what dynasty did Tianze live? }
\speak{Teacher}\colorbox{pink!25}{$\hookrightarrow$}
{ Jin dynasty }
\\
\speak{Student}{\bf What kind of interethnic marriage became common in the Jin dynasty? }
\speak{Teacher}\colorbox{pink!25}{$\hookrightarrow$}
{ between Han and Jurchen }
\\
\speak{Student}{\bf Who was Shi Tianze's father? }
\speak{Teacher}\colorbox{pink!25}{$\hookrightarrow$}
{ Shi Bingzhi }
\\
\speak{Student}{\bf What dynasty did Zhang Rhou help attack? }
\speak{Teacher}\colorbox{pink!25}{$\hookrightarrow$}
{ Song dynasty }
\\
\speak{Student}{\bf What religion was Shi Tianze? }
\speak{Teacher}\colorbox{pink!25}{$\hookrightarrow$}
{ CANNOTANSWER }
\\
\speak{Student}{\bf  In what dynasty did Tianze die? }
\speak{Teacher}\colorbox{pink!25}{$\hookrightarrow$}
{ CANNOTANSWER }
\\
\speak{Student}{\bf What kind of interethnic marriage became uncommon in the Jin dynasty? }
\speak{Teacher}\colorbox{pink!25}{$\hookrightarrow$}
{ CANNOTANSWER }
 \end{dialogue}\end{tcolorbox}\end{figure}\begin{figure}[t] \small \begin{tcolorbox}[boxsep=0pt,left=5pt,right=0pt,top=2pt,colback = yellow!5] \begin{dialogue}
 \small 
 \speak{Student}{\bf  Who was Shi Tianze's uncle? }
\speak{Teacher}\colorbox{pink!25}{$\hookrightarrow$}
{ CANNOTANSWER }
\\
\speak{Student}{\bf  What dynasty did Zhang Rhou help defend? }
\speak{Teacher}\colorbox{pink!25}{$\hookrightarrow$}
{ CANNOTANSWER }
\\
 \end{dialogue}\end{tcolorbox}\end{figure}

\end{document}

