\documentclass[11pt,a4paper, onecolumn]{article}
\usepackage{times}
\usepackage{latexsym}
\usepackage{url}
\usepackage{textcomp}
\usepackage{bbm}
\usepackage{amsmath}
\usepackage{booktabs}
\usepackage{tabularx}
\usepackage{graphicx}
\usepackage{dialogue}
\usepackage{mathtools}
\usepackage{hyperref}
%\hypersetup{draft}

\usepackage{multirow}
\usepackage{mdframed}
\usepackage{tcolorbox}

\usepackage{xcolor,pifont}
%\newcommand{\cmark}{\ding{51}}
%\newcommand{\xmark}{\ding{55}}

\setcounter{topnumber}{2}
\setcounter{bottomnumber}{2}
\setcounter{totalnumber}{4}
\renewcommand{\topfraction}{0.75}
\renewcommand{\bottomfraction}{0.75}
\renewcommand{\textfraction}{0.05}
\renewcommand{\floatpagefraction}{0.6}

\newcommand\cmark {\textcolor{green}{\ding{52}}}
\newcommand\xmark {\textcolor{red}{\ding{55}}}
\mdfdefinestyle{dialogue}{
    backgroundcolor=yellow!20,
    innermargin=5pt
}
\usepackage{amssymb}
\usepackage{soul}
\makeatletter

\begin{document}

\hspace{15pt}{\textbf{Section}:Jacksonville, Florida5\\}
\\ Context: Spain ceded Florida to the British in 1763 after the French and Indian War, and the British soon constructed the King's Road connecting St. Augustine to Georgia. The road crossed the St. Johns River at a narrow point, which the Seminole called Wacca Pilatka and the British called the Cow Ford or Cowford; these names ostensibly reflect the fact that cattle were brought across the river there. The British introduced the cultivation of sugar cane, indigo and fruits as well the export of lumber. As a result, the northeastern Florida area prospered economically more than it had under the Spanish. Britain ceded control of the territory back to Spain in 1783, after its defeat in the American Revolutionary War, and the settlement at the Cow Ford continued to grow. After Spain ceded the Florida Territory to the United States in 1821, American settlers on the north side of the Cow Ford decided to plan a town, laying out the streets and plats. They soon named the town Jacksonville, after Andrew Jackson. Led by Isaiah D. Hart, residents wrote a charter for a town government, which was approved by the Florida Legislative Council on February 9, 1832. CANNOTANSWER

\begin{figure}[t] \small \begin{tcolorbox}[boxsep=0pt,left=5pt,right=0pt,top=2pt,colback = yellow!5] \begin{dialogue}
 \small 
 \speak{Student}{\bf After what event did the Spanish concede Florida to Britain? }
\speak{Teacher}\colorbox{pink!25}{$\hookrightarrow$}
{ French and Indian War }
\\
\speak{Student}{\bf Soon after gaining Florida, what did the English do? }
\speak{Teacher}\colorbox{pink!25}{$\hookrightarrow$}
{ constructed the King's Road }
\\
\speak{Student}{\bf Why the narrow part of St. John's River called Cowford? }
\speak{Teacher}\colorbox{pink!25}{$\hookrightarrow$}
{ cattle were brought across the river there. }
\\
\speak{Student}{\bf Who gained control of Florida after the conclusion of the Revolutionary War? }
\speak{Teacher}\colorbox{pink!25}{$\hookrightarrow$}
{ Spain }
\\
\speak{Student}{\bf When was the Jacksonville town charter approved? }
\speak{Teacher}\colorbox{pink!25}{$\hookrightarrow$}
{ February 9, 1832 }
\\
\speak{Student}{\bf When did the French construct Kings Road? }
\speak{Teacher}\colorbox{pink!25}{$\hookrightarrow$}
{ CANNOTANSWER }
\\
\speak{Student}{\bf What do French construct shortly before losing Florida? }
\speak{Teacher}\colorbox{pink!25}{$\hookrightarrow$}
{ CANNOTANSWER }
\\
\speak{Student}{\bf What town charter was approved in 1821? }
\speak{Teacher}\colorbox{pink!25}{$\hookrightarrow$}
{ CANNOTANSWER }
 \end{dialogue}\end{tcolorbox}\end{figure}

\end{document}

