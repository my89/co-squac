\documentclass[11pt,a4paper, onecolumn]{article}
\usepackage{times}
\usepackage{latexsym}
\usepackage{url}
\usepackage{textcomp}
\usepackage{bbm}
\usepackage{amsmath}
\usepackage{booktabs}
\usepackage{tabularx}
\usepackage{graphicx}
\usepackage{dialogue}
\usepackage{mathtools}
\usepackage{hyperref}
%\hypersetup{draft}

\usepackage{multirow}
\usepackage{mdframed}
\usepackage{tcolorbox}

\usepackage{xcolor,pifont}
%\newcommand{\cmark}{\ding{51}}
%\newcommand{\xmark}{\ding{55}}

\setcounter{topnumber}{2}
\setcounter{bottomnumber}{2}
\setcounter{totalnumber}{4}
\renewcommand{\topfraction}{0.75}
\renewcommand{\bottomfraction}{0.75}
\renewcommand{\textfraction}{0.05}
\renewcommand{\floatpagefraction}{0.6}

\newcommand\cmark {\textcolor{green}{\ding{52}}}
\newcommand\xmark {\textcolor{red}{\ding{55}}}
\mdfdefinestyle{dialogue}{
    backgroundcolor=yellow!20,
    innermargin=5pt
}
\usepackage{amssymb}
\usepackage{soul}
\makeatletter

\begin{document}

\hspace{15pt}{\textbf{Section}:European Union law17\\}
\\ Context: Fourth, national courts have a duty to interpret domestic law ''as far as possible in the light of the wording and purpose of the directive''. Textbooks (though not the Court itself) often called this ''indirect effect''. In Marleasing SA v La Comercial SA the Court of Justice held that a Spanish Court had to interpret its general Civil Code provisions, on contracts lacking cause or defrauding creditors, to conform with the First Company Law Directive article 11, that required incorporations would only be nullified for a fixed list of reasons. The Court of Justice quickly acknowledged that the duty of interpretation cannot contradict plain words in a national statute. But, fifth, if a member state has failed to implement a Directive, a citizen may not be able to bring claims against other non-state parties, but can sue the member state itself for failure to implement the law. So, in Francovich v Italy, the Italian government had failed to set up an insurance fund for employees to claim unpaid wages if their employers had gone insolvent, as the Insolvency Protection Directive required. Francovich, the former employee of a bankrupt Venetian firm, was therefore allowed to claim 6 million Lira from the Italian government in damages for his loss. The Court of Justice held that if a Directive would confer identifiable rights on individuals, and there is a causal link between a member state's violation of EU and a claimant's loss, damages must be paid. The fact that the incompatible law is an Act of Parliament is no defence. CANNOTANSWER

\begin{figure}[t] \small \begin{tcolorbox}[boxsep=0pt,left=5pt,right=0pt,top=2pt,colback = yellow!5] \begin{dialogue}
 \small 
 \speak{Student}{\bf Which courts have a duty to interpret domestic law as far as possible? }
\speak{Teacher}\colorbox{pink!25}{$\hookrightarrow$}
{ national courts }
\\
\speak{Student}{\bf What does the First Company Law Directive article 11 require? }
\speak{Teacher}\colorbox{pink!25}{$\hookrightarrow$}
{ incorporations would only be nullified for a fixed list of reasons }
\\
\speak{Student}{\bf What did the Italian government fail to do in Francovich v Italy? }
\speak{Teacher}\colorbox{pink!25}{$\hookrightarrow$}
{ failed to set up an insurance fund for employees to claim unpaid wages if their employers had gone insolvent }
\\
\speak{Student}{\bf How much money was Francovich allowed to claim from the Italian goverment in damages? }
\speak{Teacher}\colorbox{pink!25}{$\hookrightarrow$}
{ 6 million Lira }
\\
\speak{Student}{\bf Which courts do not have a duty to interpret  domestic law as far a possible? }
\speak{Teacher}\colorbox{pink!25}{$\hookrightarrow$}
{ CANNOTANSWER }
\\
\speak{Student}{\bf What does the First Company Law Directive article 11 not require? }
\speak{Teacher}\colorbox{pink!25}{$\hookrightarrow$}
{ CANNOTANSWER }
\\
\speak{Student}{\bf What did the Court of Justice not acknowledge? }
\speak{Teacher}\colorbox{pink!25}{$\hookrightarrow$}
{ CANNOTANSWER }
\\
\speak{Student}{\bf What did the Italian government not fail to do in Francovich v Italy? }
\speak{Teacher}\colorbox{pink!25}{$\hookrightarrow$}
{ CANNOTANSWER }
 \end{dialogue}\end{tcolorbox}\end{figure}\begin{figure}[t] \small \begin{tcolorbox}[boxsep=0pt,left=5pt,right=0pt,top=2pt,colback = yellow!5] \begin{dialogue}
 \small 
 \speak{Student}{\bf How much money did Francovich not allowed to claim from the Italian government in claims? }
\speak{Teacher}\colorbox{pink!25}{$\hookrightarrow$}
{ CANNOTANSWER }
\\
 \end{dialogue}\end{tcolorbox}\end{figure}

\end{document}

