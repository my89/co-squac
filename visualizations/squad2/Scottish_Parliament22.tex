\documentclass[11pt,a4paper, onecolumn]{article}
\usepackage{times}
\usepackage{latexsym}
\usepackage{url}
\usepackage{textcomp}
\usepackage{bbm}
\usepackage{amsmath}
\usepackage{booktabs}
\usepackage{tabularx}
\usepackage{graphicx}
\usepackage{dialogue}
\usepackage{mathtools}
\usepackage{hyperref}
%\hypersetup{draft}

\usepackage{multirow}
\usepackage{mdframed}
\usepackage{tcolorbox}

\usepackage{xcolor,pifont}
%\newcommand{\cmark}{\ding{51}}
%\newcommand{\xmark}{\ding{55}}

\setcounter{topnumber}{2}
\setcounter{bottomnumber}{2}
\setcounter{totalnumber}{4}
\renewcommand{\topfraction}{0.75}
\renewcommand{\bottomfraction}{0.75}
\renewcommand{\textfraction}{0.05}
\renewcommand{\floatpagefraction}{0.6}

\newcommand\cmark {\textcolor{green}{\ding{52}}}
\newcommand\xmark {\textcolor{red}{\ding{55}}}
\mdfdefinestyle{dialogue}{
    backgroundcolor=yellow!20,
    innermargin=5pt
}
\usepackage{amssymb}
\usepackage{soul}
\makeatletter

\begin{document}

\hspace{15pt}{\textbf{Section}:Scottish Parliament22\\}
\\ Context: The Scotland Act 1998, which was passed by the Parliament of the United Kingdom and given royal assent by Queen Elizabeth II on 19 November 1998, governs the functions and role of the Scottish Parliament and delimits its legislative competence. The Scotland Act 2012 extends the devolved competencies. For the purposes of parliamentary sovereignty, the Parliament of the United Kingdom at Westminster continues to constitute the supreme legislature of Scotland. However, under the terms of the Scotland Act, Westminster agreed to devolve some of its responsibilities over Scottish domestic policy to the Scottish Parliament. Such ''devolved matters'' include education, health, agriculture and justice. The Scotland Act enabled the Scottish Parliament to pass primary legislation on these issues. A degree of domestic authority, and all foreign policy, remain with the UK Parliament in Westminster. The Scottish Parliament has the power to pass laws and has limited tax-varying capability. Another of the roles of the Parliament is to hold the Scottish Government to account. CANNOTANSWER

\begin{figure}[t] \small \begin{tcolorbox}[boxsep=0pt,left=5pt,right=0pt,top=2pt,colback = yellow!5] \begin{dialogue}
 \small 
 \speak{Student}{\bf What act sets forth the functions of the Scottish Parliament? }
\speak{Teacher}\colorbox{pink!25}{$\hookrightarrow$}
{ Scotland Act 1998 }
\\
\speak{Student}{\bf Who gave her royal assent to the Scotland Act of 1998? }
\speak{Teacher}\colorbox{pink!25}{$\hookrightarrow$}
{ Queen Elizabeth II }
\\
\speak{Student}{\bf What does the Scotland Act of 2012 extend? }
\speak{Teacher}\colorbox{pink!25}{$\hookrightarrow$}
{ devolved competencies }
\\
\speak{Student}{\bf What body constitutes the supreme legislature of Scotland? }
\speak{Teacher}\colorbox{pink!25}{$\hookrightarrow$}
{ Parliament of the United Kingdom at Westminster }
\\
\speak{Student}{\bf Who has the role of holding the Scottish Government to account? }
\speak{Teacher}\colorbox{pink!25}{$\hookrightarrow$}
{ Scottish Parliament }
\\
\speak{Student}{\bf The Scotland Act 2002 extends the devolved what? }
\speak{Teacher}\colorbox{pink!25}{$\hookrightarrow$}
{ CANNOTANSWER }
\\
\speak{Student}{\bf Who passed the Scotland Act of 1988? }
\speak{Teacher}\colorbox{pink!25}{$\hookrightarrow$}
{ CANNOTANSWER }
\\
\speak{Student}{\bf Who granted royal assent to the Scotland Act of 1988? }
\speak{Teacher}\colorbox{pink!25}{$\hookrightarrow$}
{ CANNOTANSWER }
 \end{dialogue}\end{tcolorbox}\end{figure}\begin{figure}[t] \small \begin{tcolorbox}[boxsep=0pt,left=5pt,right=0pt,top=2pt,colback = yellow!5] \begin{dialogue}
 \small 
 \speak{Student}{\bf The Scotland Act enabled the Spanish Parliament to pass what? }
\speak{Teacher}\colorbox{pink!25}{$\hookrightarrow$}
{ CANNOTANSWER }
\\
\speak{Student}{\bf Who has no power to pass laws? }
\speak{Teacher}\colorbox{pink!25}{$\hookrightarrow$}
{ CANNOTANSWER }
\\
 \end{dialogue}\end{tcolorbox}\end{figure}

\end{document}

