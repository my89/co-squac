\documentclass[11pt,a4paper, onecolumn]{article}
\usepackage{times}
\usepackage{latexsym}
\usepackage{url}
\usepackage{textcomp}
\usepackage{bbm}
\usepackage{amsmath}
\usepackage{booktabs}
\usepackage{tabularx}
\usepackage{graphicx}
\usepackage{dialogue}
\usepackage{mathtools}
\usepackage{hyperref}
%\hypersetup{draft}

\usepackage{multirow}
\usepackage{mdframed}
\usepackage{tcolorbox}

\usepackage{xcolor,pifont}
%\newcommand{\cmark}{\ding{51}}
%\newcommand{\xmark}{\ding{55}}

\setcounter{topnumber}{2}
\setcounter{bottomnumber}{2}
\setcounter{totalnumber}{4}
\renewcommand{\topfraction}{0.75}
\renewcommand{\bottomfraction}{0.75}
\renewcommand{\textfraction}{0.05}
\renewcommand{\floatpagefraction}{0.6}

\newcommand\cmark {\textcolor{green}{\ding{52}}}
\newcommand\xmark {\textcolor{red}{\ding{55}}}
\mdfdefinestyle{dialogue}{
    backgroundcolor=yellow!20,
    innermargin=5pt
}
\usepackage{amssymb}
\usepackage{soul}
\makeatletter

\begin{document}

\hspace{15pt}{\textbf{Section}:Rhine1\\}
\\ Context: The variant forms of the name of the Rhine in modern languages are all derived from the Gaulish name Rēnos, which was adapted in Roman-era geography (1st century BC) as Greek Ῥῆνος (Rhēnos), Latin Rhenus.[note 3] The spelling with Rh- in English Rhine as well as in German Rhein and French Rhin is due to the influence of Greek orthography, while the vocalisation -i- is due to the Proto-Germanic adoption of the Gaulish name as *Rīnaz, via Old Frankish giving Old English Rín, Old High German Rīn, Dutch Rijn (formerly also spelled Rhijn)). The diphthong in modern German Rhein (also adopted in Romansh Rein, Rain) is a Central German development of the early modern period, the Alemannic name Rī(n) retaining the older vocalism,[note 4] as does Ripuarian Rhing, while Palatine has diphthongized Rhei, Rhoi. Spanish is with French in adopting the Germanic vocalism Rin-, while Italian, Occitan and Portuguese retain the Latin Ren-. CANNOTANSWER

\begin{figure}[t] \small \begin{tcolorbox}[boxsep=0pt,left=5pt,right=0pt,top=2pt,colback = yellow!5] \begin{dialogue}
 \small 
 \speak{Student}{\bf Where does the name Rhine derive from?  }
\speak{Teacher}\colorbox{pink!25}{$\hookrightarrow$}
{ Gaulish name Rēnos }
\\
\speak{Student}{\bf What is the French name for the Rhine?  }
\speak{Teacher}\colorbox{pink!25}{$\hookrightarrow$}
{ Rhin }
\\
\speak{Student}{\bf What is the Proto-Germanic adaptation of the name of the Rhine? }
\speak{Teacher}\colorbox{pink!25}{$\hookrightarrow$}
{ Rīnaz }
\\
\speak{Student}{\bf What century did the name of the Rhine come from? }
\speak{Teacher}\colorbox{pink!25}{$\hookrightarrow$}
{ 1st century BC }
\\
\speak{Student}{\bf What does the name The Rhine come from?  }
\speak{Teacher}\colorbox{pink!25}{$\hookrightarrow$}
{ Gaulish name Rēnos }
\\
\speak{Student}{\bf What is the Rhine called in French? }
\speak{Teacher}\colorbox{pink!25}{$\hookrightarrow$}
{ Rhin }
\\
\speak{Student}{\bf What is the Proto-Germanic adoption of the Gaulish name of the Rhine? }
\speak{Teacher}\colorbox{pink!25}{$\hookrightarrow$}
{ Rīnaz }
\\
\speak{Student}{\bf What is the Rhine called in Dutch? }
\speak{Teacher}\colorbox{pink!25}{$\hookrightarrow$}
{ Rijn }
 \end{dialogue}\end{tcolorbox}\end{figure}\begin{figure}[t] \small \begin{tcolorbox}[boxsep=0pt,left=5pt,right=0pt,top=2pt,colback = yellow!5] \begin{dialogue}
 \small 
 \speak{Student}{\bf How was the Dutch name for the Rhine originally spelled?  }
\speak{Teacher}\colorbox{pink!25}{$\hookrightarrow$}
{ Rhijn }
\\
\speak{Student}{\bf Where are the variant forms of the name of the Rhine in ancient languages derived from? }
\speak{Teacher}\colorbox{pink!25}{$\hookrightarrow$}
{ CANNOTANSWER }
\\
\speak{Student}{\bf What century did the Germanic vocalism Rin come from? }
\speak{Teacher}\colorbox{pink!25}{$\hookrightarrow$}
{ CANNOTANSWER }
\\
\speak{Student}{\bf What century did the spelling with Rh- in English Rhine come from? }
\speak{Teacher}\colorbox{pink!25}{$\hookrightarrow$}
{ CANNOTANSWER }
\\
\speak{Student}{\bf What century did the French adopt the Germanic vocalism Rin-? }
\speak{Teacher}\colorbox{pink!25}{$\hookrightarrow$}
{ CANNOTANSWER }
\\
 \end{dialogue}\end{tcolorbox}\end{figure}

\end{document}

