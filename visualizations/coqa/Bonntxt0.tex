\documentclass[11pt,a4paper, onecolumn]{article}
\usepackage{times}
\usepackage{latexsym}
\usepackage{url}
\usepackage{textcomp}
\usepackage{bbm}
\usepackage{amsmath}
\usepackage{booktabs}
\usepackage{tabularx}
\usepackage{graphicx}
\usepackage{dialogue}
\usepackage{mathtools}
\usepackage{hyperref}
%\hypersetup{draft}

\usepackage{multirow}
\usepackage{mdframed}
\usepackage{tcolorbox}

\usepackage{xcolor,pifont}
%\newcommand{\cmark}{\ding{51}}
%\newcommand{\xmark}{\ding{55}}

\setcounter{topnumber}{2}
\setcounter{bottomnumber}{2}
\setcounter{totalnumber}{4}
\renewcommand{\topfraction}{0.75}
\renewcommand{\bottomfraction}{0.75}
\renewcommand{\textfraction}{0.05}
\renewcommand{\floatpagefraction}{0.6}

\newcommand\cmark {\textcolor{green}{\ding{52}}}
\newcommand\xmark {\textcolor{red}{\ding{55}}}
\mdfdefinestyle{dialogue}{
    backgroundcolor=yellow!20,
    innermargin=5pt
}
\usepackage{amssymb}
\usepackage{soul}
\makeatletter

\begin{document}

\hspace{15pt}{\textbf{Section}:Bonn.txt0\\}
\\ Context: The Federal City of Bonn () is a city on the banks of the Rhine in the German state of North Rhine-Westphalia, with a population of over 300,000. About south-southeast of Cologne, Bonn is in the southernmost part of the Rhine-Ruhr region, Germany's largest metropolitan area, with over 11 million inhabitants. Together with the capital Berlin, the city is the ''de facto'' joint seat of government of Germany. Bonn is the secondary seat of the President, the Chancellor, the Bundesrat and the primary seat of six federal government ministries and twenty federal authorities. The title of Federal City () reflects its important political status within Germany. Founded in the 1st century BC as a Roman settlement, Bonn is one of Germany's oldest cities. From 1597 to 1794, Bonn was the capital of the Electorate of Cologne, and residence of the Archbishops and Prince-electors of Cologne. Composer Ludwig van Beethoven was born in Bonn in 1770. From 1949 to 1990, Bonn was the provisional capital (''temporary seat of the Federal institutions'') of West Germany, and Germany's present constitution, the Basic Law, was declared in the city in 1949. From 1990 to 1999, Bonn served as the seat of government – but no longer capital – of reunited Germany. CANNOTANSWER

\begin{figure}[t] \small \begin{tcolorbox}[boxsep=0pt,left=5pt,right=0pt,top=2pt,colback = yellow!5] \begin{dialogue}
 \small 
 \speak{Student}{\bf What has a population of over 300,000 }
\speak{Teacher}\colorbox{pink!25}{$\hookrightarrow$}
{ ``The Federal City of Bonn'' (Federal City of Bonn ) }
\\
\speak{Student}{\bf Where is it located? }
\speak{Teacher}\colorbox{pink!25}{$\hookrightarrow$}
{ ``on the banks of the Rhine'' (on the banks of the Rhine ) }
\\
\speak{Student}{\bf In what land zone? }
\speak{Teacher}\colorbox{pink!25}{$\hookrightarrow$}
{ ``the German state of North Rhine-Westphalia'' (German state of North Rhine-Westphalia ) }
\\
\speak{Student}{\bf Is it in the northernmost zone? }
\speak{Teacher}\colorbox{pink!25}{$\hookrightarrow$}
\colorbox{red!25}{No,}
{ ``no'' (German state of North Rhine-Westphalia ) }
\\
\speak{Student}{\bf What is it, then? }
\speak{Teacher}\colorbox{pink!25}{$\hookrightarrow$}
{ ``southernmost'' (southernmost ) }
\\
\speak{Student}{\bf Does it house 10 million people? }
\speak{Teacher}\colorbox{pink!25}{$\hookrightarrow$}
\colorbox{red!25}{No,}
{ ``no'' (southernmost ) }
\\
\speak{Student}{\bf How many? }
\speak{Teacher}\colorbox{pink!25}{$\hookrightarrow$}
{ ``over 11 million'' (over 11 million ) }
\\
\speak{Student}{\bf What is its most important city? }
\speak{Teacher}\colorbox{pink!25}{$\hookrightarrow$}
{ ``Bonn is a city. Do you mean in Germany?'' (Bonn () is a city ) }
 \end{dialogue}\end{tcolorbox}\end{figure}\begin{figure}[t] \small \begin{tcolorbox}[boxsep=0pt,left=5pt,right=0pt,top=2pt,colback = yellow!5] \begin{dialogue}
 \small 
 \speak{Student}{\bf Is it involved in governing? }
\speak{Teacher}\colorbox{pink!25}{$\hookrightarrow$}
\colorbox{red!25}{Yes,}
{ ``yes'' (Bonn () is a city ) }
\\
\speak{Student}{\bf When was it originally created? }
\speak{Teacher}\colorbox{pink!25}{$\hookrightarrow$}
{ ``in the 1st century'' (in the 1st century ) }
\\
\speak{Student}{\bf As what? }
\speak{Teacher}\colorbox{pink!25}{$\hookrightarrow$}
{ ``as a Roman settlement'' (as a Roman ) }
\\
\speak{Student}{\bf Is it a newer town? }
\speak{Teacher}\colorbox{pink!25}{$\hookrightarrow$}
\colorbox{red!25}{No,}
{ ``no'' (as a Roman ) }
\\
\speak{Student}{\bf Was someone famous birthed there? }
\speak{Teacher}\colorbox{pink!25}{$\hookrightarrow$}
\colorbox{red!25}{Yes,}
{ ``Yes'' (as a Roman ) }
\\
\speak{Student}{\bf Who? }
\speak{Teacher}\colorbox{pink!25}{$\hookrightarrow$}
{ ``Ludwig van Beethoven'' (Ludwig van Beethoven ) }
\\
\speak{Student}{\bf And when? }
\speak{Teacher}\colorbox{pink!25}{$\hookrightarrow$}
{ ``in 1770'' (in 1770 ) }
\\
\speak{Student}{\bf What happened in 1949? }
\speak{Teacher}\colorbox{pink!25}{$\hookrightarrow$}
{ ``the Basic Law, was declared'' (Basic Law, was declared ) }
 \end{dialogue}\end{tcolorbox}\end{figure}\begin{figure}[t] \small \begin{tcolorbox}[boxsep=0pt,left=5pt,right=0pt,top=2pt,colback = yellow!5] \begin{dialogue}
 \small 
 \speak{Student}{\bf And from 1990 to 1999? }
\speak{Teacher}\colorbox{pink!25}{$\hookrightarrow$}
{ ``Bonn served as the seat of government, but not the capitol'' (Bonn served as the seat of government ) }
\\
\speak{Student}{\bf What governing bodies is it the primary house of? }
\speak{Teacher}\colorbox{pink!25}{$\hookrightarrow$}
{ ``primary seat of six federal government ministries'' (primary seat of six federal government ministries ) }
\\
\speak{Student}{\bf Does "Federal City" mean anything? }
\speak{Teacher}\colorbox{pink!25}{$\hookrightarrow$}
{ ``reflects its important political status within Germany'' (reflects its important political status within Germany ) }
\\
\speak{Student}{\bf Is a medium-sized metro? }
\speak{Teacher}\colorbox{pink!25}{$\hookrightarrow$}
\colorbox{red!25}{No,}
{ ``no'' (reflects its important political status within Germany ) }
\\
 \end{dialogue}\end{tcolorbox}\end{figure}

\end{document}

