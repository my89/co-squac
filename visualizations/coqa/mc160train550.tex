\documentclass[11pt,a4paper, onecolumn]{article}
\usepackage{times}
\usepackage{latexsym}
\usepackage{url}
\usepackage{textcomp}
\usepackage{bbm}
\usepackage{amsmath}
\usepackage{booktabs}
\usepackage{tabularx}
\usepackage{graphicx}
\usepackage{dialogue}
\usepackage{mathtools}
\usepackage{hyperref}
%\hypersetup{draft}

\usepackage{multirow}
\usepackage{mdframed}
\usepackage{tcolorbox}

\usepackage{xcolor,pifont}
%\newcommand{\cmark}{\ding{51}}
%\newcommand{\xmark}{\ding{55}}

\setcounter{topnumber}{2}
\setcounter{bottomnumber}{2}
\setcounter{totalnumber}{4}
\renewcommand{\topfraction}{0.75}
\renewcommand{\bottomfraction}{0.75}
\renewcommand{\textfraction}{0.05}
\renewcommand{\floatpagefraction}{0.6}

\newcommand\cmark {\textcolor{green}{\ding{52}}}
\newcommand\xmark {\textcolor{red}{\ding{55}}}
\mdfdefinestyle{dialogue}{
    backgroundcolor=yellow!20,
    innermargin=5pt
}
\usepackage{amssymb}
\usepackage{soul}
\makeatletter

\begin{document}

\hspace{15pt}{\textbf{Section}:mc160.train.550\\}
\\ Context: On a snowy winter morning, the brown-haired lady saw a squirrel that was hurt. It only had three legs, and it looked hungry. She put some corn out for the squirrel to eat, but other bully squirrels came, too. The brown-haired lady started giving the little squirrel peanuts to eat. She gave some to the bully squirrels, too, so they would leave the three-legged squirrel alone. The winter snow melted and then it was spring. The grass turned green and the air was warm. Now, when the little squirrel with three legs would come to see the brown-haired lady with the peanuts, it would take the peanuts and dig a little hole and hide the peanuts for later. The squirrel would hold the peanut in its mouth and dig and dig and dig, and then it would put the peanut in the hole and pat it down with its little front paw. Then it would run back over to the brown-haired lady and get some more peanuts to eat. CANNOTANSWER

\begin{figure}[t] \small \begin{tcolorbox}[boxsep=0pt,left=5pt,right=0pt,top=2pt,colback = yellow!5] \begin{dialogue}
 \small 
 \speak{Student}{\bf Who seen a squirrel? }
\speak{Teacher}\colorbox{pink!25}{$\hookrightarrow$}
{ ``the lady'' (lady ) }
\\
\speak{Student}{\bf Was it snowing? }
\speak{Teacher}\colorbox{pink!25}{$\hookrightarrow$}
\colorbox{red!25}{Yes,}
{ ``yes'' (lady ) }
\\
\speak{Student}{\bf Did the squirrel have legs? }
\speak{Teacher}\colorbox{pink!25}{$\hookrightarrow$}
\colorbox{red!25}{Yes,}
{ ``yes'' (lady ) }
\\
\speak{Student}{\bf How many? }
\speak{Teacher}\colorbox{pink!25}{$\hookrightarrow$}
{ ``three'' (three ) }
\\
\speak{Student}{\bf Was he hungry? }
\speak{Teacher}\colorbox{pink!25}{$\hookrightarrow$}
\colorbox{red!25}{Yes,}
{ ``yes'' (three ) }
\\
\speak{Student}{\bf what did the lady put out for the squirrel? }
\speak{Teacher}\colorbox{pink!25}{$\hookrightarrow$}
{ ``corn'' (corn ) }
\\
\speak{Student}{\bf What color hair did the lady have? }
\speak{Teacher}\colorbox{pink!25}{$\hookrightarrow$}
{ ``brown'' (lady ) }
\\
\speak{Student}{\bf Did She feed other squirrels? }
\speak{Teacher}\colorbox{pink!25}{$\hookrightarrow$}
\colorbox{red!25}{Yes,}
{ ``yes'' (lady ) }
 \end{dialogue}\end{tcolorbox}\end{figure}\begin{figure}[t] \small \begin{tcolorbox}[boxsep=0pt,left=5pt,right=0pt,top=2pt,colback = yellow!5] \begin{dialogue}
 \small 
 \speak{Student}{\bf What did she feed them? }
\speak{Teacher}\colorbox{pink!25}{$\hookrightarrow$}
{ ``peanuts'' (peanuts ) }
\\
\speak{Student}{\bf When did the snow melt? }
\speak{Teacher}\colorbox{pink!25}{$\hookrightarrow$}
{ ``spring'' (spring ) }
\\
\speak{Student}{\bf Was the air warm? }
\speak{Teacher}\colorbox{pink!25}{$\hookrightarrow$}
\colorbox{red!25}{Yes,}
{ ``yes'' (spring ) }
\\
\speak{Student}{\bf What would the squirrel do with the peanuts? }
\speak{Teacher}\colorbox{pink!25}{$\hookrightarrow$}
{ ``hide them'' (hide ) }
\\
\speak{Student}{\bf Where? }
\speak{Teacher}\colorbox{pink!25}{$\hookrightarrow$}
{ ``a hole'' (hole ) }
\\
\speak{Student}{\bf Was the grass green? }
\speak{Teacher}\colorbox{pink!25}{$\hookrightarrow$}
\colorbox{red!25}{Yes,}
{ ``yes'' (hole ) }
\\
\speak{Student}{\bf How did the squirrel hold the peanuts? }
\speak{Teacher}\colorbox{pink!25}{$\hookrightarrow$}
{ ``in its mouth'' (in its mouth ) }
\\
 \end{dialogue}\end{tcolorbox}\end{figure}

\end{document}

