\documentclass[11pt,a4paper, onecolumn]{article}
\usepackage{times}
\usepackage{latexsym}
\usepackage{url}
\usepackage{textcomp}
\usepackage{bbm}
\usepackage{amsmath}
\usepackage{booktabs}
\usepackage{tabularx}
\usepackage{graphicx}
\usepackage{dialogue}
\usepackage{mathtools}
\usepackage{hyperref}
%\hypersetup{draft}

\usepackage{multirow}
\usepackage{mdframed}
\usepackage{tcolorbox}

\usepackage{xcolor,pifont}
%\newcommand{\cmark}{\ding{51}}
%\newcommand{\xmark}{\ding{55}}

\setcounter{topnumber}{2}
\setcounter{bottomnumber}{2}
\setcounter{totalnumber}{4}
\renewcommand{\topfraction}{0.75}
\renewcommand{\bottomfraction}{0.75}
\renewcommand{\textfraction}{0.05}
\renewcommand{\floatpagefraction}{0.6}

\newcommand\cmark {\textcolor{green}{\ding{52}}}
\newcommand\xmark {\textcolor{red}{\ding{55}}}
\mdfdefinestyle{dialogue}{
    backgroundcolor=yellow!20,
    innermargin=5pt
}
\usepackage{amssymb}
\usepackage{soul}
\makeatletter

\begin{document}

\hspace{15pt}{\textbf{Section}:PolicePolice 29f96c2c016a593e1d09c827a3b482411f0c1088af9a86722c7498220\\}
\\ Context: A police force is a constituted body of persons empowered by the state to enforce the law, protect property, and limit civil disorder. Their powers include the legitimized use of force. The term is most commonly associated with police services of a sovereign state that are authorized to exercise the police power of that state within a defined legal or territorial area of responsibility. Police forces are often defined as being separate from military or other organizations involved in the defense of the state against foreign aggressors; however, gendarmerie are military units charged with civil policing. Law enforcement, however, constitutes only part of policing activity. Policing has included an array of activities in different situations, but the predominant ones are concerned with the preservation of order. In some societies, in the late 18th and early 19th centuries, these developed within the context of maintaining the class system and the protection of private property. Many police forces suffer from police corruption to a greater or lesser degree. The police force is usually a public sector service, meaning they are paid through taxes. CANNOTANSWER

\begin{figure}[t] \small \begin{tcolorbox}[boxsep=0pt,left=5pt,right=0pt,top=2pt,colback = yellow!5] \begin{dialogue}
 \small 
 \speak{Student}{\bf Is enforcing the law the entire goal of police? }
\speak{Teacher}\colorbox{pink!25}{$\hookrightarrow$}
\colorbox{red!25}{No,}
{ ``No'' (A ) }
\\
\speak{Student}{\bf What is their main activity concerned with? }
\speak{Teacher}\colorbox{pink!25}{$\hookrightarrow$}
{ ``preservation of order'' (preservation of order. ) }
\\
\speak{Student}{\bf In the 17-1800s, what was one other thing they were focused on? }
\speak{Teacher}\colorbox{pink!25}{$\hookrightarrow$}
{ ``maintaining the class system'' (maintaining the class system ) }
\\
\speak{Student}{\bf Anything else? }
\speak{Teacher}\colorbox{pink!25}{$\hookrightarrow$}
{ ``protection of private property'' (protection of private property. ) }
\\
\speak{Student}{\bf Is it true that there has been corruption in the police department? }
\speak{Teacher}\colorbox{pink!25}{$\hookrightarrow$}
\colorbox{red!25}{Yes,}
{ ``Yes'' (protection of private property. ) }
\\
\speak{Student}{\bf Just a few instances? }
\speak{Teacher}\colorbox{pink!25}{$\hookrightarrow$}
\colorbox{red!25}{No,}
{ ``No'' (protection of private property. ) }
\\
\speak{Student}{\bf Is the force paid for privately? }
\speak{Teacher}\colorbox{pink!25}{$\hookrightarrow$}
\colorbox{red!25}{No,}
{ ``No'' (protection of private property. ) }
\\
\speak{Student}{\bf How is it funded then? }
\speak{Teacher}\colorbox{pink!25}{$\hookrightarrow$}
{ ``taxes'' (taxes. ) }
 \end{dialogue}\end{tcolorbox}\end{figure}\begin{figure}[t] \small \begin{tcolorbox}[boxsep=0pt,left=5pt,right=0pt,top=2pt,colback = yellow!5] \begin{dialogue}
 \small 
 \speak{Student}{\bf What do they call an entity like that? }
\speak{Teacher}\colorbox{pink!25}{$\hookrightarrow$}
{ ``public sector service'' (public sector service, ) }
\\
\speak{Student}{\bf Are all police forces paid that way? }
\speak{Teacher}\colorbox{pink!25}{$\hookrightarrow$}
\colorbox{red!25}{No,}
{ ``No'' (public sector service, ) }
\\
\speak{Student}{\bf Who gives them their power? }
\speak{Teacher}\colorbox{pink!25}{$\hookrightarrow$}
{ ``the state'' (state ) }
\\
\speak{Student}{\bf How many main tasks are they asked to do? }
\speak{Teacher}\colorbox{pink!25}{$\hookrightarrow$}
{ ``three'' (a ) }
\\
\speak{Student}{\bf Do they go to war? }
\speak{Teacher}\colorbox{pink!25}{$\hookrightarrow$}
\colorbox{red!25}{No,}
{ ``unknown'' (CANNOTANSWER ) }
\\
\speak{Student}{\bf Do they protect people's personal property? }
\speak{Teacher}\colorbox{pink!25}{$\hookrightarrow$}
\colorbox{red!25}{Yes,}
{ ``Yes'' (CANNOTANSWER ) }
\\
 \end{dialogue}\end{tcolorbox}\end{figure}

\end{document}

