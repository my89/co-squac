\documentclass[11pt,a4paper, onecolumn]{article}
\usepackage{times}
\usepackage{latexsym}
\usepackage{url}
\usepackage{textcomp}
\usepackage{bbm}
\usepackage{amsmath}
\usepackage{booktabs}
\usepackage{tabularx}
\usepackage{graphicx}
\usepackage{dialogue}
\usepackage{mathtools}
\usepackage{hyperref}
%\hypersetup{draft}

\usepackage{multirow}
\usepackage{mdframed}
\usepackage{tcolorbox}

\usepackage{xcolor,pifont}
%\newcommand{\cmark}{\ding{51}}
%\newcommand{\xmark}{\ding{55}}

\setcounter{topnumber}{2}
\setcounter{bottomnumber}{2}
\setcounter{totalnumber}{4}
\renewcommand{\topfraction}{0.75}
\renewcommand{\bottomfraction}{0.75}
\renewcommand{\textfraction}{0.05}
\renewcommand{\floatpagefraction}{0.6}

\newcommand\cmark {\textcolor{green}{\ding{52}}}
\newcommand\xmark {\textcolor{red}{\ding{55}}}
\mdfdefinestyle{dialogue}{
    backgroundcolor=yellow!20,
    innermargin=5pt
}
\usepackage{amssymb}
\usepackage{soul}
\makeatletter

\begin{document}

\hspace{15pt}{\textbf{Section}:middle4330.txt0\\}
\\ Context: Diwali is perhaps the most well--known of the Hindu festivals. It is celebrated by Hindus in India and abroad. It is often called the Festival of Lights. For Hindus, Diwli is very important and it is also very exciting time for them. Normally , this holiday is celebrated in October or November and it falls on a different date each year. This year , it will be celebrated on October23. Diwali is usually celebrated for 5 days. To prepare for Diwali, Hindus spend several weeks cleaning their homes and preparing special food in order to welcome Laskhmi, the goodness of wealth into their lives. They will open the windows of their homes during this time to make sure that she can enter their home. One of other traditions of Dawali is to light up oil lamp in the homes. The oil lamps are used to make the goddess of wealth enter the homes. Hindus believe that she will not enter a home that is not lit up. During Diwali, the children in India do not have to go to school. They share gifts with one another and prepare special holiday meals to celebrate this event. Fireworks are also a big part of the Diwali festival. They are used to scare away bad spirits. CANNOTANSWER

\begin{figure}[t] \small \begin{tcolorbox}[boxsep=0pt,left=5pt,right=0pt,top=2pt,colback = yellow!5] \begin{dialogue}
 \small 
 \speak{Student}{\bf How do people get ready for Diwali? }
\speak{Teacher}\colorbox{pink!25}{$\hookrightarrow$}
{ ``Cleaning their homes and preparing special food'' (cleaning their homes and preparing special food ) }
\\
\speak{Student}{\bf What is Diwali? }
\speak{Teacher}\colorbox{pink!25}{$\hookrightarrow$}
{ ``the Festival of Lights'' (Festival of Lights ) }
\\
\speak{Student}{\bf Where is it celebrated? }
\speak{Teacher}\colorbox{pink!25}{$\hookrightarrow$}
{ ``in India'' (in India ) }
\\
\speak{Student}{\bf By who? }
\speak{Teacher}\colorbox{pink!25}{$\hookrightarrow$}
{ ``by Hindus'' (by Hindus ) }
\\
\speak{Student}{\bf For how long? }
\speak{Teacher}\colorbox{pink!25}{$\hookrightarrow$}
{ ``5 days'' (5 days ) }
\\
\speak{Student}{\bf How long do they take to get ready? }
\speak{Teacher}\colorbox{pink!25}{$\hookrightarrow$}
{ ``several weeks'' (several weeks ) }
\\
\speak{Student}{\bf Who are they expecting? }
\speak{Teacher}\colorbox{pink!25}{$\hookrightarrow$}
{ ``Laskhmi'' (Laskhmi ) }
\\
\speak{Student}{\bf Who is she? }
\speak{Teacher}\colorbox{pink!25}{$\hookrightarrow$}
{ ``the goddess of wealth'' (goddess of wealth ) }
 \end{dialogue}\end{tcolorbox}\end{figure}\begin{figure}[t] \small \begin{tcolorbox}[boxsep=0pt,left=5pt,right=0pt,top=2pt,colback = yellow!5] \begin{dialogue}
 \small 
 \speak{Student}{\bf How will she get in? }
\speak{Teacher}\colorbox{pink!25}{$\hookrightarrow$}
{ ``the windows'' (windows ) }
\\
\speak{Student}{\bf What happens if they house is dark? }
\speak{Teacher}\colorbox{pink!25}{$\hookrightarrow$}
{ ``she will not enter'' (she will not enter ) }
\\
\speak{Student}{\bf What do they use for light? }
\speak{Teacher}\colorbox{pink!25}{$\hookrightarrow$}
{ ``oil lamps'' (oil lamps ) }
\\
\speak{Student}{\bf Is the festival always the same day? }
\speak{Teacher}\colorbox{pink!25}{$\hookrightarrow$}
\colorbox{red!25}{No,}
{ ``No'' (oil lamps ) }
\\
\speak{Student}{\bf The same month? }
\speak{Teacher}\colorbox{pink!25}{$\hookrightarrow$}
\colorbox{red!25}{No,}
{ ``No'' (oil lamps ) }
\\
\speak{Student}{\bf What months does it usually take place? }
\speak{Teacher}\colorbox{pink!25}{$\hookrightarrow$}
{ ``in October or November'' (in October or November ) }
\\
\speak{Student}{\bf What day is it this year? }
\speak{Teacher}\colorbox{pink!25}{$\hookrightarrow$}
{ ``October23'' (October23 ) }
\\
\speak{Student}{\bf What is used to keep bad spirits away? }
\speak{Teacher}\colorbox{pink!25}{$\hookrightarrow$}
{ ``Fireworks'' (Fireworks ) }
 \end{dialogue}\end{tcolorbox}\end{figure}\begin{figure}[t] \small \begin{tcolorbox}[boxsep=0pt,left=5pt,right=0pt,top=2pt,colback = yellow!5] \begin{dialogue}
 \small 
 \speak{Student}{\bf Do the kids get time off of school? }
\speak{Teacher}\colorbox{pink!25}{$\hookrightarrow$}
\colorbox{red!25}{Yes,}
{ ``Yes'' (Fireworks ) }
\\
\speak{Student}{\bf What do they make for the event? }
\speak{Teacher}\colorbox{pink!25}{$\hookrightarrow$}
{ ``special holiday meals'' (special holiday meals ) }
\\
\speak{Student}{\bf Do they share anything? }
\speak{Teacher}\colorbox{pink!25}{$\hookrightarrow$}
{ ``gifts'' (gifts ) }
\\
\speak{Student}{\bf With who? }
\speak{Teacher}\colorbox{pink!25}{$\hookrightarrow$}
{ ``with one another'' (with one another ) }
\\
 \end{dialogue}\end{tcolorbox}\end{figure}

\end{document}

