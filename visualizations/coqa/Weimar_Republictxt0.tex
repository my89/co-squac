\documentclass[11pt,a4paper, onecolumn]{article}
\usepackage{times}
\usepackage{latexsym}
\usepackage{url}
\usepackage{textcomp}
\usepackage{bbm}
\usepackage{amsmath}
\usepackage{booktabs}
\usepackage{tabularx}
\usepackage{graphicx}
\usepackage{dialogue}
\usepackage{mathtools}
\usepackage{hyperref}
%\hypersetup{draft}

\usepackage{multirow}
\usepackage{mdframed}
\usepackage{tcolorbox}

\usepackage{xcolor,pifont}
%\newcommand{\cmark}{\ding{51}}
%\newcommand{\xmark}{\ding{55}}

\setcounter{topnumber}{2}
\setcounter{bottomnumber}{2}
\setcounter{totalnumber}{4}
\renewcommand{\topfraction}{0.75}
\renewcommand{\bottomfraction}{0.75}
\renewcommand{\textfraction}{0.05}
\renewcommand{\floatpagefraction}{0.6}

\newcommand\cmark {\textcolor{green}{\ding{52}}}
\newcommand\xmark {\textcolor{red}{\ding{55}}}
\mdfdefinestyle{dialogue}{
    backgroundcolor=yellow!20,
    innermargin=5pt
}
\usepackage{amssymb}
\usepackage{soul}
\makeatletter

\begin{document}

\hspace{15pt}{\textbf{Section}:Weimar Republic.txt0\\}
\\ Context: Weimar Republic was an unofficial, historical designation for the German state between 1919 and 1933. The name derives from the city of Weimar, where its constitutional assembly first took place. The official name of the state was ''Deutsches Reich''; it had remained unchanged since 1871. In English the country was usually known simply as Germany. A national assembly was convened in Weimar, where a new constitution for the ''Deutsches Reich'' was written, and adopted on 11 August 1919. In its fourteen years, the Weimar Republic faced numerous problems, including hyperinflation, political extremism (with paramilitaries – both left- and right-wing), as well as contentious relationships with the victors of the First World War. The people of Germany blamed the Weimar Republic rather than their wartime leaders for the country's defeat and for the humiliating terms of the Treaty of Versailles. Weimar Germany fulfilled most of the requirements of the Treaty of Versailles although it never completely met its disarmament requirements, and eventually paid only a small portion of the war reparations (by twice restructuring its debt through the Dawes Plan and the Young Plan). Under the Locarno Treaties, Germany accepted the western borders of the republic, but continued to dispute the Eastern border. CANNOTANSWER

\begin{figure}[t] \small \begin{tcolorbox}[boxsep=0pt,left=5pt,right=0pt,top=2pt,colback = yellow!5] \begin{dialogue}
 \small 
 \speak{Student}{\bf What was an unofficial designation }
\speak{Teacher}\colorbox{pink!25}{$\hookrightarrow$}
{ ``the Weimar Republic'' (Weimar Republic ) }
\\
\speak{Student}{\bf For what? }
\speak{Teacher}\colorbox{pink!25}{$\hookrightarrow$}
{ ``the German state'' (German state ) }
\\
\speak{Student}{\bf During what year? }
\speak{Teacher}\colorbox{pink!25}{$\hookrightarrow$}
{ ``between 1919 and 1933'' (between 1919 and 1933 ) }
\\
\speak{Student}{\bf Was Weimar a city? }
\speak{Teacher}\colorbox{pink!25}{$\hookrightarrow$}
\colorbox{red!25}{Yes,}
{ ``Yes'' (between 1919 and 1933 ) }
\\
\speak{Student}{\bf What took place there? }
\speak{Teacher}\colorbox{pink!25}{$\hookrightarrow$}
{ ``its constitutional assembly'' (its constitutional assembly ) }
\\
\speak{Student}{\bf What was the official name? }
\speak{Teacher}\colorbox{pink!25}{$\hookrightarrow$}
{ ``Deutsches Reich'' (''Deutsches Reich ) }
\\
\speak{Student}{\bf When did that begin? }
\speak{Teacher}\colorbox{pink!25}{$\hookrightarrow$}
{ ``1871'' (1871 ) }
\\
\speak{Student}{\bf What was adopted in 1919 }
\speak{Teacher}\colorbox{pink!25}{$\hookrightarrow$}
{ ``a new constitution'' (new constitution ) }
 \end{dialogue}\end{tcolorbox}\end{figure}\begin{figure}[t] \small \begin{tcolorbox}[boxsep=0pt,left=5pt,right=0pt,top=2pt,colback = yellow!5] \begin{dialogue}
 \small 
 \speak{Student}{\bf What month? }
\speak{Teacher}\colorbox{pink!25}{$\hookrightarrow$}
{ ``August'' (August ) }
\\
\speak{Student}{\bf What day? }
\speak{Teacher}\colorbox{pink!25}{$\hookrightarrow$}
{ ``the 11th'' (11 ) }
\\
\speak{Student}{\bf What happened in the 14th year }
\speak{Teacher}\colorbox{pink!25}{$\hookrightarrow$}
{ ``unknown'' (CANNOTANSWER ) }
\\
 \end{dialogue}\end{tcolorbox}\end{figure}

\end{document}

