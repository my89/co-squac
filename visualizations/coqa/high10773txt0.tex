\documentclass[11pt,a4paper, onecolumn]{article}
\usepackage{times}
\usepackage{latexsym}
\usepackage{url}
\usepackage{textcomp}
\usepackage{bbm}
\usepackage{amsmath}
\usepackage{booktabs}
\usepackage{tabularx}
\usepackage{graphicx}
\usepackage{dialogue}
\usepackage{mathtools}
\usepackage{hyperref}
%\hypersetup{draft}

\usepackage{multirow}
\usepackage{mdframed}
\usepackage{tcolorbox}

\usepackage{xcolor,pifont}
%\newcommand{\cmark}{\ding{51}}
%\newcommand{\xmark}{\ding{55}}

\setcounter{topnumber}{2}
\setcounter{bottomnumber}{2}
\setcounter{totalnumber}{4}
\renewcommand{\topfraction}{0.75}
\renewcommand{\bottomfraction}{0.75}
\renewcommand{\textfraction}{0.05}
\renewcommand{\floatpagefraction}{0.6}

\newcommand\cmark {\textcolor{green}{\ding{52}}}
\newcommand\xmark {\textcolor{red}{\ding{55}}}
\mdfdefinestyle{dialogue}{
    backgroundcolor=yellow!20,
    innermargin=5pt
}
\usepackage{amssymb}
\usepackage{soul}
\makeatletter

\begin{document}

\hspace{15pt}{\textbf{Section}:high10773.txt0\\}
\\ Context: A couple of weeks ago, my 12-year-old daughter, Ella threatened to take my phone and break it. ''At night you'll always have your phone out and break you'll just type,'' Ella says. ''I'm ready to go to bed, and try to get you to read stories for me and you're just standing there reading your texts and texting other people,'' she adds. I came to realize that I was ignoring her as a father. Ella isn't the only kid who feels this way about her parent's relationship with devices. Catherine Steiner-Adair, a psychologist at Harvard, wrote The Big Disconnect: Protecting Childhood and Family Relationships in the Digital Age. For her book, Steiner-Adair interviewed more than 1,000 kids from the ages of 4 to 18. She talked to hundreds of teachers and parents. One of the many things that knocked my socks off, '' she says, ''was the consistency with which children -- whether they were 4 or 8 or 18 or 24-- talked about feeling exhausted and frustrated or mad trying to get their parents' attention, competing with computer screens or iPhone screens or any kind of technology.'' A couple of years ago, my daughter got a laptop for school. And because she was becoming more independent, we got her a phone. We set up rules for when she could use the device and when she'd need to put it away. We created a charging station, outside her bedroom, where she had to plug in these devices every night. Basically -- except for homework-- she has to put it all away when she comes home. Steiner-Adair says most adults don't set up similar limits in their own lives. ''We've lost the boundaries that protect work and family life,'' she says. ''So it is very hard to manage yourself and be present in the moments your children need you.'' After my daughter's little intervention ,I made myself a promise to create my own charging station. To plug my phone in-- somewhere faraway -- when I am done working for the day. I've been trying to leave it there untouched for most of the weekend CANNOTANSWER

\begin{figure}[t] \small \begin{tcolorbox}[boxsep=0pt,left=5pt,right=0pt,top=2pt,colback = yellow!5] \begin{dialogue}
 \small 
 \speak{Student}{\bf Who threatened to take a phone? }
\speak{Teacher}\colorbox{pink!25}{$\hookrightarrow$}
{ ``Ella.'' (Ella ) }
\\
\speak{Student}{\bf What age is she? }
\speak{Teacher}\colorbox{pink!25}{$\hookrightarrow$}
{ ``12.'' (Ella ) }
\\
\speak{Student}{\bf Who works at a college? }
\speak{Teacher}\colorbox{pink!25}{$\hookrightarrow$}
{ ``Catherine Steiner-Adair.'' (Catherine Steiner-Adair, ) }
\\
\speak{Student}{\bf What did her child receive for school? }
\speak{Teacher}\colorbox{pink!25}{$\hookrightarrow$}
{ ``A laptop.'' (laptop ) }
\\
\speak{Student}{\bf What else? }
\speak{Teacher}\colorbox{pink!25}{$\hookrightarrow$}
{ ``A phone.'' (phone. ) }
\\
\speak{Student}{\bf What did they make outside of her room? }
\speak{Teacher}\colorbox{pink!25}{$\hookrightarrow$}
{ ``A charging station.'' (charging station, ) }
\\
\speak{Student}{\bf She mentions a lot of grown ups don't make what in their lifetime? }
\speak{Teacher}\colorbox{pink!25}{$\hookrightarrow$}
{ ``Limits.'' (limits ) }
\\
\speak{Student}{\bf What does she vow to not touch during a day? }
\speak{Teacher}\colorbox{pink!25}{$\hookrightarrow$}
{ ``Phone.'' (phone ) }
 \end{dialogue}\end{tcolorbox}\end{figure}\begin{figure}[t] \small \begin{tcolorbox}[boxsep=0pt,left=5pt,right=0pt,top=2pt,colback = yellow!5] \begin{dialogue}
 \small 
 \speak{Student}{\bf What does she say her child CAN be active on her electronics? }
\speak{Teacher}\colorbox{pink!25}{$\hookrightarrow$}
{ ``Homework.'' (homework-- ) }
\\
\speak{Student}{\bf Besides weekdays - when else does she vow to not touch it very often? }
\speak{Teacher}\colorbox{pink!25}{$\hookrightarrow$}
{ ``Most of the weekend.'' (most of the weekend ) }
\\
 \end{dialogue}\end{tcolorbox}\end{figure}

\end{document}

