\documentclass[11pt,a4paper, onecolumn]{article}
\usepackage{times}
\usepackage{latexsym}
\usepackage{url}
\usepackage{textcomp}
\usepackage{bbm}
\usepackage{amsmath}
\usepackage{booktabs}
\usepackage{tabularx}
\usepackage{graphicx}
\usepackage{dialogue}
\usepackage{mathtools}
\usepackage{hyperref}
%\hypersetup{draft}

\usepackage{multirow}
\usepackage{mdframed}
\usepackage{tcolorbox}

\usepackage{xcolor,pifont}
%\newcommand{\cmark}{\ding{51}}
%\newcommand{\xmark}{\ding{55}}

\setcounter{topnumber}{2}
\setcounter{bottomnumber}{2}
\setcounter{totalnumber}{4}
\renewcommand{\topfraction}{0.75}
\renewcommand{\bottomfraction}{0.75}
\renewcommand{\textfraction}{0.05}
\renewcommand{\floatpagefraction}{0.6}

\newcommand\cmark {\textcolor{green}{\ding{52}}}
\newcommand\xmark {\textcolor{red}{\ding{55}}}
\mdfdefinestyle{dialogue}{
    backgroundcolor=yellow!20,
    innermargin=5pt
}
\usepackage{amssymb}
\usepackage{soul}
\makeatletter

\begin{document}

\hspace{15pt}{\textbf{Section}:Soviet Union.txt0\\}
\\ Context: The Soviet Union, officially the Union of Soviet Socialist Republics (USSR, Russian: СССР), was a socialist state in Eurasia that existed from 1922 to 1991. Nominally a union of multiple equal national Soviet republics, its government and economy were highly centralized. The country was a one-party state, governed by the Communist Party with Moscow as its capital in its largest republic, the Russian Soviet Federative Socialist Republic. The Russian nation had constitutionally equal status among the many nations of the union but exerted de facto dominance in various respects. Other major urban centres were Leningrad, Kiev, Minsk, Alma-Ata and Novosibirsk. The Soviet Union was one of the five recognized nuclear weapons states and possessed the largest stockpile of weapons of mass destruction. It was a founding permanent member of the United Nations Security Council, as well as a member of the Organization for Security and Co-operation in Europe (OSCE), and the leading member of the Council for Mutual Economic Assistance (CMEA) and the Warsaw Pact. The Soviet Union had its roots in the October Revolution of 1917, when the Bolsheviks, led by Vladimir Lenin, overthrew the Russian Provisional Government which had replaced Tsar Nicholas II during World War I. In 1922, the Soviet Union was formed with the unification of the Russian, Transcaucasian, Ukrainian, and Byelorussian republics. Following Lenin's death in 1924 and a brief power struggle, Joseph Stalin came to power in the mid-1920s. Stalin committed the state's ideology to Marxism–Leninism (which he created), and initiated a centrally planned economy which led to a period of rapid industrialization and collectivization. During this period of totalitarian rule, Stalin imposed political paranoia; the mid-1930s Great Purge removed his opponents within and outside of the party via arbitrary arrests and persecutions of many people. Suppression of political critics, forced labor, and famines were perpetrated by Stalin; in 1933, a major famine struck Soviet Ukraine, causing the deaths of over 7 million people. CANNOTANSWER

\begin{figure}[t] \small \begin{tcolorbox}[boxsep=0pt,left=5pt,right=0pt,top=2pt,colback = yellow!5] \begin{dialogue}
 \small 
 \speak{Student}{\bf Who led the Bolsheviks? }
\speak{Teacher}\colorbox{pink!25}{$\hookrightarrow$}
{ ``Vladimir Lenin,'' (Vladimir Lenin, ) }
\\
\speak{Student}{\bf What did they overthrow? }
\speak{Teacher}\colorbox{pink!25}{$\hookrightarrow$}
{ ``Russian Provisional Government'' (Russian Provisional Government ) }
\\
\speak{Student}{\bf Which had replaced the rule of whom? }
\speak{Teacher}\colorbox{pink!25}{$\hookrightarrow$}
{ ``Tsar Nicholas II'' (Tsar Nicholas II ) }
\\
\speak{Student}{\bf When did the overthrow occur? }
\speak{Teacher}\colorbox{pink!25}{$\hookrightarrow$}
{ ``1917,'' (1917, ) }
\\
\speak{Student}{\bf Who succeeded Lenin? }
\speak{Teacher}\colorbox{pink!25}{$\hookrightarrow$}
{ ``Joseph Stalin'' (Joseph Stalin ) }
\\
\speak{Student}{\bf When? }
\speak{Teacher}\colorbox{pink!25}{$\hookrightarrow$}
{ ``the mid-1920s.'' (mid-1920s. ) }
\\
\speak{Student}{\bf When was the Great Purge? }
\speak{Teacher}\colorbox{pink!25}{$\hookrightarrow$}
{ ``the mid-1930s'' (mid-1930s ) }
\\
\speak{Student}{\bf True or False: Stalin tolerated political criticism. }
\speak{Teacher}\colorbox{pink!25}{$\hookrightarrow$}
{ ``false'' (of ) }
 \end{dialogue}\end{tcolorbox}\end{figure}\begin{figure}[t] \small \begin{tcolorbox}[boxsep=0pt,left=5pt,right=0pt,top=2pt,colback = yellow!5] \begin{dialogue}
 \small 
 \speak{Student}{\bf What did he cause in 1933? }
\speak{Teacher}\colorbox{pink!25}{$\hookrightarrow$}
{ ``a major famine'' (major famine ) }
\\
\speak{Student}{\bf Where? }
\speak{Teacher}\colorbox{pink!25}{$\hookrightarrow$}
{ ``Soviet Ukraine,'' (Soviet Ukraine, ) }
\\
\speak{Student}{\bf For how many deaths there was he basically responsible? }
\speak{Teacher}\colorbox{pink!25}{$\hookrightarrow$}
{ ``f over 7 million people.'' (over 7 million people. ) }
\\
\speak{Student}{\bf Did the Soviet Union have nuclear weapons? }
\speak{Teacher}\colorbox{pink!25}{$\hookrightarrow$}
\colorbox{red!25}{Yes,}
{ ``yes'' (over 7 million people. ) }
\\
\speak{Student}{\bf How many other states had them? }
\speak{Teacher}\colorbox{pink!25}{$\hookrightarrow$}
{ ``four others'' (of ) }
\\
\speak{Student}{\bf What does USSR stand for? }
\speak{Teacher}\colorbox{pink!25}{$\hookrightarrow$}
{ ``Union of Soviet Socialist Republics'' (Union of Soviet Socialist Republics ) }
\\
\speak{Student}{\bf What was the Russian version of this acronym? }
\speak{Teacher}\colorbox{pink!25}{$\hookrightarrow$}
{ ``СССР'' (СССР ) }
\\
\speak{Student}{\bf True or False: The USSR was a two-party state. }
\speak{Teacher}\colorbox{pink!25}{$\hookrightarrow$}
{ ``false'' (t ) }
 \end{dialogue}\end{tcolorbox}\end{figure}\begin{figure}[t] \small \begin{tcolorbox}[boxsep=0pt,left=5pt,right=0pt,top=2pt,colback = yellow!5] \begin{dialogue}
 \small 
 \speak{Student}{\bf What party did they have? }
\speak{Teacher}\colorbox{pink!25}{$\hookrightarrow$}
{ ``governed by the Communist Party'' (governed by the Communist Party ) }
\\
\speak{Student}{\bf What was the capital of the USSR? }
\speak{Teacher}\colorbox{pink!25}{$\hookrightarrow$}
{ ``Moscow'' (Moscow ) }
\\
\speak{Student}{\bf Name another major city of the USSR. }
\speak{Teacher}\colorbox{pink!25}{$\hookrightarrow$}
{ ``Minsk'' (Minsk, ) }
\\
\speak{Student}{\bf And another? }
\speak{Teacher}\colorbox{pink!25}{$\hookrightarrow$}
{ ``Leningrad'' (Leningrad, ) }
\\
 \end{dialogue}\end{tcolorbox}\end{figure}

\end{document}

