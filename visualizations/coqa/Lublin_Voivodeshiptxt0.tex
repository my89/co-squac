\documentclass[11pt,a4paper, onecolumn]{article}
\usepackage{times}
\usepackage{latexsym}
\usepackage{url}
\usepackage{textcomp}
\usepackage{bbm}
\usepackage{amsmath}
\usepackage{booktabs}
\usepackage{tabularx}
\usepackage{graphicx}
\usepackage{dialogue}
\usepackage{mathtools}
\usepackage{hyperref}
%\hypersetup{draft}

\usepackage{multirow}
\usepackage{mdframed}
\usepackage{tcolorbox}

\usepackage{xcolor,pifont}
%\newcommand{\cmark}{\ding{51}}
%\newcommand{\xmark}{\ding{55}}

\setcounter{topnumber}{2}
\setcounter{bottomnumber}{2}
\setcounter{totalnumber}{4}
\renewcommand{\topfraction}{0.75}
\renewcommand{\bottomfraction}{0.75}
\renewcommand{\textfraction}{0.05}
\renewcommand{\floatpagefraction}{0.6}

\newcommand\cmark {\textcolor{green}{\ding{52}}}
\newcommand\xmark {\textcolor{red}{\ding{55}}}
\mdfdefinestyle{dialogue}{
    backgroundcolor=yellow!20,
    innermargin=5pt
}
\usepackage{amssymb}
\usepackage{soul}
\makeatletter

\begin{document}

\hspace{15pt}{\textbf{Section}:Lublin Voivodeship.txt0\\}
\\ Context: Lublin Voivodeship, or Lublin Province (in Polish, ''województwo lubelskie'' ), is a voivodeship, or province, located in southeastern Poland. It was created on January 1, 1999, out of the former Lublin, Chełm, Zamość, Biała Podlaska and (partially) Tarnobrzeg and Siedlce Voivodeships, pursuant to Polish local government reforms adopted in 1998. The province is named after its largest city and regional capital, Lublin, and its territory is made of four historical lands: the western part of the voivodeship, with Lublin itself, belongs to Lesser Poland, the eastern part of Lublin Area belongs to Red Ruthenia, and the northeast belongs to Polesie and Podlasie. Lublin Voivodeship is bordered by Subcarpathian Voivodeship to the south, Świętokrzyskie Voivodeship to the south-west, Masovian Voivodeship to the west and north, Podlaskie Voivodeship along a short boundary to the north, and Belarus and Ukraine to the east. The province's population as of 2006 was 2,175,251. It covers an area of . The that encompasses Lublin, and approximates Lublin Voivodeship as it was before the Partitions of Poland, is known as ''Lubelszczyzna''. Provinces centred on Lublin have existed throughout much of Poland's history; for details see the section below on Previous Lublin Voivodeships. The region was, before World War II, one of the world's leading centres of Judaism. Before the middle of the 16th century, there were few Jews in the area, concentrated in Lublin, Kazimierz Dolny, and perhaps Chełm; but the founding of new private towns led to a large movement of Jews into the region to develop trade and services. Since these new towns competed with the existing towns for business, there followed a low-intensity, long-lasting feeling of resentment, with failed attempts to limit the Jewish immigration. The Jews tended to settle mostly in the cities and towns, with only individual families setting up businesses in the rural regions; this urban/rural division became another factor feeding resentment of the newly arrived economic competitors. By the middle of the 18th century, Jews were a significant part of the population in Kraśnik, Lubartów and Łęczna. CANNOTANSWER

\begin{figure}[t] \small \begin{tcolorbox}[boxsep=0pt,left=5pt,right=0pt,top=2pt,colback = yellow!5] \begin{dialogue}
 \small 
 \speak{Student}{\bf What is the location that is the focus of the article? }
\speak{Teacher}\colorbox{pink!25}{$\hookrightarrow$}
{ ``Lublin Voivodeship'' (Lublin Voivodeship ) }
\\
\speak{Student}{\bf What is on its northern border? }
\speak{Teacher}\colorbox{pink!25}{$\hookrightarrow$}
{ ``Podlaskie Voivodeship'' (Podlaskie Voivodeship ) }
\\
\speak{Student}{\bf What is to its east? }
\speak{Teacher}\colorbox{pink!25}{$\hookrightarrow$}
{ ``Belarus and Ukraine'' (Belarus and Ukraine ) }
\\
\speak{Student}{\bf To its northwest? }
\speak{Teacher}\colorbox{pink!25}{$\hookrightarrow$}
{ ``Masovian Voivodeship'' (Masovian Voivodeship ) }
\\
\speak{Student}{\bf To the southwest? }
\speak{Teacher}\colorbox{pink!25}{$\hookrightarrow$}
{ ``Świętokrzyskie Voivodeship'' (Świętokrzyskie Voivodeship ) }
\\
\speak{Student}{\bf How about to the south? }
\speak{Teacher}\colorbox{pink!25}{$\hookrightarrow$}
{ ``Subcarpathian Voivodeship'' (Subcarpathian Voivodeship ) }
\\
\speak{Student}{\bf What country is it in? }
\speak{Teacher}\colorbox{pink!25}{$\hookrightarrow$}
{ ``Poland'' (Poland ) }
\\
\speak{Student}{\bf When was it formed? }
\speak{Teacher}\colorbox{pink!25}{$\hookrightarrow$}
{ ``1999'' (1999 ) }
 \end{dialogue}\end{tcolorbox}\end{figure}\begin{figure}[t] \small \begin{tcolorbox}[boxsep=0pt,left=5pt,right=0pt,top=2pt,colback = yellow!5] \begin{dialogue}
 \small 
 \speak{Student}{\bf From which previous entities? }
\speak{Teacher}\colorbox{pink!25}{$\hookrightarrow$}
{ ``out of the former Lublin, Chełm, Zamość, Biała Podlaska and (partially) Tarnobrzeg and Siedlce Voivodeships'' (out of the former Lublin, Chełm, Zamość, Biała Podlaska and (partially) Tarnobrzeg and Siedlce Voivodeships ) }
\\
\speak{Student}{\bf What is its  name derived from? }
\speak{Teacher}\colorbox{pink!25}{$\hookrightarrow$}
{ ``after its largest city and regional capital'' (after its largest city and regional capital ) }
\\
\speak{Student}{\bf Who does the western portion belong to? }
\speak{Teacher}\colorbox{pink!25}{$\hookrightarrow$}
{ ``Lesser Poland'' (Lesser Poland ) }
\\
\speak{Student}{\bf The northeast part? }
\speak{Teacher}\colorbox{pink!25}{$\hookrightarrow$}
{ ``Polesie and Podlasie'' (Polesie and Podlasie ) }
\\
\speak{Student}{\bf The eastern part? }
\speak{Teacher}\colorbox{pink!25}{$\hookrightarrow$}
{ ``Red Ruthenia'' (Red Ruthenia ) }
\\
\speak{Student}{\bf How many people live in the area? }
\speak{Teacher}\colorbox{pink!25}{$\hookrightarrow$}
{ ``2,175,251'' (2,175,251 ) }
\\
\speak{Student}{\bf As of what year? }
\speak{Teacher}\colorbox{pink!25}{$\hookrightarrow$}
{ ``2006'' (2006 ) }
\\
\speak{Student}{\bf What religion did it used to be an important location for? }
\speak{Teacher}\colorbox{pink!25}{$\hookrightarrow$}
{ ``one of the world's leading centres of Judaism'' (one of the world's leading centres of Judaism ) }
 \end{dialogue}\end{tcolorbox}\end{figure}\begin{figure}[t] \small \begin{tcolorbox}[boxsep=0pt,left=5pt,right=0pt,top=2pt,colback = yellow!5] \begin{dialogue}
 \small 
 \speak{Student}{\bf Since when? }
\speak{Teacher}\colorbox{pink!25}{$\hookrightarrow$}
{ ``By the middle of the 18th century,'' (By the middle of the 18th century, ) }
\\
 \end{dialogue}\end{tcolorbox}\end{figure}

\end{document}

