\documentclass[11pt,a4paper, onecolumn]{article}
\usepackage{times}
\usepackage{latexsym}
\usepackage{url}
\usepackage{textcomp}
\usepackage{bbm}
\usepackage{amsmath}
\usepackage{booktabs}
\usepackage{tabularx}
\usepackage{graphicx}
\usepackage{dialogue}
\usepackage{mathtools}
\usepackage{hyperref}
%\hypersetup{draft}

\usepackage{multirow}
\usepackage{mdframed}
\usepackage{tcolorbox}

\usepackage{xcolor,pifont}
%\newcommand{\cmark}{\ding{51}}
%\newcommand{\xmark}{\ding{55}}

\setcounter{topnumber}{2}
\setcounter{bottomnumber}{2}
\setcounter{totalnumber}{4}
\renewcommand{\topfraction}{0.75}
\renewcommand{\bottomfraction}{0.75}
\renewcommand{\textfraction}{0.05}
\renewcommand{\floatpagefraction}{0.6}

\newcommand\cmark {\textcolor{green}{\ding{52}}}
\newcommand\xmark {\textcolor{red}{\ding{55}}}
\mdfdefinestyle{dialogue}{
    backgroundcolor=yellow!20,
    innermargin=5pt
}
\usepackage{amssymb}
\usepackage{soul}
\makeatletter

\begin{document}

\hspace{15pt}{\textbf{Section}:El Gran Combo de Puerto Rico -- The 1970s0\\}
\\ Context: In 1970, El Gran Combo's contract with Gema Records was not renewed. Despite offers from other record companies including the Motown label, the group decided to self-release recordings under their own newly created independent label, Combo Records (alternatively known as EGC Records). The first album released on the label was the 1970 album entitled Estamos Primeros. On February 15, 1970, the members of El Gran Combo shared a near death experience. They were at Venezuela, and scheduled to fly the following day to Las Americas International Airport in Santo Domingo. After landing they were told about the Dominicana Airlines DC-9 that crashed off the Caribbean coast which occurred the night before. Therefore, the tale about a group member having a bad feeling regarding that flight while stranded at the Santo Domingo airport is not entirely true, since they did not arrive there until the day after the crash. In 1971, El Gran Combo introduced the trombone to their instrument mix. The trombone was played by Fanny Ceballos. Soon after, their production named De Punta a Punta was released. In 1972, they released the album ''Por el Libro'', which marks the 10th anniversary of the orchestra. Pellin Rodriguez left the group to embark on a solo career. Rodriguez was replaced by Charlie Aponte at the recommendation of Jerry Concepcion and the well known sportscaster Rafael Bracero, both friends of Ithier. In 1973, El Gran Combo sang in front of 50,000 fans at the famous Yankee Stadium in New York City as the opening act for the Fania All-Stars' sold out concert. Montanez left the band in early 1977 and went to live in Venezuela where he replaced Oscar D'Leon in another orchestra, Dimension Latina. Jerry Rivas was then chosen to join the orchestra. Both Rivas and Aponte are still members of the orchestra to this day. The success of this new duo was proved with their 1977 album International and 1978's En Las Vegas which reached gold record status. In 1966, En Navidad, a Christmas album, was released, with Martin Quinones appearing as Santa Claus in the album's cover. After an automobile accident in early 1977, Quinones was replaced in the band by his son, Martin Quinones Jr. He stayed until 1979, being replaced by Luis Diaz. CANNOTANSWER

\begin{figure}[t] \small \begin{tcolorbox}[boxsep=0pt,left=5pt,right=0pt,top=2pt,colback = yellow!5] \begin{dialogue}
 \small 
 \speak{Student}{\bf What happened in the 1970s? }
\speak{Teacher}\colorbox{pink!25}{$\hookrightarrow$}
{ In 1970, El Gran Combo's contract with Gema Records was not renewed. Despite offers from other record companies }
\\
\speak{Student}{\bf What did they do after that? }
\speak{Teacher}\colorbox{pink!25}{$\hookrightarrow$}
{ the group decided to self-release recordings under their own newly created independent label, Combo Records }
\\
\speak{Student}{\bf Were they successful under the new label? }
\speak{Teacher}\colorbox{pink!25}{ $\bar{\hookrightarrow}$}
{ CANNOTANSWER }
\\
\speak{Student}{\bf What did they release under Combo Records? }
\speak{Teacher}\colorbox{pink!25}{$\hookrightarrow$}
{ the 1970 album entitled Estamos Primeros. On February 15, 1970, the members of El Gran Combo shared a near death experience. }
\\
\speak{Student}{\bf What was the near death experience? }
\speak{Teacher}\colorbox{pink!25}{ $\bar{\hookrightarrow}$}
{ scheduled to fly the following day to Las Americas International Airport in Santo Domingo. After landing they were told about the Dominicana Airlines DC-9 that crashed }
\\
\speak{Student}{\bf Did this affect their outlook on life? }
\speak{Teacher}\colorbox{pink!25}{ $\bar{\hookrightarrow}$}
{ CANNOTANSWER }
\\
\speak{Student}{\bf Did they release any other albums? }
\speak{Teacher}\colorbox{pink!25}{ $\bar{\hookrightarrow}$}
{ En Navidad, a Christmas album, was released, with Martin Quinones appearing as }
\\
\speak{Student}{\bf Was En Navidad a popular album? }
\speak{Teacher}\colorbox{pink!25}{$\not\hookrightarrow$}
{ CANNOTANSWER }
 \end{dialogue}\end{tcolorbox}\end{figure}

\end{document}

