\documentclass[11pt,a4paper, onecolumn]{article}
\usepackage{times}
\usepackage{latexsym}
\usepackage{url}
\usepackage{textcomp}
\usepackage{bbm}
\usepackage{amsmath}
\usepackage{booktabs}
\usepackage{tabularx}
\usepackage{graphicx}
\usepackage{dialogue}
\usepackage{mathtools}
\usepackage{hyperref}
%\hypersetup{draft}

\usepackage{multirow}
\usepackage{mdframed}
\usepackage{tcolorbox}

\usepackage{xcolor,pifont}
%\newcommand{\cmark}{\ding{51}}
%\newcommand{\xmark}{\ding{55}}

\setcounter{topnumber}{2}
\setcounter{bottomnumber}{2}
\setcounter{totalnumber}{4}
\renewcommand{\topfraction}{0.75}
\renewcommand{\bottomfraction}{0.75}
\renewcommand{\textfraction}{0.05}
\renewcommand{\floatpagefraction}{0.6}

\newcommand\cmark {\textcolor{green}{\ding{52}}}
\newcommand\xmark {\textcolor{red}{\ding{55}}}
\mdfdefinestyle{dialogue}{
    backgroundcolor=yellow!20,
    innermargin=5pt
}
\usepackage{amssymb}
\usepackage{soul}
\makeatletter

\begin{document}

\hspace{15pt}{\textbf{Section}:Pinky and the Brain -- Pinky0\\}
\\ Context: Pinky (voiced by Rob Paulsen) is another genetically modified mouse who shares the same cage as Brain at Acme Labs. Although intelligent in his own right, Pinky is an extremely unstable and hyperactive mouse. He has several verbal tics, such as ''narf'', ''zort'', ''poit'', and ''troz'' (the last of which he started saying after noticing it was ''zort in the mirror''). Pinky's appearance is the complete opposite of Brain's--while Brain is short, has a crooked tail and pink eyes, and speaks in a deeper, more eloquent manner, Pinky has a straight tail, blue eyes, and a severe overbite, is taller than the Brain, and speaks in a higher pitched voice with a Cockney accent. Pinky's name was inadvertently given to him by Brain himself: when insulting the two scientists responsible for their gene splicing while talking to himself, Brain claimed the scientists had ''less knowledge in both their heads than I do in my... pinky!'' Pinky then responded with ''Yes?'', believing Brain was referring to him. Pinky is more open-minded, kinder, and happier than the Brain. Troubles never ruin his day, arguably because he is too scatter-brained to notice them. He steadfastly helps Brain toward world domination, even though Brain usually berates, belittles, and abuses him. Pinky actually seems to enjoy this, laughing after he is hit. He is obsessed with trivia, spending a lot of time watching television in the lab and following popular culture fads. Sometimes Pinky even finds non-rational solutions to problems. An entire episode (entitled ''The Pinky P.O.V.'') even shows a typical night of attempted world domination from his point of view, showing his thought process and how he comes to the strange, seemingly nonsensical responses to the Brain's famous question, ''Are you pondering what I'm pondering?'' Pinky often points out flaws in the Brain's plans, which the Brain consistently ignores. The issues Pinky brings up can ironically lead to the downfall of the given night's plot. He is also arguably Brain's moral compass and only real friend. When Pinky sold his soul to get Brain the world in ''A Pinky and the Brain Halloween'', Brain saved him because he missed him and the world was not worth ruling without him. Pinky also has shown signs of intelligence despite his supposed childish stupidity. In ''Welcome to the Jungle'', Pinky was able to survive using his instincts and become a leader to Brain, who, despite his intelligence, could not survive in the wild on his own. And in ''The Pink Candidate'', when Pinky became President, he later began citing various constitutional amendments and legal problems that would bar Brain from his latest plot to take over the world. When Brain attempted to pressure him into helping, Pinky refused, claiming that the plan ''goes against everything I've come to stand for.'' CANNOTANSWER

\begin{figure}[t] \small \begin{tcolorbox}[boxsep=0pt,left=5pt,right=0pt,top=2pt,colback = yellow!5] \begin{dialogue}
 \small 
 \speak{Student}{\bf Are there any other interesting aspects about this article? }
\speak{Teacher}\colorbox{pink!25}{$\hookrightarrow$}
\colorbox{red!25}{Yes,}
{ Pinky (voiced by Rob Paulsen) is another genetically modified mouse who shares the same cage as Brain }
\\
\speak{Student}{\bf When was it created }
\speak{Teacher}\colorbox{pink!25}{$\hookrightarrow$}
{ CANNOTANSWER }
\\
\speak{Student}{\bf Tell me something about Pinky }
\speak{Teacher}\colorbox{pink!25}{$\hookrightarrow$}
{ Pinky is an extremely unstable and hyperactive mouse. }
\\
\speak{Student}{\bf What is Brain }
\speak{Teacher}\colorbox{pink!25}{$\hookrightarrow$}
{ Brain is short, has a crooked tail and pink eyes, and speaks in a deeper, more eloquent manner, }
\\
\speak{Student}{\bf What do they have in common }
\speak{Teacher}\colorbox{pink!25}{$\hookrightarrow$}
{ shares the same cage }
\\
\speak{Student}{\bf What do they like to do }
\speak{Teacher}\colorbox{pink!25}{$\hookrightarrow$}
{ He steadfastly helps Brain toward world domination, even though Brain usually berates, belittles, and abuses him. }
\\
\speak{Student}{\bf How do he abuse him }
\speak{Teacher}\colorbox{pink!25}{$\hookrightarrow$}
{ He is also arguably Brain's moral compass and only real friend. }
\\
\speak{Student}{\bf What else is great in this topic }
\speak{Teacher}\colorbox{pink!25}{$\hookrightarrow$}
{ In ''Welcome to the Jungle'', Pinky was able to survive using his instincts and become a leader }
 \end{dialogue}\end{tcolorbox}\end{figure}\begin{figure}[t] \small \begin{tcolorbox}[boxsep=0pt,left=5pt,right=0pt,top=2pt,colback = yellow!5] \begin{dialogue}
 \small 
 \speak{Student}{\bf What year that came out }
\speak{Teacher}\colorbox{pink!25}{$\not\hookrightarrow$}
{ CANNOTANSWER }
\\
\speak{Student}{\bf When did pinky and the brain start }
\speak{Teacher}\colorbox{pink!25}{$\not\hookrightarrow$}
{ CANNOTANSWER }
\\
 \end{dialogue}\end{tcolorbox}\end{figure}

\end{document}

