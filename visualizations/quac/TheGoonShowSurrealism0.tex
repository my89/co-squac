\documentclass[11pt,a4paper, onecolumn]{article}
\usepackage{times}
\usepackage{latexsym}
\usepackage{url}
\usepackage{textcomp}
\usepackage{bbm}
\usepackage{amsmath}
\usepackage{booktabs}
\usepackage{tabularx}
\usepackage{graphicx}
\usepackage{dialogue}
\usepackage{mathtools}
\usepackage{hyperref}
%\hypersetup{draft}

\usepackage{multirow}
\usepackage{mdframed}
\usepackage{tcolorbox}

\usepackage{xcolor,pifont}
%\newcommand{\cmark}{\ding{51}}
%\newcommand{\xmark}{\ding{55}}

\setcounter{topnumber}{2}
\setcounter{bottomnumber}{2}
\setcounter{totalnumber}{4}
\renewcommand{\topfraction}{0.75}
\renewcommand{\bottomfraction}{0.75}
\renewcommand{\textfraction}{0.05}
\renewcommand{\floatpagefraction}{0.6}

\newcommand\cmark {\textcolor{green}{\ding{52}}}
\newcommand\xmark {\textcolor{red}{\ding{55}}}
\mdfdefinestyle{dialogue}{
    backgroundcolor=yellow!20,
    innermargin=5pt
}
\usepackage{amssymb}
\usepackage{soul}
\makeatletter

\begin{document}

\hspace{15pt}{\textbf{Section}:The Goon Show -- Surrealism0\\}
\\ Context: The Goon Show has been variously described as ''avant-garde'', ''surrealist'', ''abstract'', and ''four dimensional''. The show paved the way for surreal and alternative humour, as acknowledged by comedians such as Eddie Izzard. The surreality was part of the attraction for Sellers, and this exacerbated his mental instability especially during the third series. Many of the sequences have been cited as being visionary in the way that they challenged the traditional conventions of comedy. In the Pythons' autobiography, Terry Jones states ''The Goons of course were my favourite. It was the surreality of the imagery and the speed of the comedy that I loved - the way they broke up the conventions of radio and played with the very nature of the medium.'' This is reiterated by Michael Palin and John Cleese in their contributions to Ventham's (2002) book. Cleese recalls listening to The Goon Show as a teenager in the mid-1950s ''and being absolutely amazed by its surreal humour. It came at a key stage in my own development and I never missed a show''. Additionally, Milligan played games with the medium itself. Whole scenes were written in which characters would leave, close the door behind themselves, yet still be inside the room. Further to this, characters would announce their departure, slam a door, but it would be another character who had left the room. That character would then beat on the door for re-admittance, the door would open and close and again the wrong character would be locked out. CANNOTANSWER

\begin{figure}[t] \small \begin{tcolorbox}[boxsep=0pt,left=5pt,right=0pt,top=2pt,colback = yellow!5] \begin{dialogue}
 \small 
 \speak{Student}{\bf What was surrealist about The Goon Show? }
\speak{Teacher}\colorbox{pink!25}{$\hookrightarrow$}
{ The surreality was part of the attraction for Sellers, and this exacerbated his mental instability especially during the third series. }
\\
\speak{Student}{\bf Are there other examples of surrealism in the show? }
\speak{Teacher}\colorbox{pink!25}{$\hookrightarrow$}
\colorbox{red!25}{Yes,}
{ The show paved the way for surreal and alternative humour, as acknowledged by comedians such as Eddie Izzard. The }
\\
\speak{Student}{\bf What else made the show strange? }
\speak{Teacher}\colorbox{pink!25}{$\hookrightarrow$}
{ - the way they broke up the conventions of radio and played with the very nature of the medium. }
\\
\speak{Student}{\bf What did people say about the show? }
\speak{Teacher}\colorbox{pink!25}{$\hookrightarrow$}
{ It was the surreality of the imagery and the speed of the comedy that I loved - }
\\
\speak{Student}{\bf What else have people said about the show? }
\speak{Teacher}\colorbox{pink!25}{$\hookrightarrow$}
{ and being absolutely amazed by its surreal humour. }
\\
 \end{dialogue}\end{tcolorbox}\end{figure}

\end{document}

