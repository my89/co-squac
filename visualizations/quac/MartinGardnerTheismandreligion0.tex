\documentclass[11pt,a4paper, onecolumn]{article}
\usepackage{times}
\usepackage{latexsym}
\usepackage{url}
\usepackage{textcomp}
\usepackage{bbm}
\usepackage{amsmath}
\usepackage{booktabs}
\usepackage{tabularx}
\usepackage{graphicx}
\usepackage{dialogue}
\usepackage{mathtools}
\usepackage{hyperref}
%\hypersetup{draft}

\usepackage{multirow}
\usepackage{mdframed}
\usepackage{tcolorbox}

\usepackage{xcolor,pifont}
%\newcommand{\cmark}{\ding{51}}
%\newcommand{\xmark}{\ding{55}}

\setcounter{topnumber}{2}
\setcounter{bottomnumber}{2}
\setcounter{totalnumber}{4}
\renewcommand{\topfraction}{0.75}
\renewcommand{\bottomfraction}{0.75}
\renewcommand{\textfraction}{0.05}
\renewcommand{\floatpagefraction}{0.6}

\newcommand\cmark {\textcolor{green}{\ding{52}}}
\newcommand\xmark {\textcolor{red}{\ding{55}}}
\mdfdefinestyle{dialogue}{
    backgroundcolor=yellow!20,
    innermargin=5pt
}
\usepackage{amssymb}
\usepackage{soul}
\makeatletter

\begin{document}

\hspace{15pt}{\textbf{Section}:Martin Gardner -- Theism and religion0\\}
\\ Context: Gardner believed in a personal God, in an afterlife, and in prayer, but rejected established religion. He considered himself a philosophical theist and a fideist. He had an abiding fascination with religious belief but was critical of organized religion. In his autobiography, he stated: ''When many of my fans discovered that I believed in God and even hoped for an afterlife, they were shocked and dismayed... I do not mean the God of the Bible, especially the God of the Old Testament, or any other book that claims to be divinely inspired. For me God is a ''Wholly Other'' transcendent intelligence, impossible for us to understand. He or she is somehow responsible for our universe and capable of providing, how I have no inkling, an afterlife.'' Gardner described his own belief as philosophical theism inspired by the works of philosopher Miguel de Unamuno. While eschewing systematic religious doctrine, he retained a belief in God, asserting that this belief cannot be confirmed or disconfirmed by reason or science. At the same time, he was skeptical of claims that any god has communicated with human beings through spoken or telepathic revelation or through miracles in the natural world. Gardner has been quoted as saying that he regarded parapsychology and other research into the paranormal as tantamount to ''tempting God'' and seeking ''signs and wonders''. He stated that while he would expect tests on the efficacy of prayers to be negative, he would not rule out a priori the possibility that as yet unknown paranormal forces may allow prayers to influence the physical world. Gardner wrote repeatedly about what public figures such as Robert Maynard Hutchins, Mortimer Adler, and William F. Buckley, Jr. believed and whether their beliefs were logically consistent. In some cases, he attacked prominent religious figures such as Mary Baker Eddy on the grounds that their claims are unsupportable. His semi-autobiographical novel The Flight of Peter Fromm depicts a traditionally Protestant Christian man struggling with his faith, examining 20th century scholarship and intellectual movements and ultimately rejecting Christianity while remaining a theist. Gardner said that he suspected that the fundamental nature of human consciousness may not be knowable or discoverable, unless perhaps a physics more profound than (''underlying'') quantum mechanics is some day developed. In this regard, he said, he was an adherent of the ''New Mysterianism''. CANNOTANSWER

\begin{figure}[t] \small \begin{tcolorbox}[boxsep=0pt,left=5pt,right=0pt,top=2pt,colback = yellow!5] \begin{dialogue}
 \small 
 \speak{Student}{\bf What is his relation with Theism and religion? }
\speak{Teacher}\colorbox{pink!25}{$\hookrightarrow$}
{ Gardner believed in a personal God, in an afterlife, and in prayer, but rejected established religion. }
\\
\speak{Student}{\bf What does he believe in if no religion? }
\speak{Teacher}\colorbox{pink!25}{$\hookrightarrow$}
{ He had an abiding fascination with religious belief but was critical of organized religion. }
\\
\speak{Student}{\bf Why was he critical of organized religion? }
\speak{Teacher}\colorbox{pink!25}{$\hookrightarrow$}
{ While eschewing systematic religious doctrine, he retained a belief in God, asserting that this belief cannot be confirmed or disconfirmed by reason or science. }
\\
\speak{Student}{\bf What other things does he believe about religion? }
\speak{Teacher}\colorbox{pink!25}{$\hookrightarrow$}
{ he was skeptical of claims that any god has communicated with human beings through spoken or telepathic revelation or through miracles in the natural world. }
\\
\speak{Student}{\bf What conclusion can he draw from there believes? }
\speak{Teacher}\colorbox{pink!25}{$\hookrightarrow$}
{ he regarded parapsychology and other research into the paranormal as tantamount to ''tempting God'' and seeking ''signs and wonders''. }
\\
\speak{Student}{\bf Did he make other statement related to this? }
\speak{Teacher}\colorbox{pink!25}{$\hookrightarrow$}
{ negative, he would not rule out a priori the possibility that as yet unknown paranormal forces may allow prayers to influence the physical world. }
\\
\speak{Student}{\bf Did he write any book at all? }
\speak{Teacher}\colorbox{pink!25}{$\hookrightarrow$}
\colorbox{red!25}{Yes,}
{ His semi-autobiographical novel The Flight of Peter Fromm }
\\
\speak{Student}{\bf What was the book about? }
\speak{Teacher}\colorbox{pink!25}{$\hookrightarrow$}
{ depicts a traditionally Protestant Christian man struggling with his faith, examining 20th century scholarship and intellectual movements and ultimately rejecting Christianity while remaining a theist. }
 \end{dialogue}\end{tcolorbox}\end{figure}\begin{figure}[t] \small \begin{tcolorbox}[boxsep=0pt,left=5pt,right=0pt,top=2pt,colback = yellow!5] \begin{dialogue}
 \small 
 \speak{Student}{\bf Did he write any other book apart from this one? }
\speak{Teacher}\colorbox{pink!25}{$\not\hookrightarrow$}
{ CANNOTANSWER }
\\
 \end{dialogue}\end{tcolorbox}\end{figure}

\end{document}

