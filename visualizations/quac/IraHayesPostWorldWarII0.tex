\documentclass[11pt,a4paper, onecolumn]{article}
\usepackage{times}
\usepackage{latexsym}
\usepackage{url}
\usepackage{textcomp}
\usepackage{bbm}
\usepackage{amsmath}
\usepackage{booktabs}
\usepackage{tabularx}
\usepackage{graphicx}
\usepackage{dialogue}
\usepackage{mathtools}
\usepackage{hyperref}
%\hypersetup{draft}

\usepackage{multirow}
\usepackage{mdframed}
\usepackage{tcolorbox}

\usepackage{xcolor,pifont}
%\newcommand{\cmark}{\ding{51}}
%\newcommand{\xmark}{\ding{55}}

\setcounter{topnumber}{2}
\setcounter{bottomnumber}{2}
\setcounter{totalnumber}{4}
\renewcommand{\topfraction}{0.75}
\renewcommand{\bottomfraction}{0.75}
\renewcommand{\textfraction}{0.05}
\renewcommand{\floatpagefraction}{0.6}

\newcommand\cmark {\textcolor{green}{\ding{52}}}
\newcommand\xmark {\textcolor{red}{\ding{55}}}
\mdfdefinestyle{dialogue}{
    backgroundcolor=yellow!20,
    innermargin=5pt
}
\usepackage{amssymb}
\usepackage{soul}
\makeatletter

\begin{document}

\hspace{15pt}{\textbf{Section}:Ira Hayes -- Post World War II0\\}
\\ Context: Hayes attempted to lead a normal civilian life after the war. ''I kept getting hundreds of letters. And people would drive through the reservation, walk up to me and ask, ''Are you the Indian who raised the flag on Iwo Jima?'' Although Hayes rarely spoke about the flag raising, he talked more generally about his service in the Marine Corps with great pride. Hayes seemed to be disturbed that Harlon Block was still being misrepresented publicly as ''Hank'' Hansen. One day in 1946, Hayes walked and hitchhiked 1,300 miles from the Gila River Indian Community in Arizona to Edward Frederick Block, Sr.'s farm in Weslaco, Texas, to reveal the truth to Block's family about their son Harlon being in the Rosenthal photograph. He was instrumental in having the mistaken second flag-raiser controversy resolved by the Marine Corps in January 1947. Block's family was grateful to Hayes, especially Harlon's mother. She said that she had known from the time she first saw the photo in the newspaper, that it was Harlon in the photo. Mrs. Block took what Hayes said and wrote to her congressman. In 1949, Hayes appeared briefly as himself in the film Sands of Iwo Jima, starring John Wayne. In the movie, Wayne handed the American flag to Gagnon, Hayes, and Bradley, who at the time were considered the three surviving second flag-raisers (the second flag that was raised on Mount Suribachi is used in the film and is handed directly to Gagnon). After this Hayes was unable to hold on to a steady job for a long period, as he had become alcoholic. He was arrested 52 times for alcohol intoxication in public at various places in the country, including Chicago in October 1953. Referring to his alcoholism, he once said: ''I was sick. I guess I was about to crack up thinking about all my good buddies. They were better men than me and they're not coming back. Much less back to the White House, like me.'' Hayes was sober while attending the Marine Corps War Memorial dedication on November 10, 1954 where he was lauded by President Dwight D. Eisenhower as a hero. A reporter there approached Hayes and asked him, ''How do you like the pomp and circumstance?'' Hayes hung his head and said, ''I don't.'' His disquiet about his unwanted fame and his subsequent post-war problems were first recounted in detail by the author William Bradford Huie in ''The Outsider,'' published in 1959 as part of his collection Wolf Whistle and Other Stories. The Outsider was filmed in 1961, was directed by World War II veteran turned film director Delbert Mann and starred Tony Curtis as Hayes. The 2006 film Flags of Our Fathers, directed by Clint Eastwood, suggests that Hayes suffered from post-traumatic stress disorder (PTSD). CANNOTANSWER

\begin{figure}[t] \small \begin{tcolorbox}[boxsep=0pt,left=5pt,right=0pt,top=2pt,colback = yellow!5] \begin{dialogue}
 \small 
 \speak{Student}{\bf What was Ira Hayes doing after the War? }
\speak{Teacher}\colorbox{pink!25}{$\hookrightarrow$}
{ Hayes attempted to lead a normal civilian life after the war. }
\\
\speak{Student}{\bf Did he have a family? }
\speak{Teacher}\colorbox{pink!25}{$\not\hookrightarrow$}
{ CANNOTANSWER }
\\
\speak{Student}{\bf Is there something notable that he does after the war? }
\speak{Teacher}\colorbox{pink!25}{$\hookrightarrow$}
\colorbox{red!25}{Yes,}
{ Hayes walked and hitchhiked 1,300 miles from the Gila River Indian Community in Arizona to Edward Frederick Block, Sr.'s farm in Weslaco, Texas, to reveal the truth }
\\
\speak{Student}{\bf What truth is he wanting to reveal? }
\speak{Teacher}\colorbox{pink!25}{$\hookrightarrow$}
{ to Block's family about their son Harlon being in the Rosenthal photograph. }
\\
\speak{Student}{\bf What can you tell me about that? }
\speak{Teacher}\colorbox{pink!25}{$\hookrightarrow$}
{ He was instrumental in having the mistaken second flag-raiser controversy resolved by the Marine Corps in January 1947. }
\\
\speak{Student}{\bf Was there anyone opposed to him in this? }
\speak{Teacher}\colorbox{pink!25}{$\not\hookrightarrow$}
{ CANNOTANSWER }
\\
\speak{Student}{\bf Are there any other interesting aspects about this article? }
\speak{Teacher}\colorbox{pink!25}{$\hookrightarrow$}
\colorbox{red!25}{Yes,}
{ Hayes was unable to hold on to a steady job for a long period, as he had become alcoholic. He was arrested 52 times }
\\
 \end{dialogue}\end{tcolorbox}\end{figure}

\end{document}

