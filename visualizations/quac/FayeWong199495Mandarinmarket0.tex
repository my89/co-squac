\documentclass[11pt,a4paper, onecolumn]{article}
\usepackage{times}
\usepackage{latexsym}
\usepackage{url}
\usepackage{textcomp}
\usepackage{bbm}
\usepackage{amsmath}
\usepackage{booktabs}
\usepackage{tabularx}
\usepackage{graphicx}
\usepackage{dialogue}
\usepackage{mathtools}
\usepackage{hyperref}
%\hypersetup{draft}

\usepackage{multirow}
\usepackage{mdframed}
\usepackage{tcolorbox}

\usepackage{xcolor,pifont}
%\newcommand{\cmark}{\ding{51}}
%\newcommand{\xmark}{\ding{55}}

\setcounter{topnumber}{2}
\setcounter{bottomnumber}{2}
\setcounter{totalnumber}{4}
\renewcommand{\topfraction}{0.75}
\renewcommand{\bottomfraction}{0.75}
\renewcommand{\textfraction}{0.05}
\renewcommand{\floatpagefraction}{0.6}

\newcommand\cmark {\textcolor{green}{\ding{52}}}
\newcommand\xmark {\textcolor{red}{\ding{55}}}
\mdfdefinestyle{dialogue}{
    backgroundcolor=yellow!20,
    innermargin=5pt
}
\usepackage{amssymb}
\usepackage{soul}
\makeatletter

\begin{document}

\hspace{15pt}{\textbf{Section}:Faye Wong -- 1994-95: Mandarin market0\\}
\\ Context: Besides two Cantonese albums in 1994, Wong released two other albums in Mandarin in Taiwan, Mystery (Mi ) and Sky (Tian Kong ). The runaway hit ''I'm Willing'' (Wo Yuan Yi ) in Mystery became her trademark hit in the Mandarin-speaking communities for years, and has been covered by other singers such as Gigi Leung, Sammi Cheng and Jay Chou. Sky was seen by fans as a successful amalgam of artistic experimentation and commercialism. While her hits in Hong Kong were noticeably alternative, her two Mandarin albums were more lyrical and traditional. Critics generally credit Taiwanese producer Yang Ming-huang for their success. Four best-selling albums in Cantonese and Mandarin, a record-breaking 18 consecutive concerts in Hong Kong, and a widely acclaimed film (Chungking Express) made Faye Wong the most eminent female Hong Kong singer in the mid-1990s. Meanwhile, her distaste for the profit-oriented HK entertainment industry became more and more apparent. She was frequently in touch with the rock circle in Beijing. Given her somewhat reticent and nonchalant personality, she would sometimes give terse, direct, and somewhat unexpected answers when asked personal questions by the HK media. In 1995, she released Decadent Sounds of Faye (Fei Mi Mi Zhi Yin ), a cover album of songs originally recorded by her idol Teresa Teng, one of the most revered Chinese singers of the 20th century. A duet with Teng was planned for the album, but unfortunately she died before this could be recorded. Decadent Sounds sold well despite initial negative criticism, and has come to be recognised as an example of imaginative covering by recent critics. In December, she released her Cantonese album Di-Dar which mixes an alternative yodelling style with a touch of Indian and Middle Eastern flavour. This album was a success, partly because it was so different from the mainstream Cantopop music, but--ironically--a couple of very traditional romantic songs topped the charts. CANNOTANSWER

\begin{figure}[t] \small \begin{tcolorbox}[boxsep=0pt,left=5pt,right=0pt,top=2pt,colback = yellow!5] \begin{dialogue}
 \small 
 \speak{Student}{\bf What was the Mandarin Market? }
\speak{Teacher}\colorbox{pink!25}{$\hookrightarrow$}
{ Besides two Cantonese albums in 1994, Wong released two other albums in Mandarin in Taiwan, }
\\
\speak{Student}{\bf What were those albums? }
\speak{Teacher}\colorbox{pink!25}{ $\bar{\hookrightarrow}$}
{ Mystery (Mi ) and Sky (Tian Kong }
\\
\speak{Student}{\bf Did those albums do well? }
\speak{Teacher}\colorbox{pink!25}{$\hookrightarrow$}
\colorbox{red!25}{Yes,}
{ The runaway hit ''I'm Willing'' (Wo Yuan Yi ) in Mystery became her trademark hit in the Mandarin-speaking communities for years, }
\\
\speak{Student}{\bf What else came of that? }
\speak{Teacher}\colorbox{pink!25}{$\hookrightarrow$}
{ Four best-selling albums in Cantonese and Mandarin, a record-breaking 18 consecutive concerts in Hong Kong, and a widely acclaimed film }
\\
\speak{Student}{\bf What was the film called? }
\speak{Teacher}\colorbox{pink!25}{ $\bar{\hookrightarrow}$}
{ Chungking Express) }
\\
\speak{Student}{\bf How was her film performance? }
\speak{Teacher}\colorbox{pink!25}{$\hookrightarrow$}
{ made Faye Wong the most eminent female Hong Kong singer in the mid-1990s. }
\\
 \end{dialogue}\end{tcolorbox}\end{figure}

\end{document}

