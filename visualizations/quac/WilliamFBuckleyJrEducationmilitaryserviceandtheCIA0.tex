\documentclass[11pt,a4paper, onecolumn]{article}
\usepackage{times}
\usepackage{latexsym}
\usepackage{url}
\usepackage{textcomp}
\usepackage{bbm}
\usepackage{amsmath}
\usepackage{booktabs}
\usepackage{tabularx}
\usepackage{graphicx}
\usepackage{dialogue}
\usepackage{mathtools}
\usepackage{hyperref}
%\hypersetup{draft}

\usepackage{multirow}
\usepackage{mdframed}
\usepackage{tcolorbox}

\usepackage{xcolor,pifont}
%\newcommand{\cmark}{\ding{51}}
%\newcommand{\xmark}{\ding{55}}

\setcounter{topnumber}{2}
\setcounter{bottomnumber}{2}
\setcounter{totalnumber}{4}
\renewcommand{\topfraction}{0.75}
\renewcommand{\bottomfraction}{0.75}
\renewcommand{\textfraction}{0.05}
\renewcommand{\floatpagefraction}{0.6}

\newcommand\cmark {\textcolor{green}{\ding{52}}}
\newcommand\xmark {\textcolor{red}{\ding{55}}}
\mdfdefinestyle{dialogue}{
    backgroundcolor=yellow!20,
    innermargin=5pt
}
\usepackage{amssymb}
\usepackage{soul}
\makeatletter

\begin{document}

\hspace{15pt}{\textbf{Section}:William F. Buckley Jr. -- Education, military service, and the CIA0\\}
\\ Context: Buckley was homeschooled through the 8th grade using the Calvert School of Baltimore's Homeschool Curriculum. Buckley attended the National Autonomous University of Mexico (or UNAM) in 1943. The following year upon his graduation from the U.S. Army Officer Candidate School, he was commissioned as a second lieutenant in the United States Army. In his book, Miles Gone By, he briefly recounts being a member of Franklin Roosevelt's honor guard upon the President's death. He served stateside throughout the war at Fort Benning, Georgia; Fort Gordon, Georgia; and Fort Sam Houston, Texas. With the end of World War II in 1945, he enrolled in Yale University, where he became a member of the secret Skull and Bones society and was a masterful debater. He was an active member of the Independent Party of the Yale Political Union, and also served as Chairman of the Yale Daily News and as an informer for the FBI. Buckley studied political science, history, and economics at Yale, graduating with honors in 1950. He excelled on the Yale Debate Team, and under the tutelage of Yale professor Rollin G. Osterweis, Buckley honed his acerbic style. In 1951, along with many other Ivy League alumni, Buckley was recruited into the Central Intelligence Agency (CIA); he served for two years including one year in Mexico City working on political action for E. Howard Hunt, who was later jailed for his part in The Watergate affair.These two officers remained lifelong friends. In a November 1, 2005, column for National Review, Buckley recounted that while he worked for the CIA, the only employee of the organization that he knew was Hunt, his immediate boss. While in Mexico, Buckley edited The Road to Yenan, a book by Peruvian author Eudocio Ravines. CANNOTANSWER

\begin{figure}[t] \small \begin{tcolorbox}[boxsep=0pt,left=5pt,right=0pt,top=2pt,colback = yellow!5] \begin{dialogue}
 \small 
 \speak{Student}{\bf Where did he go to school? }
\speak{Teacher}\colorbox{pink!25}{$\hookrightarrow$}
{ Buckley was homeschooled through the 8th grade using the Calvert School of Baltimore's Homeschool Curriculum. }
\\
\speak{Student}{\bf Where did he go to high school? }
\speak{Teacher}\colorbox{pink!25}{$\hookrightarrow$}
{ CANNOTANSWER }
\\
\speak{Student}{\bf When did he join the military? }
\speak{Teacher}\colorbox{pink!25}{$\hookrightarrow$}
{ 1943. }
\\
\speak{Student}{\bf How long did he serve in the military? }
\speak{Teacher}\colorbox{pink!25}{$\not\hookrightarrow$}
{ He served stateside throughout the war at Fort Benning, Georgia; Fort Gordon, Georgia; and Fort Sam Houston, Texas. }
\\
\speak{Student}{\bf When did he join the CIA? }
\speak{Teacher}\colorbox{pink!25}{$\hookrightarrow$}
{ In 1951, along with many other Ivy League alumni, Buckley was recruited into the Central Intelligence Agency (CIA); }
\\
\speak{Student}{\bf How long was he in the CIA? }
\speak{Teacher}\colorbox{pink!25}{$\hookrightarrow$}
{ he served for two years including one year in Mexico City working on political action for E. Howard Hunt, }
\\
\speak{Student}{\bf What did he do next? }
\speak{Teacher}\colorbox{pink!25}{$\hookrightarrow$}
{ In a November 1, 2005, column for National Review, Buckley recounted that while he worked for the CIA, }
\\
 \end{dialogue}\end{tcolorbox}\end{figure}

\end{document}

