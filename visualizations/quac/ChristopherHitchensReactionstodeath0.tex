\documentclass[11pt,a4paper, onecolumn]{article}
\usepackage{times}
\usepackage{latexsym}
\usepackage{url}
\usepackage{textcomp}
\usepackage{bbm}
\usepackage{amsmath}
\usepackage{booktabs}
\usepackage{tabularx}
\usepackage{graphicx}
\usepackage{dialogue}
\usepackage{mathtools}
\usepackage{hyperref}
%\hypersetup{draft}

\usepackage{multirow}
\usepackage{mdframed}
\usepackage{tcolorbox}

\usepackage{xcolor,pifont}
%\newcommand{\cmark}{\ding{51}}
%\newcommand{\xmark}{\ding{55}}

\setcounter{topnumber}{2}
\setcounter{bottomnumber}{2}
\setcounter{totalnumber}{4}
\renewcommand{\topfraction}{0.75}
\renewcommand{\bottomfraction}{0.75}
\renewcommand{\textfraction}{0.05}
\renewcommand{\floatpagefraction}{0.6}

\newcommand\cmark {\textcolor{green}{\ding{52}}}
\newcommand\xmark {\textcolor{red}{\ding{55}}}
\mdfdefinestyle{dialogue}{
    backgroundcolor=yellow!20,
    innermargin=5pt
}
\usepackage{amssymb}
\usepackage{soul}
\makeatletter

\begin{document}

\hspace{15pt}{\textbf{Section}:Christopher Hitchens -- Reactions to death0\\}
\\ Context: Former British prime minister Tony Blair said, ''Christopher Hitchens was a complete one-off, an amazing mixture of writer, journalist, polemicist, and unique character. He was fearless in the pursuit of truth and any cause in which he believed. And there was no belief he held that he did not advocate with passion, commitment, and brilliance. He was an extraordinary, compelling, and colourful human being whom it was a privilege to know.'' Richard Dawkins, a friend of Hitchens, said, ''I think he was one of the greatest orators of all time. He was a polymath, a wit, immensely knowledgeable, and a valiant fighter against all tyrants, including imaginary supernatural ones.'' American theoretical physicist and cosmologist Lawrence Krauss said, ''Christopher was a beacon of knowledge and light in a world that constantly threatens to extinguish both. He had the courage to accept the world for just what it is and not what he wanted it to be. That's the highest praise, I believe, one can give to any intellect. He understood that the universe doesn't care about our existence or welfare and he epitomized the realization that our lives have meaning only to the extent that we give them meaning.'' Bill Maher paid tribute to Hitchens on his show Real Time with Bill Maher, saying, ''We lost a hero of mine, a friend, and one of the great talk show guests of all time.'' Salman Rushdie and English comedian Stephen Fry paid tribute at the Christopher Hitchens Vanity Fair Memorial 2012. Three weeks before Hitchens's death, George Eaton of the New Statesman wrote, ''He is determined to ensure that he is not remembered simply as a 'lefty who turned right' or as a contrarian and provocateur. Throughout his career, he has retained a commitment to the Enlightenment values of reason, secularism and pluralism. His targets--Mother Teresa, Bill Clinton, Henry Kissinger, God--are chosen not at random, but rather because they have offended one or more of these principles. The tragedy of Hitchens' illness is that it came at a time when he enjoyed a larger audience than ever. The great polemicist is certain to be remembered, but, as he is increasingly aware, perhaps not as he would like.'' The Chronicle of Higher Education asked if Hitchens was the last public intellectual. In 2015, an annual prize of  50,000 was established in his honour for ''an author or journalist whose work reflects a commitment to free expression and inquiry, a range and depth of intellect, and a willingness to pursue the truth without regard to personal or professional consequence.'' CANNOTANSWER

\begin{figure}[t] \small \begin{tcolorbox}[boxsep=0pt,left=5pt,right=0pt,top=2pt,colback = yellow!5] \begin{dialogue}
 \small 
 \speak{Student}{\bf What is the reactions to death about? }
\speak{Teacher}\colorbox{pink!25}{$\hookrightarrow$}
{ Christopher was a beacon of knowledge and light }
\\
\speak{Student}{\bf Who was the one that said that about him? }
\speak{Teacher}\colorbox{pink!25}{$\hookrightarrow$}
{ American theoretical physicist and cosmologist Lawrence Krauss said, }
\\
\speak{Student}{\bf What was someone else's reaction to it? }
\speak{Teacher}\colorbox{pink!25}{$\hookrightarrow$}
{ Christopher Hitchens was a complete one-off, an amazing mixture of writer, journalist, polemicist, and unique character. }
\\
\speak{Student}{\bf How did he pass away? }
\speak{Teacher}\colorbox{pink!25}{$\not\hookrightarrow$}
{ CANNOTANSWER }
\\
\speak{Student}{\bf Are there any other interesting aspects about this article? }
\speak{Teacher}\colorbox{pink!25}{$\hookrightarrow$}
\colorbox{red!25}{Yes,}
{ In 2015, an annual prize of  50,000 was established in his honour }
\\
\speak{Student}{\bf Who does the prize money go to? }
\speak{Teacher}\colorbox{pink!25}{$\hookrightarrow$}
{ an author or journalist whose work reflects a commitment to free expression and inquiry, a range and depth of intellect, }
\\
\speak{Student}{\bf What is he most remembered for now that he has passed? }
\speak{Teacher}\colorbox{pink!25}{$\not\hookrightarrow$}
{ I think he was one of the greatest orators of all time. }
\\
\speak{Student}{\bf When did he pass away? }
\speak{Teacher}\colorbox{pink!25}{$\not\hookrightarrow$}
{ CANNOTANSWER }
 \end{dialogue}\end{tcolorbox}\end{figure}\begin{figure}[t] \small \begin{tcolorbox}[boxsep=0pt,left=5pt,right=0pt,top=2pt,colback = yellow!5] \begin{dialogue}
 \small 
 \speak{Student}{\bf Was there anything else that he was remembered for? }
\speak{Teacher}\colorbox{pink!25}{$\not\hookrightarrow$}
\colorbox{red!25}{Yes,}
{ He was an extraordinary, compelling, and colourful human being whom it was a privilege to know. }
\\
 \end{dialogue}\end{tcolorbox}\end{figure}

\end{document}

