\documentclass[11pt,a4paper, onecolumn]{article}
\usepackage{times}
\usepackage{latexsym}
\usepackage{url}
\usepackage{textcomp}
\usepackage{bbm}
\usepackage{amsmath}
\usepackage{booktabs}
\usepackage{tabularx}
\usepackage{graphicx}
\usepackage{dialogue}
\usepackage{mathtools}
\usepackage{hyperref}
%\hypersetup{draft}

\usepackage{multirow}
\usepackage{mdframed}
\usepackage{tcolorbox}

\usepackage{xcolor,pifont}
%\newcommand{\cmark}{\ding{51}}
%\newcommand{\xmark}{\ding{55}}

\setcounter{topnumber}{2}
\setcounter{bottomnumber}{2}
\setcounter{totalnumber}{4}
\renewcommand{\topfraction}{0.75}
\renewcommand{\bottomfraction}{0.75}
\renewcommand{\textfraction}{0.05}
\renewcommand{\floatpagefraction}{0.6}

\newcommand\cmark {\textcolor{green}{\ding{52}}}
\newcommand\xmark {\textcolor{red}{\ding{55}}}
\mdfdefinestyle{dialogue}{
    backgroundcolor=yellow!20,
    innermargin=5pt
}
\usepackage{amssymb}
\usepackage{soul}
\makeatletter

\begin{document}

\hspace{15pt}{\textbf{Section}:Kalākaua -- Early life and family0\\}
\\ Context: Kalakaua was born on November 16, 1836, to Caesar Kaluaiku Kapa`akea and Analea Keohokalole, in the grass hut compound, belonging to his maternal grandfather `Aikanaka, at the base of Punchbowl Crater in Honolulu on the island of O`ahu. Of the ali`i class of Hawaiian nobility, his family were considered collateral relations of the reigning House of Kamehameha sharing common descent from the 18th-century ali`i nui Keawe`ikekahiali`iokamoku. From his biological parents, he descended from Keaweaheulu and Kame`eiamoku, two of the five royal counselors of Kamehameha I during his conquest of the Kingdom of Hawaii. Kame`eiamoku, the grandfather of both his mother and father, was one of the royal twins alongside Kamanawa depicted on the Hawaiian coat of arms. However, Kalakaua and his siblings traced their high rank from their mother's line of descent, referring to themselves as members the ''Keawe-a-Heulu line'', although later historians would refer to the family as the House of Kalakaua. The second surviving child of a large family, his biological siblings included his elder brother James Kaliokalani, and younger siblings Lyda Kamaka`eha (later renamed Lili`uokalani), Anna Ka`iulani, Ka`imina`auao, Miriam Likelike and William Pitt Leleiohoku II. Given the name Kalakaua which translates into ''The Day [of] Battle'', the date of his birth coincided with the signing of the unequal treaty imposed by British Captain Lord Edward Russell of the Actaeon on Kamehameha III. He along with his siblings were hanai (informally adopted) to other family members in the Native Hawaiian tradition. Prior to birth, his parents had promised to give their child in hanai to Kuini Liliha, a high ranking chiefess and the widow of High Chief Boki. However, after he was born, Kuhina Nui (regent) Elizabeth Kina`u, who disliked Liliha, order his parents to give him to Ha`aheo Kaniu and her husband Keaweamahi Kinimaka instead. When Ha`aheo died in 1843 she bequeathed all her properties to him. After Ha`aheo's death, his guardianship was entrusted in his hanai father, who was a chief of lesser rank; he took Kalakaua to live in Lahaina. Kinimaka would later marry Pai, a subordinate Tahitian chiefess, who treated Kalakaua as her own until the birth of her own son. At the age of four, Kalakaua returned to O`ahu to live with his biological parents and to begin his education at the Chiefs' Children's School. At the school, Kalakaua became fluent in English and the Hawaiian language. After graduating from the Royal School, he studied law under Charles Coffin Harris, who later became Chief Justice of the Supreme Court of Hawaii. Kalakaua was briefly engaged to marry Princess Victoria Kamamalu, the younger sister of Kamehameha IV and Kamehameha V. However, the match was terminated when the princess decided to renew her on and off betrothal to her cousin William Charles Lunalilo. On December 8, 1863, Kalakaua married Kapi`olani in a quiet ceremony conducted by a minister of the Episcopal Church of Hawaii. The timing of the wedding was heavily criticized since it fell during the mourning period for King Kamehameha IV. A descendant of King Kaumuali`i of Kauai, Kapi`olani had been the widow aunt and lady-in-waiting of Kamehameha IV's wife Queen Emma. CANNOTANSWER

\begin{figure}[t] \small \begin{tcolorbox}[boxsep=0pt,left=5pt,right=0pt,top=2pt,colback = yellow!5] \begin{dialogue}
 \small 
 \speak{Student}{\bf Where was Kalakaua born? }
\speak{Teacher}\colorbox{pink!25}{$\hookrightarrow$}
{ in the grass hut compound, belonging to his maternal grandfather `Aikanaka, at the base of Punchbowl Crater in Honolulu on the island of O`ahu. }
\\
\speak{Student}{\bf In what  year was he born? }
\speak{Teacher}\colorbox{pink!25}{$\hookrightarrow$}
{ 1836, }
\\
\speak{Student}{\bf Where did he go to school? }
\speak{Teacher}\colorbox{pink!25}{$\hookrightarrow$}
{ CANNOTANSWER }
\\
\speak{Student}{\bf What school did he go to next? }
\speak{Teacher}\colorbox{pink!25}{$\hookrightarrow$}
{ After graduating from the Royal School, }
\\
\speak{Student}{\bf Where did he go after the Royal school? }
\speak{Teacher}\colorbox{pink!25}{$\not\hookrightarrow$}
{ he studied law under Charles Coffin Harris, who later became Chief Justice of the Supreme Court of Hawaii. }
\\
\speak{Student}{\bf Did he graduate from law school? }
\speak{Teacher}\colorbox{pink!25}{$\not\hookrightarrow$}
{ CANNOTANSWER }
\\
\speak{Student}{\bf What did he do after law school? }
\speak{Teacher}\colorbox{pink!25}{$\hookrightarrow$}
{ Kalakaua was briefly engaged to marry Princess Victoria Kamamalu, the younger sister of Kamehameha IV and Kamehameha V. }
\\
\speak{Student}{\bf Why did the engagement end? }
\speak{Teacher}\colorbox{pink!25}{$\hookrightarrow$}
{ the match was terminated when the princess decided to renew her on and off betrothal to her cousin William Charles Lunalilo. }
 \end{dialogue}\end{tcolorbox}\end{figure}

\end{document}

