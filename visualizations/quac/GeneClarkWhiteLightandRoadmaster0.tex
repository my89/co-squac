\documentclass[11pt,a4paper, onecolumn]{article}
\usepackage{times}
\usepackage{latexsym}
\usepackage{url}
\usepackage{textcomp}
\usepackage{bbm}
\usepackage{amsmath}
\usepackage{booktabs}
\usepackage{tabularx}
\usepackage{graphicx}
\usepackage{dialogue}
\usepackage{mathtools}
\usepackage{hyperref}
%\hypersetup{draft}

\usepackage{multirow}
\usepackage{mdframed}
\usepackage{tcolorbox}

\usepackage{xcolor,pifont}
%\newcommand{\cmark}{\ding{51}}
%\newcommand{\xmark}{\ding{55}}

\setcounter{topnumber}{2}
\setcounter{bottomnumber}{2}
\setcounter{totalnumber}{4}
\renewcommand{\topfraction}{0.75}
\renewcommand{\bottomfraction}{0.75}
\renewcommand{\textfraction}{0.05}
\renewcommand{\floatpagefraction}{0.6}

\newcommand\cmark {\textcolor{green}{\ding{52}}}
\newcommand\xmark {\textcolor{red}{\ding{55}}}
\mdfdefinestyle{dialogue}{
    backgroundcolor=yellow!20,
    innermargin=5pt
}
\usepackage{amssymb}
\usepackage{soul}
\makeatletter

\begin{document}

\hspace{15pt}{\textbf{Section}:Gene Clark -- White Light and Roadmaster0\\}
\\ Context: In 1971, Clark released his second solo album, White Light (the title was not on the cover sleeve, and thus some later reviewers mistakenly assumed that the title was Gene Clark). The album was produced by the American Indian guitarist Jesse Ed Davis, with whom Clark developed great rapport, partly due to their common ancestry. An intimate, poetic and mostly acoustic work supplemented by Davis's slide guitar, the album contained many introspective tracks, such as ''With Tomorrow'', ''Because of You'', ''Where My Love Lies Asleep'' and ''For a Spanish Guitar'' (which Bob Dylan supposedly hailed as one of the greatest songs ever written). All of the material was written by Clark, with the exception of ''Tears of Rage'', by Dylan and Richard Manuel. Launched to considerable critical acclaim, the album failed to gain commercial success, except in the Netherlands, where it was voted album of the year by rock music critics. Once more, modest promotion and Clark's refusal to undertake promotional touring adversely affected sales. In the spring of 1971, Clark was commissioned by Dennis Hopper to contribute the tracks ''American Dreamer'' and ''Outlaw Song'' to Hopper's film project American Dreamer. A rerecorded, longer version of the song ''American Dreamer'' was later used in the 1977 film The Farmer, along with an instrumental version of the same song plus ''Outside the Law (The Outlaw)'', a rerecording of ''Outlaw Song''. In 1972, Clark attempted to record a follow-up album. Progress was slow and expensive, and A&M terminated the project before completion. The resulting eight tracks, including ''Full Circle Song'' and ''In a Misty Morning'', along with those recorded with the Byrds in 1970 and 1971 (''She's the Kind of Girl'' and ''One in a Hundred'') and with the Flying Burrito Brothers (''Here Tonight''), were released in 1973 as Roadmaster in the Netherlands only. CANNOTANSWER

\begin{figure}[t] \small \begin{tcolorbox}[boxsep=0pt,left=5pt,right=0pt,top=2pt,colback = yellow!5] \begin{dialogue}
 \small 
 \speak{Student}{\bf What does the article say about white light? }
\speak{Teacher}\colorbox{pink!25}{$\hookrightarrow$}
{ In 1971, Clark released his second solo album, White Light }
\\
\speak{Student}{\bf Was the album a success? }
\speak{Teacher}\colorbox{pink!25}{$\hookrightarrow$}
\colorbox{red!25}{No,}
{ Launched to considerable critical acclaim, the album failed to gain commercial success, except in the Netherlands, }
\\
\speak{Student}{\bf What did clark do after the failed album? }
\speak{Teacher}\colorbox{pink!25}{$\hookrightarrow$}
{ In the spring of 1971, Clark was commissioned by Dennis Hopper to contribute the tracks ''American Dreamer'' and ''Outlaw Song'' to Hopper's film project American Dreamer. }
\\
\speak{Student}{\bf Who worked on White Light with Clark? }
\speak{Teacher}\colorbox{pink!25}{$\hookrightarrow$}
{ The album was produced by the American Indian guitarist Jesse Ed Davis, with whom Clark developed great rapport, partly due to their common ancestry. }
\\
\speak{Student}{\bf What songs were on the album? }
\speak{Teacher}\colorbox{pink!25}{$\hookrightarrow$}
{ the album contained many introspective tracks, such as ''With Tomorrow'', ''Because of You'', ''Where My Love Lies Asleep'' and ''For a Spanish Guitar'' ( }
\\
\speak{Student}{\bf Did he have any songs released in films? }
\speak{Teacher}\colorbox{pink!25}{$\hookrightarrow$}
\colorbox{red!25}{Yes,}
{ A rerecorded, longer version of the song ''American Dreamer'' was later used in the 1977 film The Farmer, }
\\
\speak{Student}{\bf What does the article say about RoadMaster? }
\speak{Teacher}\colorbox{pink!25}{$\hookrightarrow$}
{ In 1972, Clark attempted to record a follow-up album. Progress was slow and expensive, and A&M terminated the project before completion. }
\\
\speak{Student}{\bf Why did they terminate the project? }
\speak{Teacher}\colorbox{pink!25}{ $\bar{\hookrightarrow}$}
{ Progress was slow and expensive, }
 \end{dialogue}\end{tcolorbox}\end{figure}

\end{document}

