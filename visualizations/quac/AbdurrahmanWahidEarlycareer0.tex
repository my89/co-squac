\documentclass[11pt,a4paper, onecolumn]{article}
\usepackage{times}
\usepackage{latexsym}
\usepackage{url}
\usepackage{textcomp}
\usepackage{bbm}
\usepackage{amsmath}
\usepackage{booktabs}
\usepackage{tabularx}
\usepackage{graphicx}
\usepackage{dialogue}
\usepackage{mathtools}
\usepackage{hyperref}
%\hypersetup{draft}

\usepackage{multirow}
\usepackage{mdframed}
\usepackage{tcolorbox}

\usepackage{xcolor,pifont}
%\newcommand{\cmark}{\ding{51}}
%\newcommand{\xmark}{\ding{55}}

\setcounter{topnumber}{2}
\setcounter{bottomnumber}{2}
\setcounter{totalnumber}{4}
\renewcommand{\topfraction}{0.75}
\renewcommand{\bottomfraction}{0.75}
\renewcommand{\textfraction}{0.05}
\renewcommand{\floatpagefraction}{0.6}

\newcommand\cmark {\textcolor{green}{\ding{52}}}
\newcommand\xmark {\textcolor{red}{\ding{55}}}
\mdfdefinestyle{dialogue}{
    backgroundcolor=yellow!20,
    innermargin=5pt
}
\usepackage{amssymb}
\usepackage{soul}
\makeatletter

\begin{document}

\hspace{15pt}{\textbf{Section}:Abdurrahman Wahid -- Early career0\\}
\\ Context: Wahid returned to Jakarta expecting that in a year's time, he would be abroad again to study at McGill University in Canada. He kept himself busy by joining the Institute for Economic and Social Research, Education and Information (LP3ES), an organization which consisted of intellectuals with progressive Muslims and social-democratic views. LP3ES established the magazine Prisma and Wahid became one of the main contributors to the magazine. Whilst working as a contributor for LP3ES, he also conducted tours to pesantren and madrasah across Java. It was a time when pesantren were desperate to gain state funding by adopting state-endorsed curricula and Wahid was concerned that the traditional values of the pesantren were being damaged because of this change. He was also concerned with the poverty of the pesantren which he saw during his tours. At the same time as it was encouraging pesantren to adopt state-endorsed curricula, the Government was also encouraging pesantren as agents for change and to help assist the government in the economic development of Indonesia. It was at this time that Wahid finally decided to drop plans for overseas studies in favor of promoting the development of the pesantren. Wahid continued his career as a journalist, writing for the magazine Tempo and Kompas, a leading Indonesian newspaper. His articles were well received, and he began to develop a reputation as a social commentator. Wahid's popularity was such that at this time he was invited to give lectures and seminars, obliging him to travel back and forth between Jakarta and Jombang, where he now lived with his family. Despite having a successful career up to that point, Wahid still found it hard to make ends meet, and he worked to earn extra income by selling peanuts and delivering ice to be used for his wife's Es Lilin (popsicle) business. In 1974, he found extra employment in Jombang as a Muslim Legal Studies teacher at Pesantren Tambakberas and soon developed a good reputation. A year later, Wahid added to his workload as a teacher of Kitab al-Hikam, a classical text of sufism. In 1977, Wahid joined the Hasyim Asyari University as Dean of the Faculty of Islamic Beliefs and Practices. Again he excelled in his job and the University wanted to him to teach extra subjects such as pedagogy, sharia, and missiology. However, his excellence caused some resentment from within the ranks of university and he was blocked from teaching the subjects. Whilst undertaking all these ventures he also regularly delivered speeches during Ramadan to the Muslim community in Jombang. CANNOTANSWER

\begin{figure}[t] \small \begin{tcolorbox}[boxsep=0pt,left=5pt,right=0pt,top=2pt,colback = yellow!5] \begin{dialogue}
 \small 
 \speak{Student}{\bf What did Wahid's early career consist of? }
\speak{Teacher}\colorbox{pink!25}{ $\bar{\hookrightarrow}$}
{ He kept himself busy by joining the Institute for Economic and Social Research, Education and Information ( }
\\
\speak{Student}{\bf When did his early career start? }
\speak{Teacher}\colorbox{pink!25}{ $\bar{\hookrightarrow}$}
{ Jakarta }
\\
\speak{Student}{\bf What else did he do in his early career? }
\speak{Teacher}\colorbox{pink!25}{ $\bar{\hookrightarrow}$}
{ he also conducted tours to pesantren and madrasah across Java. }
\\
\speak{Student}{\bf Was there anything else notable about his early career? }
\speak{Teacher}\colorbox{pink!25}{ $\bar{\hookrightarrow}$}
\colorbox{red!25}{Yes,}
{ Wahid continued his career as a journalist, writing for the magazine Tempo and Kompas, }
\\
\speak{Student}{\bf Did he get any awards or recognition? }
\speak{Teacher}\colorbox{pink!25}{ $\bar{\hookrightarrow}$}
\colorbox{red!25}{Yes,}
{ His articles were well received, and he began to develop a reputation as a social commentator. }
\\
\speak{Student}{\bf When was this? }
\speak{Teacher}\colorbox{pink!25}{$\not\hookrightarrow$}
{ CANNOTANSWER }
\\
\speak{Student}{\bf Who or what else did he write for? }
\speak{Teacher}\colorbox{pink!25}{$\not\hookrightarrow$}
{ CANNOTANSWER }
\\
\speak{Student}{\bf Is there anything else interesting in your article? }
\speak{Teacher}\colorbox{pink!25}{ $\bar{\hookrightarrow}$}
\colorbox{red!25}{Yes,}
{ Despite having a successful career up to that point, Wahid still found it hard to make ends meet, }
 \end{dialogue}\end{tcolorbox}\end{figure}\begin{figure}[t] \small \begin{tcolorbox}[boxsep=0pt,left=5pt,right=0pt,top=2pt,colback = yellow!5] \begin{dialogue}
 \small 
 \speak{Student}{\bf What jobs did he do? }
\speak{Teacher}\colorbox{pink!25}{ $\bar{\hookrightarrow}$}
{ he worked to earn extra income by selling peanuts and delivering ice }
\\
 \end{dialogue}\end{tcolorbox}\end{figure}

\end{document}

