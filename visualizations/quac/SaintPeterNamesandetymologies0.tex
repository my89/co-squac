\documentclass[11pt,a4paper, onecolumn]{article}
\usepackage{times}
\usepackage{latexsym}
\usepackage{url}
\usepackage{textcomp}
\usepackage{bbm}
\usepackage{amsmath}
\usepackage{booktabs}
\usepackage{tabularx}
\usepackage{graphicx}
\usepackage{dialogue}
\usepackage{mathtools}
\usepackage{hyperref}
%\hypersetup{draft}

\usepackage{multirow}
\usepackage{mdframed}
\usepackage{tcolorbox}

\usepackage{xcolor,pifont}
%\newcommand{\cmark}{\ding{51}}
%\newcommand{\xmark}{\ding{55}}

\setcounter{topnumber}{2}
\setcounter{bottomnumber}{2}
\setcounter{totalnumber}{4}
\renewcommand{\topfraction}{0.75}
\renewcommand{\bottomfraction}{0.75}
\renewcommand{\textfraction}{0.05}
\renewcommand{\floatpagefraction}{0.6}

\newcommand\cmark {\textcolor{green}{\ding{52}}}
\newcommand\xmark {\textcolor{red}{\ding{55}}}
\mdfdefinestyle{dialogue}{
    backgroundcolor=yellow!20,
    innermargin=5pt
}
\usepackage{amssymb}
\usepackage{soul}
\makeatletter

\begin{document}

\hspace{15pt}{\textbf{Section}:Saint Peter -- Names and etymologies0\\}
\\ Context: Peter's original name, as indicated in the New Testament, was ''Simon'' (Simon Simon in Greek) or (only in Acts 15:14 and 2 Peter 1:1) ''Simeon'' (Sumeon in Greek). The Simon/Simeon variation has been explained as reflecting ''the well-known custom among Jews at the time of giving the name of a famous patriarch or personage of the Old Testament to a male child along with a similar sounding Greek/Roman name''. He was later given the name ke'ypa (Kepha) in Aramaic, which was rendered in Greek (by transliteration and the addition of a final sigma to make it a masculine word) as Kephas, whence Latin and English Cephas (9 occurrences in the New Testament); or (by translation with masculine termination) as Petros, whence Latin Petrus and English Peter (156 occurrences in the New Testament). The precise meaning of the Aramaic word is disputed, some saying that its usual meaning is ''rock'' or ''crag'', others saying that it means rather ''stone'' and, particularly in its application by Jesus to Simon, ''precious stone'' or ''jewel'', but most scholars agree that as a proper name it denotes a rough or tough character. Both meanings, ''stone'' (jewel or hewn stone) and ''rock'', are indicated in dictionaries of Aramaic and Syriac. Catholic theologian Rudolf Pesch argues that the Aramaic cepha means ''stone, ball, clump, clew'' and that ''rock'' is only a connotation; that in the Attic Greek petra denotes ''grown rock, rocky range, cliff, grotto''; and that petros means ''small stone, firestone, sling stone, moving boulder''. The combined name Simon Petros (Simon Peter) appears 19 times in the New Testament. In some Syriac documents he is called, in English translation, Simon Cephas. CANNOTANSWER

\begin{figure}[t] \small \begin{tcolorbox}[boxsep=0pt,left=5pt,right=0pt,top=2pt,colback = yellow!5] \begin{dialogue}
 \small 
 \speak{Student}{\bf what does the names and etymologies have to do with saint peter }
\speak{Teacher}\colorbox{pink!25}{$\hookrightarrow$}
{ Peter's original name, as indicated in the New Testament, was ''Simon'' (Simon Simon in Greek) or (only in Acts 15:14 and 2 Peter 1:1) ''Simeon'' ( }
\\
\speak{Student}{\bf what names did he use more }
\speak{Teacher}\colorbox{pink!25}{$\hookrightarrow$}
{ He was later given the name ke'ypa (Kepha) in Aramaic, which was rendered in Greek (by transliteration and the addition of a final sigma }
\\
\speak{Student}{\bf what does that name mean }
\speak{Teacher}\colorbox{pink!25}{$\hookrightarrow$}
{ which was rendered in Greek (by transliteration and the addition of a final sigma to make it a masculine word) as Kephas, }
\\
\speak{Student}{\bf what about the etymologies? }
\speak{Teacher}\colorbox{pink!25}{$\hookrightarrow$}
{ The combined name Simon Petros (Simon Peter) appears 19 times in the New Testament. In some Syriac documents he is called, in English translation, Simon Cephas. }
\\
 \end{dialogue}\end{tcolorbox}\end{figure}

\end{document}

